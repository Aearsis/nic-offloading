\chapter{The Demonstrator}
\label{app:doc}

The attached source code contains an implementation of the subsystem under
a working title Match-Action Tables, MAT for short. In this appendix, we would
like to document the source code from a high-level point of view. We do not
discuss implementation details, as they are documented in the source code.
Also, we do not discuss the functionality itself, as it closely follows the
model presented in Section \ref{chap:rfc}.

You will not find an implementation of a Linux kernel module in the archive.
Instead, we created an environment where the implementation looks like if it
was written as a kernel module -- we could say that the MAT subsystem is
written against a \emph{mocked} kernel. Most importantly, all of the code
actually runs in the userspace, even though it simulates a code being run in
both the kernel mode and the userspace.

\section{Usage}

The source code is written in the C language and follows the GNU89 standard.
No libraries are required to compile it. The code still uses some Linux
headers for convenience, we do not expect it to be compilable on other
platforms. Due to the target audience, we do not consider it a problem.

The archive contains a \texttt{Makefile} that should compile all the sources by
running \texttt{make}. If the compilation was successful, you can run any of
the tests from the \texttt{tests} directory. Also, you can run the
\texttt{test.sh} script that runs all the tests and compares their outputs to
the expected ones.

The tests usually do some configuration and simulate receiving packets in
between individual configuration steps. As the subsystem produces debug messages to the
output, you can see what the individual components do. The expected behavior is
usually printed before running the steps and documented more extensively in
the source code.

The sources of the tests are more interesting than their outputs. They show
the expected userspace interface and the configuration of the subsystem. To
simulate packet reception, the test code acts both as the userspace and the
\a{NIC} driver, as shown on Figure \ref{fig:mat-comps}.

\begin{figure}
	\begin{tikzpicture}[
			fill=hl, very thin,
			mod/.style={minimum height=.8cm,minimum width=2.6cm,draw,fill=white,inner xsep=0.5em}
		]
		\filldraw (0,0) -- (0, 7.2) -- (7.5, 7.2)
			-- (7.5, 6) -- (0.5, 6)
			-- (0.5, 1.2) -- (7.5, 1.2)
			-- (7.5, 0) -- cycle;
		\node at (1.5, 6.6) {Test};
		\node[mod,minimum width=4cm] (tst-netdev) at (4.9, 0.6) {\netdev{} impl.};
		\node[mod] (tst-nl) at (4.05, 6.6) {setup};

		\filldraw (0.7, 1.4) rectangle (7.5, 3.1);
		\node at (4.05, 2.7) {Mock kernel};
		\node[mod] (krn-tc) at (2.2, 2) {TC};
		\node[mod] (krn-netdev) at (5.6, 2) {netdev iface};

		\filldraw (0.7, 3.6) rectangle (7.5, 5.4);
		\node at (4.05, 4.55) {Match-Action Tables};
		\node[mod] (mat-nl) at (4.05, 5.4) {netlink iface};
		\node[mod] (mat-block) at (2.2, 3.7) {block iface};
		\node[mod] (mat-netdev) at (5.6, 3.7) {driver iface};

		\draw[very thick] (5.6,1) -- (krn-netdev)
			(krn-netdev) -- (krn-tc)
			(krn-tc) -- (mat-block)
			(krn-netdev) -- (mat-netdev)
			(mat-nl) -- (tst-nl);
	\end{tikzpicture}
	\centering
	\caption[Components of the demonstrator]{The components of the demonstration implementation.}
	\label{fig:mat-comps}
\end{figure}

\section{Implementation Overview}

At the top level, we can split the implementation into three components. First,
there is the kernel mock. It defines structures like \skb{} or \netdev{}, but
also contains bits extracted from the internal API of \a{TC}. Also, we
extracted the implementation of linked lists and several general-purpose
macros. To avoid copying the \struct|idr| structure\footnote{Radix tree used to
allocate and map integer identifiers to objects.} implementation, we simply
used linearly increasing values and a very simple hashtable/array. Also, we
implement the kernel dynamic memory allocation routines \fnc|kzalloc| and
\fnc|kfree| using the \sw{libc} \fnc|calloc| and \fnc|free|, just to make the
code look like a kernel source code. In contrast, we decided not to implement
\fnc|printk| to make it obvious that the prints are for demonstration purposes
only.

Second, there is the implementation of the MAT subsystem. We will describe this
part in detail in the rest of the appendix. Together with the kernel mock, the
implementation is linked to a single archive \texttt{libmat.a} to represent
``the kernel''.

Finally, the archive contains a few test scenarios to demonstrate how the
subsystem is expected to be used. As already mentioned, the test scenarios
interact with the kernel from both userspace and hardware ends. Let us have
a look at them first.

\subsection{Tests}

There are five tests included:

\begin{description}
	\item[\texttt{exact}, \texttt{hash}, and \texttt{tcam}] \hfill \\
		These configure a table of the given type, set the default action chain
		and insert a rule to drop matching packets. The tests demonstrate how
		the tables are configured and test the software implementation of the
		subsystem.

	\item[\texttt{simple-nic}] \hfill \\
		This test demonstrates the driver interface and the hardware offload
		mechanism. The emulated \a{NIC} contains a blacklist 5-tuple filter in
		its pipeline. The test configures a table which matches on IP addresses
		and simulates receiving packets. It demonstrates how the ingress
		processing is moved to the \a{NIC}.

	\item[\texttt{parser}] \hfill \\
		This test extends the configurable parser with the \a{VXLAN} header.
		The parser tree is printed before and after to demonstrate the
		extension. Then, a simple table is created to show how the driver can
		read the parser state to offload it.

	\item[\texttt{multi-table}] \hfill \\
		A chain of tables demonstrating a more complex pipeline is tested.
		The tables represent a non-optimal firewall and unicast \a{MAC} filter.
\end{description}

\noindent Please note that having multiple demonstrations of the offload
mechanism would require to implement all the functionality to simulate the
hardware processing, duplicating the functionality of the subsystem in the
tests. As the principles stay the same for other types of tables, we believe
one demonstration is enough to show the principles of offloading.

\subsection{Match-Action Tables}

\begin{figure}
	\centering
	\begin{tikzpicture}[
			text depth=0.25ex,
			text height=1.5ex,
			inner xsep=.5em,
			module/.style={rectangle,draw,minimum width=2cm,minimum height=1cm,rounded corners=1mm,fill=white},
			subsys/.style={rectangle,draw,minimum width=1cm,minimum height=1cm,rounded corners=5mm},
		]
		\node[module] (table) {table};
		\node[module] (netlink) [right=of table] {netlink};
		\node[module] (executor) [below=of table] {executor};
		\node[module] (tc) [left=of executor] {tc};
		\node[module] (parser) [above=of table] {parser};
		\node[module] (tcam) [left=of parser] {tcam};
		\node[module] (netdev) [left=of table] {netdev};
		\node[module] (uapi) [above=of netlink] {uAPI headers};
		\node[subsys] (userspace) [right=of netlink] {userspace};
		\node[subsys] (drivers) [left=of netdev] {drivers};
		\node[subsys] (TC) [left=of tc] {TC};

		\draw (tc) -- (executor)
			(netdev) -- (executor)
			(netlink) -- (userspace)
			(netlink) -- (parser)
			(table) -- (tc)
			(table) -- (parser)
			(table) -- (executor)
			(table) -- (netdev)
			(table) -- (netlink)
			(netlink) -- (executor)
			(TC) -- (tc)
			(netdev) -- (drivers)
			(netdev) -- (parser)
			(parser) -- (tcam);

		\node[module,draw=none,fill=none,right=of executor] {\bf MAT};

		\scoped[on background layer]
			\node [fill=hl, fit=(tc) (parser) (uapi) (executor)] {};
	\end{tikzpicture}
	\caption[Modules of the demonstrator]{The modules of our implementation. Highlighted is the scope of the
	MAT subsystem. Relations show communicating modules.}
	\label{fig:modules}
\end{figure}

The implementation of the subsystem can be split into several modules, as seen
on Figure \ref{fig:modules}. Not all dependencies are displayed.

The implementation is rather simplified. It is not meant to be
production-ready, but rather serve as a demonstration of thoughts and
principles. It should not even be used as a base for the real implementation,
because some of the things we ignore are hard to add as an afterthought. Most
notably, we purposefully ignore:

\begin{description}
	\item[Multithreading and synchronization] \hfill \\
		As synchronization in the kernel is vastly different from that of
		user\-space, we decided to omit it completely. This alone should be
		a reason to write the real implementation from scratch.

	\item[Deconfiguration] \hfill \\
		To deliver the ideas, it is necessary to support creating the
		configuration. However, there is a lot of bookkeeping code to support
		removing or changing the configured entities. We decided to keep
		things simple for the demonstration.

		Also, when things are not removed, they are usually never deallocated.
		We believe that the destruction routines are easy to imagine.

	\item[Error paths] \hfill \\
		As the implementation focuses on the ideas, we consider handling error
		paths an unnecessary noise. Also, as there are no routines to handle
		removing things, there is no sensible way how to handle errors.

	\item[Netlink method calling] \hfill \\
		Where the netlink interface is just a plain call to a function with
		fixed arguments, we decided to just call the function directly. The
		netlink protocol is designed to be very flexible and binary
		compatible, but not particularly readable. Hiding the actual interface
		into netlink messages would introduce unnecessary clutter.
\end{description}

\noindent However, we did not simplify the important things. We strictly separate the
kernel and userspace memory areas. Data structures can be shared only from the
userspace to the kernel, not the other way around. Such structures are declared
in a shared header file, which is separate from the implementation. As already
said, the header file contains also the definitions of functions, which can be
easily transformed into simple calls over the netlink interface.

The \texttt{netdev} and \texttt{tc} are simple modules that serve as brokers to
communicate with other parts of the kernel. In the real-world implementation, these
would probably get thicker to better separate the implementation from the drivers,
supporting the stability of the kernel \a{ABI}.

\subsubsection{Parser}

The \texttt{parser} module keeps the representation of the packet parser. The
parser is defined completely as a runtime data structure to allow extensions.
The module defines two important structures: \struct|mat_parser| and
\struct|mat_header_field|. Instances of both are identified by globally unique indices.

In the following paragraphs, we use the term ``parser'' for both the instances of
the \struct|mat_parser| structure and the parser tree. In fact, the parser tree is
nothing more than the root parser instance for the Ethernet header.

An instance of \struct|mat_parser| is created for every protocol defined. The
parser is used to identify fields from a packet of the corresponding protocol
and to determine the parser for the next-layer header.

The \struct|mat_header_field| structure instances correspond to individual
header fields. The structure contains everything that is needed to extract and
interpret the field value from the byte stream. In case the field is expected
to be used to match on, it is assigned to a parser.

Every parser (except for the innermost protocols) has two fields by default
-- \Verb|nexthdr| and \Verb|hdrsize|. The former identifies the field used
to determine the next-layer parser, the latter determines the distance to the
next header in bytes.

To define which parser will be used to parse the next header, a mask-value
match is performed. The rules are stored in an instance of \struct|mat_tcam| structure,
provided by the \texttt{tcam} module. The structure simulates a \a{TCAM}.

The preconfigured parser identifies some basic fields and shows how to solve
several challenges. As an example of a challenge, the \emph{Ethertype} field is
identified by the \a{NIC}s as the Ethertype of the last \a{VLAN} tag, or the
\a{MAC} header directly if no \a{VLAN} tags are present. As such header field can
be attached to at most one parser, we define the Ethertype as a standalone
parser and make the Ethernet and \a{VLAN} parsers ``look ahead'', out of the
area that is skipped by parsing the corresponding header. For illustration, have a look at
Figure \ref{fig:ethertype-parser}.

\begin{figure}
	\begin{tikzpicture}[
		y = -0.8cm, x=7mm,
		helper/.style = {
			font=\scriptsize,
			draw,
			inner ysep=1mm,
			outer sep=0,
			text depth=0.25ex,
			text height=1.5ex,
			minimum width=1.4cm,
		},
		field/.style = {
			helper,
			fill = yellow!30,
		}]

		\node[helper] at (0,-1) {\a{MAC}};
		\node[helper] at (2,-1) {\a{MAC}};
		\node[helper] at (4,-1) {\texttt{0x8100}};
		\node[helper] at (6,-1) {VLAN};
		\node[helper] at (8,-1) {\texttt{0x0800}};
		\node[helper] at (10,-1) {IP};
		\node[helper] at (12,-1) {Payload};

		\node[field] at (0,0) {\a{MAC}};
		\node[field] at (2,0) {\a{MAC}};
		\node[helper,color=red] (type-1-u) at (4,0) {\texttt{0x8100}};

		\node[helper] (type-1-d) at (4,1) {\texttt{0x8100}};
		\node[field] at (6,1) {VLAN};
		\node[helper,color=red] (type-2-u) at (8,1) {\texttt{0x0800}};

		\node[field] (ethertype) at (8,2) {\texttt{0x0800}};

		\node[field] (ip) at (10,3) {IP};

		%\node at (-1, -1) [label=left:Packet data:] {};
		\node at (-1, 0)  [label=left:Ethernet:] {};
		\node at (-1, 1)  [label=left:VLAN:] {};
		\node at (-1, 2)  [label=left:Ethertype:] {};
		\node at (-1, 3)  [label=left:IP:] {};

		\draw[->] (type-1-u) to (type-1-d);
		\draw[->] (type-2-u) to (ethertype);
		\draw[->] (ethertype) to (ip);
	\end{tikzpicture}
	\centering
	\caption[An example of packet parsing]{Parsing a \a{VLAN}-tagged packet with the preconfigured parser.
	Parser states and their scope are in horizontal layers. Look-ahead fields
	shown in red, identified fields in yellow. Size not to scale.}
	\label{fig:ethertype-parser}
\end{figure}

It is important to note that the preconfigured parser tree should never be actually
used to extract fields. The values of well-known fields are already extracted
by the kernel flow dissector and thus present in the \skb. However, the
definition needs to be precise so as not to break the offload to the programmable parsers.

\subsubsection{Table}

The \texttt{table} module stores the configured tables. There are two main
structures defined: \struct|mat_table| and \struct|mat_flow_key|.

The purpose of the \struct|mat_table| structure is to keep all the information
about a table. Even though it is one of the largest structures in the codebase,
there is nothing unexpected.

The \struct|mat_flow_key| structure is a pure data carrier for flow keys. Both
its size and interpretation of content is determined by the table for which the
flow key is created. Therefore, the flow key is always associated with
a concrete table instance, even though it does not carry a pointer to the table.

When a table is created, a mapping of fields to the flow key bits is computed.
This part solves Bin packing problem to distribute variable-sized fields to
the least number of 64-bit \emph{parts} of the flow key. The number of parts is
then recorded in the table.

As the flow key interpretation depends on the table, it is the \texttt{table}
module that manipulates it, including filling the flow key with the data parsed
from the packet.

Every table holds its flows and action chains separately. Action chains are
stored in a hash table, in which the chains are identified by a unique 64-bit
number, \emph{Action ID}. In the future, the ID could be computed from the
content of the chain, allowing to merge identical chains into one.

Determining the Action ID varies with the table type. Tables of the \emph{Hash}
type use the flow hash directly, and therefore do not need to store any additional
data. Other types use different data structures following their purpose.

\subsubsection{Executor}

The software functionality of the subsystem is implemented in the
\texttt{executor} module. That includes both executing the action chains and
taking the packet through the pipeline of tables.

To allow fine-grained control by the device drivers, which might need to
resume the processing in software after offloading it partially to hardware,
the module exports its control
structure, \struct|mat_executor|. The structure contains the execution state
as well as a few variables that control the execution.

To start the executor from the beginning, the exported \a{TC} filter
initializes the structure and calls \fnc|mat_executor_run|. The executor
then performs all the necessary steps, until the final verdict is returned.

When the driver needs to start the execution from a different point (e.g.\ when
the hardware classified the packet, but is not able to execute the actions),
the driver initializes the \struct|mat_executor| structure, fills the known
information and uses the structure to resume the execution in software.

The \struct|mat_executor| structure holds a \a{TTL} number. With every step
executed, the \a{TTL} is decreased. When the \a{TTL} drops to zero, the
execution is terminated with a special result code.

When the \a{NIC} partially offloads the egress path, the driver has to run the
executor for any prefix of the table graph, such that the remaining suffix is
offloaded. The driver can either precisely choose the \a{TTL} to make the executor
stop at the right time, or just run the execution step-by-step.

The \a{TTL} field also protects from being caught in an infinite loop when the
subsystem is misconfigured.
