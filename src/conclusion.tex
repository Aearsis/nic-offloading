\chapwithtoc{Conclusion}

We succeeded in our goal to design a mechanism that would allow the user to
define a policy to process packets in the kernel, while allowing to offload the
policy into compatible \a{NIC}s. Rather than following the initial plan to pick
a few common features and create a specific mechanism for them, we
proposed a generic mechanism to offload the match-action processing.

We did so after carefully studying the pipeline of five high-end \a{NIC}s.
The subsystem we proposed is descriptive enough to represent the majority of
classification and packet-modification features these \a{NIC}s have while
being restricted enough to be offloaded easily.

Unfortunately, we have not succeeded in getting access to confidential
documents to support the research. Multiple vendors refused to give us
information that is not already public. Therefore, the thesis builds upon
public information only.

The proposed subsystem could replace the \a{TC} offload. It is better suited to be
offloaded to hardware, and thanks to a different approach it resolves some
serious problems of the \a{TC} offload.

It is not the first work of its kind. Most notably, John Fastabend proposed
the Flow API \cite{flow-api}, which might seem very similar. The
Flow API, however, serves to precisely describe a fixed pipeline of the
controller and to allow the userspace to program it directly, bypassing the
kernel. In contrast, our subsystem starts with doing the work in the kernel and
then offering the drivers to offload it.

Another project that is somewhat similar is the support for Flowtables in
Netfilter, implemented by Pablo Neira Ayuso \cite{flowtables}. The Flowtables
are limited to offloading forwarding, and take a ``reversed'' approach -- it is
the user who decides which particular rules should be offloaded. The drivers
create the tables and the user fills them.

To support our design, we created a proof-of-concept implementation, which
is attached to the electronic version of the thesis. It is also available
on Github\footnote{\href{https://github.com/Aearsis/mat}{\texttt{https://github.com/Aearsis/mat}}}.

The documentation of the implementation can be found in Appendix
\ref{app:doc}. The demonstration shows that the subsystem can be created, can
be used to process packets and that the drivers can offload its work.

While studying the source codes, we discovered two bugs in the Linux kernel.
Both of them has been reported to the respective maintainers. Unfortunately,
only one of them replied and confirmed the bug. As for the other one, we are
currently testing the patch and plan to submit it soon.

In the future, we plan to send the proposal of the subsystem to the Linux
NetDev community.
