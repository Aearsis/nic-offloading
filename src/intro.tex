\chapter{Introduction}

The networking development is a never-ending race towards wider bandwidth,
lower latencies and higher rates of processed packets. Where general-purpose
\a{CPU}s stay behind, dedicated hardware can increase the network performance.
Modern \acrfullpl{NIC}, in addition to connecting the host
computer to the network, also have advanced features that assist with
processing packets. The technique when the hardware is used to perform a part of
a job originally done in software is usually called \emph{hardware offloading}.

Recently, high-end controllers learned how to classify and modify packets. For
example, drop packets with certain properties or automatically extract an inner
packet encapsulated in a tunneling protocol. These features are useful (not
only) in environments where virtual machines running on shared hosts communicate via isolated
virtual networks that span over shared physical wires and devices. The less time the
host spends on preprocessing network traffic for the virtual machines, the more
it is left for doing the fruitful work.

On the opposite end of the network stack, there are applications that process
packets, while being part of the network function. For example the popular
Open vSwitch, that implements a full-featured software switch. There is
a considerable interest in offloading its work to the hardware to increase
speed and lower resource consumption.

To use the advanced processing capabilities of the controllers, the configured
policy must find its way from the userspace to the hardware. Currently, the
kernel is often bypassed with solutions like \acrfull{DPDK}, allowing an
application from userspace to configure the controller directly. Then more
features of the card can be used to process packets. However, the software is
highly specialized for this purpose and generally cannot be combined with
features the Linux network stack provides.

In the Linux kernel, there are several mechanisms that can be used for generic
in-kernel packet processing -- among others Netfilter, \a{TC}, \a{XDP}.
Unfortunately, none of them really fits to be offloaded using the packet
modification capabilities of the controllers.

New subsystems could be created in Linux kernel to support specific features of
individual controllers. However, the Linux kernel philosophy is to abstract the
hardware, so the subsystems created would have to work even without the
specific hardware installed. Therefore, solutions which would support devices
from a single vendor only are likely to be rejected by the community.

Because the hardware release cycle is long compared to the speed of evolution
of modern networking, the features that vendors put in their controllers are
getting more and more flexible. Between designing the controller and starting
to sell a finished adapter, new protocols are being invented. For the hardware,
being flexible is the only way to keep up with the software.

The flexibility of the controllers can be utilized to smudge differences
between individual controllers, allowing to create a subsystem which would be
offloadable by multiple controllers. The main goal of the thesis is to design
such a subsystem. The subsystem should provide the glue between userspace doing
packet processing and drivers of network controllers, creating a generic
platform for packet processing offloads. Ideally, any configuration should work
independently on the hardware installed, while allowing controllers to offload
as much work as possible to the hardware.

To achieve this goal, we selected five recent high-performance controllers and
decided to examine their capabilities in detail. As with the simpler offload
techniques, there is no literature which would contain the needed information
with the right amount of detail. There are advertisements and marketing
articles, which present rather vague terms and keywords, but usually do not
give any idea of how the controller works. Some controllers have manuals for
proprietary drivers from which the range of available features can be
deducted, but we cannot tell apart the work done by the driver and the
controller. Then there are the source codes of the Linux kernel and the
\a{DPDK} that can give us a very good understanding of features which are
utilized, but only after decoding the big codebase of relevant drivers.
Finally, there are public datasheets and manuals for some controllers, that
contain all the information needed, scattered in hundreds of pages with
additional information which is not relevant.

One of the most painful problems of Linux is that its documentation cannot keep
up with the immense speed of development. Usually, the initial idea is
documented, presented on conferences and so on, but subsequent changes do not
update the overall picture presented in the documentation. Therefore, the
current state of a feature is hard to understand if one does not follow the
development from the beginning.

The majority of information provided in this thesis is gathered from the source
code directly or assembled from little pieces found in the kernel documentation
and commit messages, providing the complete image of the current state of
hardware offloading. It is, to our knowledge, the only documentation of its
kind.

\section{Scope}

When talking about networking, we mostly limit ourselves to \a{IP} over Ethernet.
We are not fond of supporting protocol ossification, but \a{IP} and Ethernet are
unarguably the most widespread technology in the computer networking. As for the
network layer protocol, watching \a{IPv6} having a hard time in replacing
\a{IPv4}, it is hard to imagine completely different protocol taking over. For
the Ethernet, the situation is curious. A lot of different communication
standards over different mediums share the common Ethernet marketing label. It
is the presence of common paradigms that allows network controllers to support
multiple Ethernet standards at once, making a gradual transition to newer
standards possible.

For the target market of high-performance networking, other communication
technologies are available (e.g.\ InfiniBand), but their support and adoption by
operating systems is far from that of Ethernet and \a{IP}.

At the time of writing the thesis, the most recent released version of Linux
kernel was 4.16 \cite{linux-kernel}. All the information about the kernel is
based on this version. As the topic is still hot, we used also the David
Miller's net-next tree \cite{net-next} with the most recent updates for the networking subsystem to
learn more about the controllers. However, we do not refer to commits from
there, due to their experimental nature.

\section{Linux Network Stack}

As the thesis expects a brief orientation in the Linux networking stack,
a condensed and very simplified overview follows. We skip a lot of details and
intermediate packet processing and focus on parts which are important to
understanding the rest of the thesis. More comprehensive description can be
found in \cite{lkn-iat}, but the only literature that is always up-to-date is
the source code itself.

Let us explore the life of a datagram being transmitted using \a{UDP} over
\a{IPv4} between two applications that run on Linux hosts. First, we will look
at the \emph{egress} direction (sending the packet from the host to the
network), then the \emph{ingress} direction (receiving a packet from the
network to the host). Suppose that the datagram is small enough to be delivered
in one \a{IP} fragment and let us ignore all the errors that might happen.

When an application wants to communicate via the network in Unix-like operating
systems, it opens a \emph{socket}. The socket is an entity in the kernel memory, that
can be controlled by the application using a handle (a file descriptor). The
socket can be created using a system call of
the same name. In our case, the socket is created with \macro|AF_INET| address
family (\a{IPv4}), \macro|SOCK_DGRAM| socket type to communicate using
datagrams and \macro|IPPROTO_UDP| protocol to encapsulate data in the \a{UDP}.

Once created, the corresponding file descriptor is used as an argument to
subsequent system calls, controlling the entity in the kernel. For the sender,
no further setup is necessary, the socket is ready to send datagrams right away.
The sending application prepares the data in a memory buffer, and requests them
to be sent by the system calls \fnc|sendto| or \fnc|sendmsg|. (The system calls
\fnc|write| or \fnc|send| can be used as well, but the socket must be
configured with the intended recipient first.)

No matter which system call was used, it is handled by calling a \fnc|sendmsg|
protocol callback. In our case, the \fnc|udp_sendmsg| function. To move
packet-related data around the kernel, an \skb{} structure is created. This
structure represents a data buffer in the networking subsystem and is used
almost everywhere. Its lifetime is dynamic, thus a reference counting is
employed.

Once the \skb{} is created, it is filled with already known metadata and packet
headers are constructed. An unexpected fact is that the \a{IP} header is
filled in earlier than the \a{UDP} header. As \a{UDP} is closely tied to
\a{IP}, the layered network model is not followed strictly in Linux.

The next important decision to be made is routing the packet. Routing (among
other things) selects the port, which will be used to push the packet out from
the host. Ports are represented by \netdev{} structures in Linux. It is common
for high-performance \a{NIC}s to present multiple ports to the system, then
multiple \netdev s may correspond to a single physical controller.

As there might be multiple applications trying to send data out from the
selected device, the packets are not given directly to the driver. Instead,
they are temporarily stored in queues in the \acrfull{TC} subsystem. A detailed
look at the subsystem is provided in Section \ref{sec:tc}.

The \a{NIC} usually utilizes circular queues to communicate with the host.
Packets are dequeued by the controller at its convenience. When there is an
empty slot for a new packet (and such a packet is available), it is dequeued
from \a{TC} and handed out to the driver.

The \netdev{} structure, among a lot of things, carries a pointer to
a \struct|net_device_ops| structure. This structure holds callbacks, which
implement the network device interface for the rest of the system. One of the
most important callbacks is \fnc|ndo_start_xmit|, that is called to transmit a packet.

The driver usually needs to fill some descriptor structure for the packet,
configuring the processing which will happen in the hardware. An obvious part
of the descriptor is the memory location of the buffer where the packet is
stored. Once the descriptor is ready, its virtual ownership is transferred to
the controller.

But the processing for the sending host is not over yet. All the memory used by
the packet must not be deallocated, because it is potentially accessed by the
controller in the background. Therefore, the driver still holds one reference
for the \skb{} and drops it only after the respective slot in the hardware
queue is marked as empty.

Before the packet reaches the kernel, the receiving application must be
prepared to receive it. If it was not, the kernel would drop the packet as
not wanted. The application does so by opening a socket and \emph{binding} it.
When the socket is bound to an address and port, the kernel notes that packets
sent to such a destination should be delivered to this particular socket.

The ingress path of a packet is a bit simpler. First, the host needs to prepare
memory for the received packets in advance. It allocates free pages and
enqueues the receive descriptors, similarly to what it does when sending
packets.

Once the controller decodes a network frame from the medium, it copies it to
the prepared memory. As there is usually some processing in the controller
itself, some metadata about the packet are already known. Some of the metadata
are stored inside the descriptor, to potentially speed up the processing on the
host.

Next, the controller marks the slot in the receive queue as ready. The driver
picks it up and creates a \skb{} structure for the received packet. It can use
some metadata from the descriptor to pre-fill fields of the \skb{}. Then the
driver calls \fnc|netif_receive_skb| to give the received packet to the network
stack.

At first, the packet is parsed to identify some header fields, up to the
transport layers. This is needed to support some early optimizations, which are
further discussed in Section \ref{sec:rps}. In contrast to egress path, there
is no buffering of packets in \a{TC}, all the packets are delivered
immediately. In case of \a{IPv4}, delivery is realized by calling the
\fnc|ip_rcv| function.

Similarly to egress, routing tables are consulted to decide, whether the
traffic is local. If so, the \a{UDP} handler is called, which enqueues the
packet to a socket queue. There the packet waits until it is picked up by the
application calling the system calls \fnc|recvmsg|, \fnc|recvfrom|, \fnc|recv|,
or \fnc|read|.
