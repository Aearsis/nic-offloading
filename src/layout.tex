%%% The main file. It contains definitions of basic parameters and includes all other parts.

%% Settings for single-side (simplex) printing
% Margins: left 40mm, right 25mm, top and bottom 25mm
% (but beware, LaTeX adds 1in implicitly)
%\documentclass[12pt,a4paper]{report}
%\setlength\textwidth{145mm}
%\setlength\textheight{247mm}
%\setlength\oddsidemargin{15mm}
%\setlength\evensidemargin{15mm}
%\setlength\topmargin{0mm}
%\setlength\headsep{0mm}
%\setlength\headheight{0mm}
%\let\openright=\clearpage

%% Settings for two-sided (duplex) printing
\documentclass[12pt,a4paper,twoside,openright]{report}
\setlength\textwidth{145mm}
\setlength\textheight{247mm}
\setlength\oddsidemargin{14.2mm}
\setlength\evensidemargin{0mm}
\setlength\topmargin{0mm}
\setlength\headsep{0mm}
\setlength\headheight{0mm}
\let\openright=\cleardoublepage

%% Generate PDF/A-2u
\usepackage[a-2u]{pdfx}

%% Character encoding: usually latin2, cp1250 or utf8:
\usepackage[utf8]{inputenc}

%% Prefer Latin Modern fonts
\usepackage{lmodern}

\PassOptionsToPackage{hyphens}{url}

%% Further useful packages (included in most LaTeX distributions)
\usepackage[english]{babel}
\usepackage[date=iso,urldate=iso]{biblatex}
\usepackage[nottoc]{tocbibind} % makes sure that bibliography and the lists
			    % of figures/tables are included in the table
			    % of contents
\usepackage[usenames]{xcolor}  % typesetting in color
\usepackage[T1]{fontenc}
\usepackage[chapter,cachedir=listings,finalizecache,outputdir=build]{minted}
\usepackage[toc,acronym,nogroupskip,nomain]{glossaries}
\usepackage{glossary-mcols,enumitem,tabularx,tikz,graphicx,fancyvrb}

\widowpenalty 9000
\clubpenalty 9000

\setglossarystyle{mcolindex}

\usetikzlibrary{backgrounds,fit,decorations.pathmorphing,decorations.pathreplacing,arrows.meta,positioning}
\tikzset{>=To}

\hyphenation{sche-duling net-ro-nome}

\definecolor{lightgray}{gray}{0.95}
\colorlet{darkred}{red!35!black}
\colorlet{darkblue}{blue!35!black}
\colorlet{darkgreen}{green!35!black}

\colorlet{hl}{yellow!30}
\colorlet{hl-red}{red!20}
\colorlet{hl-blue}{blue!10}

\setlist[itemize,enumerate]{noitemsep}

\usemintedstyle{borland}
%\setminted[shell]{autogobble,linenos,breaklines,frame=single,framesep=10pt}
%\newmintinline{shell}{}
%\newenvironment{shell}{\VerbatimEnvironment\begin{minted}{shell}}{\end{minted}}

%%% Basic information on the thesis

% Thesis title in English (exactly as in the formal assignment)
\def\ThesisTitle{Network Interface Controller Offloading in Linux}

% Author of the thesis
\def\ThesisAuthor{Bc. Ondřej Hlavatý}

% Year when the thesis is submitted
\def\YearSubmitted{2018}

% Name of the department or institute, where the work was officially assigned
% (according to the Organizational Structure of MFF UK in English,
% or a full name of a department outside MFF)
\def\Department{Department of Distributed and Dependable Systems}

% Is it a department (katedra), or an institute (ústav)?
\def\DeptType{Department}

% Thesis supervisor: name, surname and titles
\def\Supervisor{prof. Ing. Petr Tůma, Dr.}

% Supervisor's department (again according to Organizational structure of MFF)
\def\SupervisorsDepartment{Department of Distributed and Dependable Systems}

% Study programme and specialization
\def\StudyProgramme{Computer Science}
\def\StudyBranch{Software Systems}

% An optional dedication: you can thank whomever you wish (your supervisor,
% consultant, a person who lent the software, etc.)
\def\Dedication{%
	I would like to thank my supervisor prof. Ing. Petr Tůma, Dr. for guidance.
	Also, my consultant Jiří Benc and the Red Hat company, for lending me
	hardware and offering me the topic. Last but not least, my family and
	friends, for supporting me throughout my studies. Thank you.
}

% Abstract (recommended length around 80-200 words; this is not a copy of your thesis assignment!)
\def\Abstract{%
%	Network interface controllers assist the host with
%	packet processing. For modern controlle lers, these features include
%	increasingly flexible classification and modification of packets. At the present time, the Linux
%	kernel utilizes these features only by offloading the work of the Traffic
%	Control subsystem, which was designed for a completely different purpose,
%	making the offload hard to understand for users and unnecessarily complex for drivers.
%
%	This thesis proposes a new subsystem that would allow utilization of features of
%	modern controllers, while avoiding the problems that come with offloading
%	Traffic Control. Five high-end controllers were examined and their
%	capabilities were determined in detail. The information
%	about these controllers, extracted from datasheets and source codes, is
%	also included in the thesis. Finally, as there is no extensive up-to-date
%	overview of the current hardware offloading techniques, we start by
%	introducing the reader to them in general.
	Modern network interface controllers allow the host to offload packet
	processing to hardware in order to improve performance. At the present
	time, the advanced features are utilized in the Linux kernel by offloading
	the Traffic Control subsystem. Since this subsystem has been designed for
	a completely different purpose, its usage for hardware offloading is
	impractical and unreliable. Furthermore, in its current state the subsystem
	is not capable of utilizing all hardware features, which are often poorly
	documented.

	The presented work adopts a different approach to the problem. Five
	high-end controllers and their packet-processing pipelines were examined in
	detail. Accounting for their projected future development, common traits
	and features were identified. The researched information was used to draft
	a proposal for a new Linux subsystem, more compatible with hardware
	offloading than the current solution. The proposed subsystem defines
	a sufficiently descriptive interface to utilize the majority of
	hardware-offloaded features while avoiding common problems caused by
	excessively generalized approach of Traffic Control.
}

% 3 to 5 keywords (recommended), each enclosed in curly braces
\def\Keywords{%
	{hardware offloading} {network} {linux} {traffic control}
}

%% The hyperref package for clickable links in PDF and also for storing
%% metadata to PDF (including the table of contents).
%% Most settings are pre-set by the pdfx package.
\hypersetup{unicode}
\hypersetup{hidelinks}
\hypersetup{breaklinks=true}

% Definitions of macros (see description inside)
%%% This file contains definitions of various useful macros and environments %%%
%%% Please add more macros here instead of cluttering other files with them. %%%

%%% Minor tweaks of style

% These macros employ a little dirty trick to convince LaTeX to typeset
% chapter headings sanely, without lots of empty space above them.
% Feel free to ignore.
\makeatletter
\def\@makechapterhead#1{
  {\parindent \z@ \raggedright \normalfont
   \Huge\bfseries \thechapter. #1
   \par\nobreak
   \vskip 20\p@
}}
\def\@makeschapterhead#1{
  {\parindent \z@ \raggedright \normalfont
   \Huge\bfseries #1
   \par\nobreak
   \vskip 20\p@
}}
\makeatother

% This macro defines a chapter, which is not numbered, but is included
% in the table of contents.
\def\chapwithtoc#1{
\chapter*{#1}
\addcontentsline{toc}{chapter}{#1}
}

% Draw black "slugs" whenever a line overflows, so that we can spot it easily.
%\overfullrule=1mm

%%% Macros for definitions, theorems, claims, examples, ... (requires amsthm package)

%\theoremstyle{plain}
%\newtheorem{thm}{Theorem}
%\newtheorem{lemma}[thm]{Lemma}
%\newtheorem{claim}[thm]{Claim}
%
%\theoremstyle{plain}
%\newtheorem{defn}{Definition}
%
%\theoremstyle{remark}
%\newtheorem*{cor}{Corollary}
%\newtheorem*{rem}{Remark}
%\newtheorem*{example}{Example}

%%% An environment for proofs

%%% FIXME %%% \newenvironment{proof}{
%%% FIXME %%%   \par\medskip\noindent
%%% FIXME %%%   \textit{Proof}.
%%% FIXME %%% }{
%%% FIXME %%% \newline
%%% FIXME %%% \rightline{$\square$}  % or \SquareCastShadowBottomRight from bbding package
%%% FIXME %%% }

%%% An environment for typesetting of program code and input/output
%%% of programs. (Requires the fancyvrb package -- fancy verbatim.)

\DefineVerbatimEnvironment{code}{Verbatim}{fontsize=\small, frame=single}
\DefineVerbatimEnvironment{shell}{Verbatim}{fontsize=\small, numbers=left,numbersep=3mm,frame=single, framesep=2ex}

%%% The field of all real and natural numbers
\newcommand{\R}{\mathbb{R}}
\newcommand{\N}{\mathbb{N}}

%%% Various math goodies
\newcommand{\cdotop}{\mathop{\cdot}}

%%% Code snippets
\CustomVerbatimCommand{\macro}{Verb}{formatcom=\color{darkgreen},fontfamily=lmtt}
\CustomVerbatimCommand{\fnc}{Verb}{formatcom=\color{darkblue},fontfamily=lmtt}
\CustomVerbatimCommand{\struct}{Verb}{formatcom=\color{darkred},fontfamily=lmtt}
\CustomVerbatimCommand{\field}{Verb}{formatcom=\color{darkred},fontfamily=lmtt}

\newcommand{\macrofmt}[1]{\textcolor{darkgreen}{\texttt{#1}}}
\newcommand{\fncfmt}[1]{\textcolor{darkblue}{\texttt{#1}}}
\newcommand{\structfmt}[1]{\textcolor{darkred}{\texttt{#1}}}
\let\fieldfmt\structfmt

\newcommand{\cmd}[1]{\texttt{#1}}
\newcommand{\sw}[1]{\texttt{#1}}

\newcommand{\netdev}{\structfmt{net\_device}}
\newcommand{\skb}{\structfmt{sk\_buff}}

\newcommand{\kernelcommit}[2]{\href{https://git.kernel.org/pub/scm/linux/kernel/git/#1.git/commit/?id=#2}{#2}}
\newcommand{\kerneltag}[2]{\href{https://git.kernel.org/pub/scm/linux/kernel/git/#1.git/commit/?h=#2}{#2}}

\newcommand{\linuxcommit}[1]{\kernelcommit{torvalds/linux}{#1}}
\newcommand{\netnextcommit}[1]{\kernelcommit{davem/net-next}{#1}}

\newcommand{\linuxtag}[1]{\kerneltag{torvalds/linux}{#1}}

\newcommand{\linuxfile}[1]{\href{https://git.kernel.org/pub/scm/linux/kernel/git/torvalds/linux.git/tree/#1?h=v4.16}{\texttt{#1}}}

% \linuxfnc{file}{line}{fnc}
\newcommand{\linuxfnc}[3]{\href{https://git.kernel.org/pub/scm/linux/kernel/git/torvalds/linux.git/tree/#1?h=v4.16\#n#2}{\fncfmt{#3}\footnote{Linux kernel~\cite{linux-kernel}, file \texttt{#1}, line #2}}}

% \linuxstruct{file}{line}{struct}
\newcommand{\linuxstruct}[3]{\href{https://git.kernel.org/pub/scm/linux/kernel/git/torvalds/linux.git/tree/#1?h=v4.16\#n#2}{\structfmt{#3}\footnote{Linux kernel~\cite{linux-kernel}, file \texttt{#1}, line #2}}}

% \ethtoolfile{file}{line}{label}
\newcommand{\ethtoolfile}[3]{\href{https://git.kernel.org/pub/scm/network/ethtool/ethtool.git/tree/#1?h=v4.16\#n#2}{\texttt{#3}}}



\makeglossaries
\loadglsentries[main]{abbreviations}

% Use the acronyms as acronyms. We can't define everything...
\let\a\acrshort

\addbibresource{bib/ieee.bib}
\addbibresource{bib/linux.bib}
\addbibresource{bib/manuals.bib}
\addbibresource{bib/rfc.bib}

\thesispreamble

% Title page and various mandatory informational pages
\begin{document}
\include{title}

%%% A page with automatically generated table of contents of the master thesis
\tableofcontents

\chapter{Introduction}

The networking technology development is a never-ending race towards wider bandwidth,
lower latencies and higher rates of processed packets. Where general-purpose
\a{CPU}s stay behind, dedicated hardware can increase network performance.
Modern \acrfullpl{NIC}, in addition to connecting the host
computer to the network, also have advanced features that assist with
processing packets. When hardware is used to perform a part of
a job originally done in software the technique is usually called \emph{hardware offloading}.

Recently, high-end controllers learned how to classify and modify packets, for
example, to drop packets with certain properties or to automatically extract an inner
packet encapsulated in a tunneling protocol. These features are useful (not
only) in environments where virtual machines running on shared hosts communicate via isolated
virtual networks that span over shared physical wires and devices. The less time the
host spends on preprocessing the network traffic for the virtual machines, the more
is left for doing the fruitful work.

On the opposite end of the network stack are applications that process
packets while still being a part of the network function. For example
Open vSwitch implements a full-featured software switch. There is
a considerable interest in offloading its work to hardware to increase
speed and lower resource consumption.

To use the advanced processing capabilities of the controllers, the configured
policy must find its way from the userspace to the hardware. Currently, the
kernel is often bypassed with solutions like \acrfull{DPDK}, allowing an
application to configure the controller directly from userspace, using more
features of the card to process packets. However, the software is
highly specialized for this purpose and generally cannot be combined with
features the Linux network stack provides.

In the Linux kernel, there are several mechanisms that can be used for generic
in-kernel packet processing -- among others Netfilter, \a{TC}, \a{XDP}.
Unfortunately, none of them really fits to be offloaded using the packet
modification capabilities of the controllers.

New subsystems could be created in the Linux kernel to support specific features of
the individual controllers. However, the Linux kernel philosophy is to abstract away from the
hardware, so the subsystems created would have to work even without the
specific hardware installed. Therefore, solutions which would support devices
from a single vendor only are likely to be rejected by the community.

Because the hardware release cycle is long compared to the speed of evolution
of modern networking, the features that vendors put in their controllers are
getting more and more flexible. Between designing the controller and starting
to sell a finished adapter, new protocols are being invented. For the hardware,
being flexible is the only way to keep up with the software.

The flexibility of the controllers can be utilized to reduce differences
between the individual controllers, allowing to create a subsystem which would be
offloadable by multiple controllers. The main goal of the thesis is to design
such a subsystem. The subsystem should provide the glue between userspace doing
packet processing and the drivers of the network controllers, creating a generic
platform for packet processing offloads. Ideally, any configuration should work
independently of the hardware installed, while allowing software to offload
as much work as possible to the hardware.

To achieve this goal, we selected five recent high-performance controllers and
decided to examine their capabilities in detail. As with the simpler offload
techniques, there is no literature which would contain the needed information
with the right amount of detail. There are advertisements and marketing
articles, which present rather vague terms and keywords, but usually do not
give any idea of how the controller works. Some controllers have manuals for
proprietary drivers from which the range of available features can be
deduced, but we cannot tell apart the work done by the driver and the
controller. Then there is the source code of the Linux kernel and the
\a{DPDK} that can give us a very good understanding of the features which are
utilized, but only after decoding the big codebase of the relevant drivers.
Finally, there are public datasheets and manuals for some controllers that
contain all the information needed, scattered in hundreds of pages with
additional information which is not relevant.

One of the most painful problems of Linux is that its documentation cannot keep
up with the immense speed of development. Usually, the initial idea is
documented, presented on conferences and so on, but subsequent changes do not
update the overall picture presented in the documentation. Therefore, the
current state of a feature is hard to understand if one does not follow its
development from the beginning.

The majority of information provided in this thesis is gathered from the source
code directly or assembled from little pieces found in the kernel documentation
and commit messages, providing the complete image of the current state of
hardware offloading. It is, to our knowledge, the only document of its
kind.

In summary, this thesis contributes:
\begin{itemize}
	\item The review of the hardware offloading techniques in Chapter \ref{chap:offloading}.
	\item The review of the features of the selected controllers in Chapter \ref{chap:nics}.
	\item The overview of the Linux kernel with respect to the network hardware offloading in Chapter \ref{chap:linux}.
	\item The proposal of the new subsystem in Chapter \ref{chap:rfc}.
\end{itemize}

\section{Scope}

When talking about networking, we mostly limit ourselves to \a{IP} over Ethernet.
We are not fond of supporting protocol ossification, but \a{IP} and Ethernet are
arguably the most widespread technologies in the computer networking. As for the
network layer protocol, watching \a{IPv6} having hard time replacing
\a{IPv4}, it is hard to imagine a completely different protocol taking over. For
the Ethernet, the situation is curious. A lot of different communication
standards over different media share the common Ethernet marketing label. It
is the presence of the common paradigms that allows the network controllers to support
multiple Ethernet standards, making a gradual transition to newer
standards possible.

For high-performance networking, other communication
technologies are available (e.g.\ InfiniBand), but their support and adoption by
the operating systems is far from that of Ethernet and \a{IP}.

At the time of writing the thesis, the most recent released version of Linux
kernel was 4.16~\cite{linux-kernel}. All the information about the kernel is
based on this version. As the topic is still hot, we also used the David
Miller's net-next tree~\cite{net-next} with the most recent updates for the networking subsystem to
learn more about the controllers. However, we do not refer to commits from
there, due to their experimental nature.

\section{Linux Network Stack}

As the thesis is requires the reader to orient briefly in the Linux networking stack,
a condensed and very simplified overview follows. We skip a lot of detail and
intermediate packet processing and focus on parts which are important to
understanding the rest of the thesis. A more comprehensive description can be
found in~\cite{lkn-iat}, but the only literature that is always up-to-date is
the source code itself.

Let us explore the life of a datagram being transmitted using \a{UDP} over
\a{IPv4} between two applications that run on Linux hosts. First, we will look
at the \emph{egress} direction (sending the packet from the host to the
network), then the \emph{ingress} direction (receiving a packet from the
network). Suppose that the datagram is small enough to be delivered
in one \a{IP} fragment and let us ignore all the errors that might happen.

When an application wants to communicate via the network in Unix-like operating
systems, it opens a \emph{socket}. The socket is an entity in kernel memory that
can be controlled by the application using a handle (a file descriptor). The
socket can be created using a system call of
the same name. In our case, the socket is created with the \macro|AF_INET| address
family (\a{IPv4}), the \macro|SOCK_DGRAM| socket type to communicate using
datagrams and the \macro|IPPROTO_UDP| protocol to encapsulate data in the \a{UDP}.

Once created, the corresponding file descriptor is used as an argument to
subsequent system calls, controlling the entity in the kernel. For the sender,
no further setup is necessary, the socket is ready to send datagrams right away.
The sending application prepares the data in a memory buffer, and requests them
to be sent by the \fnc|sendto| or \fnc|sendmsg| system calls. (The \fnc|write|
or \fnc|send| system calls can be used as well, but the socket must be
configured with the intended recipient first.)

No matter which system call was used, it is handled by calling a \fnc|sendmsg|
protocol callback -- in our case, the \fnc|udp_sendmsg| function. To move
the packet-related data around the kernel, an \skb{} structure is created. This
structure represents a data buffer in the networking subsystem and is used
almost everywhere. Its lifetime is dynamic, thus reference counting is
employed.

Once the \skb{} is created, it is filled with already known metadata and packet
headers are constructed. An unexpected fact is that the \a{IP} header is
filled earlier than the \a{UDP} header. As \a{UDP} is closely tied to
\a{IP}, the layered network model is not followed strictly in Linux.

The next important decision to be made is routing the packet. Routing (among
other things) selects the port that will be used to push the packet out from
the host. Ports are represented by the \netdev{} structures in Linux. It is common
for high-performance \a{NIC}s to present multiple ports to the system,
multiple \netdev s may then correspond to a single physical controller.

As there might be multiple applications trying to send data from the
selected device, the packets are not given directly to the driver. Instead,
they are temporarily stored in queues in the Traffic Control (\a{TC}) subsystem. A detailed
look at the subsystem is provided in Section \ref{sec:tc}.

The \a{NIC} usually utilizes circular queues to communicate with the host.
Packets are dequeued by the controller at its convenience. When there is an
empty slot for a new packet (and such a packet is available), it is dequeued
from \a{TC} and handed out to the driver.

The \netdev{} structure, among many things, carries a pointer to
a static instance of \struct|net_device_ops| structure. This structure holds callbacks that
implement the network device interface for the rest of the system. One of the
most important callbacks is \fnc|ndo_start_xmit|, which is called to transmit a packet.

The driver usually needs to fill some descriptor structure for the packet,
configuring the processing that will happen in the hardware. An obvious part
of the descriptor is the memory location of the buffer where the packet is
stored. Once the descriptor is ready, its virtual ownership is transferred to
the controller.

But the processing for the sending host is not over yet. The memory used by
the packet must not be deallocated, because it is potentially accessed by the
controller in the background. Therefore, the driver still holds one reference
for the \skb{} and drops it only after the respective slot in the hardware
queue is marked as empty.

Before the packet reaches the kernel on the other host, the receiving application must be
prepared to receive it. If it was not, the kernel would drop the packet as
not wanted. The application does so by opening a socket and \emph{binding} it.
When the socket is bound to an address and port, the kernel notes that packets
sent to this destination should be delivered to this particular socket.

The ingress path of a packet is a bit simpler. First, the host needs to prepare
memory for the received packets in advance. It allocates free pages and
enqueues the receive descriptors, similarly to what it does when sending
packets.

Once the controller decodes a network frame from the medium, it copies it to
the prepared memory. As there is usually some processing in the controller
itself, some metadata about the packet is already known. Some of the metadata
is stored inside the descriptor, to potentially speed up the processing on the
host.

Next, the controller marks the slot in the receive queue as ready. The driver
picks it up and creates an \skb{} structure for the received packet. It can use
some metadata from the descriptor to pre-fill fields of the \skb{}. Then the
driver calls \fnc|netif_receive_skb| to give the received packet to the network
stack.

First, the packet is parsed to identify some header fields, up to the
transport layers. This is needed to support some early optimizations, which are
further discussed in Section \ref{sec:rps}. In contrast to the egress path, there
is no buffering of packets in \a{TC}, all packets are delivered
immediately. In case of \a{IPv4}, the delivery is realized by calling the
\fnc|ip_rcv| function.

Similarly to egress, the routing tables are consulted to decide whether the
traffic is local. If so, the \a{UDP} handler is called, which enqueues the
packet in the socket queue. There, the packet waits until it is picked up by the
application calling the \fnc|recvmsg|, \fnc|recvfrom|, \fnc|recv|,
or \fnc|read| system calls.

\chapter{Hardware Offloading in General}
\label{chap:offload}

In this chapter we would like to introduce the current techniques used to reduce
the amount of work that the network stack has to do in software (by the host CPU)
by doing it in the \a{NIC} instead. The simplest techniques are implemented by
virtually every \a{NIC} currently on the market, the interfaces are stable and
well-supported. On the other end of the spectrum there are techniques which
are specific to high-performance cards and are under rapid development at the
time of writing.

All of the information given in this chapter is public. However, there is no
comprehensive overview of available techniques. Individual offloads are given
business names and the implementation is buried deep inside datasheets.
Available documentation for drivers and operating systems focuses on how to
control the mechanisms and when it is good to use them, but usually do not
cover the principles behind. Therefore, this chapter aims to be an introduction
to modern networking world.

It is worth noting that every vendor of a network controller uses their
specific terminology. The implementation details might differ as well, but the
high-level principles described in this chapter are implemented by multiple
\a{NIC} vendors.

An important aspect of an \emph{offload} is that the functionality provided by
the hardware is not required for frames to be processed. The functionality is
strictly opt-in and only if the driver of the device is capable of doing so, it
can enable it and benefit from skipping some steps in the network frame
processing on the host.

At first, we would like to mention two techniques that can be
classified as offloads, though this requires a rather loose interpretation of the
term. First is the scatter-gather capability of the controller, which frees the
host from assembling multiple buffers into one. Second is the interrupt
moderation, which reduces the overhead of switching contexts and allows to
batch-process multiple frames. Both of these techniques save some CPU cycles to
process a frame, but they do not depend on the frame content.

\section{Checksum Offload}

A common approach to ensure consistency of data transmitted over a network
involves checksums. Those are values that are usually inexpensive to compute,
yet provide a good level of reliability in indicating data corruption.

At the link layer the \a{NIC} usually computes the checksum itself, because it
is well aware of the protocol being used. In the case of Ethernet the
checksum is called \emph{Frame Check Sequence} and is transmitted at the end of
the frame, after the payload \cite{ethernet}. Therefore, the \a{NIC} is able to
compute the intermediate value of the checksum while sending and finish the frame
by the final checksum. Similarly, on the ingress path, after a frame is received
completely (with the checksum included), the value should be zero -- otherwise
the frame is dropped.

Moving up to the network layer, \a{IPv6} does not embody a checksum at
all \cite{RFC2460}. The \a{IPv4} header contains a one's complement checksum of the
header itself \cite{RFC0791}. The header is built completely by the software
and its maximum size is 60 bytes,
therefore the checksum is not expensive to compute. Still, many controllers are
capable of computing the \a{IPv4} header checksum before transmission.

Matters get complicated at the transport layer. A handful of protocols uses 16-bit
one's complement checksum of the whole packet -- not only \a{TCP} and \a{UDP},
the same approach is used by \a{DCCP} or \a{GRE}, for example. The offset at
which the checksum field is placed in the header varies with the protocol.

Both receive and send offloading capabilities for the transport layer checksums are
commonly offered by \a{NIC}s. On the receiving side, a \a{NIC} can either
validate the packet directly or just calculate the checksum including the
checksum field. The second way is preferred because of its flexibility, as we
will see in detail in Section \ref{tag:linux-rxcsum}.

Due to the algebraic properties of the checksum, it is not necessary to know
where the checksum field is to verify the received packet, making the receive
checksum offload both simple and protocol-agnostic. However, the
transport-layer packet must be received completely, in other words, the \a{IP}
packet must not be fragmented.

When sending a packet, older \a{NIC}s might be able to only check for
the presence of a known transport protocol and compute the checksum for them.
Recent controllers have the ability to compute checksum of any suffix of the
packet and write it to an arbitrary offset, allowing the system software to
offload compatible protocols in a generic way.

Both \a{TCP} and \a{UDP} compute the checksum not only from the packet header
and payload, but also from a \emph{Pseudo Header} which is composed of the source and
destination addresses.
These are not duplicated in the transport layer \a{PDU}, and because of that
the device might require the system software to precompute the checksum of the pseudo
header, exploiting the associativity and commutativity of the checksum
algorithm. An example of this requirement can be found in the Intel XL710
Controller (\cite{XL710}, Section 8.4.4.3.2). When a packet is received, its
computed checksum (including the checksum field) should be equal to
a complement of the pseudo header checksum.

For different checksumming algorithms, alternatives are also often provided.
For example, the Intel 82599 Controller can offload \a{SCTP} \a{CRC}
(\cite{82599}, Section 7.2.5.3).

\section{\a{TCP} Segmentation Offload}

As \a{TCP} is offering an interface of a stream pipe, it includes is a rather
complicated mechanism which creates the illusion of the pipe on top of a network that is only able to
transmit individual packets of bounded length.
When chunks of data are inserted into the pipe, they are split into \emph{segments}.
Segments are labeled with sequential numbers and sent individually. The
sequential numbering is used to ensure that no segments are lost and data is
delivered (passed to the application layer) in the correct order.
However, the size of the chunks is not defined. Taken to the extreme, \a{TCP}
could deliver individual bytes.

Obviously, a lot of overhead can be mitigated by handling data in the biggest chunks
possible, traversing the entire stack the least number of times. Speaking about
software, the network stack usually performs the segmentation at the latest
possible moment, performing as much processing as possible in batches. When receiving, the stack
might try to coalesce segments of a single \a{TCP} stream before delivering them.

This effort can be further supported by the hardware. A \a{NIC} can offer an
interface which allows to enqueue \a{TCP} packets that are larger than the link
\a{MTU}. Before transmitting them on the wire, the controller splits the packet
into multiple segments by itself. This way, the network stack is no longer
bound to the link \a{MTU} and can work with coarser chunks. This feature is
often referred to as \emph{Large Send Offloading}.

As this offload needs to have insight up to the \a{TCP} header, only a limited
set of header combinations is typically supported. Apart from plain \a{TCP} over
\a{IP}, more advanced \a{NIC}s can support \a{TCP} encapsulated in multiple
tunnel types.

The receiving counterpart of \a{LSO} is usually called \acrfull{LRO} and
does the expected opposite. Prior to enqueueing the packet into the receive
queue of the operating system, it tries to coalesce multiple received segments
belonging to one \a{TCP} stream into one super-frame. It is important to make
this feature optional, because it would violate the responsibilities of a network
when the packet is bridged or routed in the software.

Careful reader might notice that both directions of \a{TCP} segmentation offload depend on
checksum offload, as the newly-forged packets containing segments need to have their checksum
computed by the device on all layers. On the ingress path, the to-be-forgotten
packets carrying segments need to be verified prior to coalescing, and the
coalesced packet either needs to have a valid checksum or it must be marked in
order to skip verification by the software stack.

The checksums are the key issue the controller has to solve to support
segmentation offloads over tunnelling protocols, as multiple checksums might
need to be computed along the way to the inner \a{TCP} packet, while it is the
\a{TCP} packet payload being segmented. Therefore, not only the \a{TCP}
checksum, but also the outer \a{UDP} checksum changes for e.g.\ \a{VXLAN}.

As a side note, \a{TSO} can increase performance in virtualized environments,
as it increases effective \a{MTU} of the virtual links without
breaking isolation.

\section{UDP Fragmentation Offload}

A similar but much simpler technique exists for \a{UDP} datagrams. The length of one
\a{UDP} datagram is limited to 64~KB, larger than \a{MTU} of common links.
\a{UDP} itself does not define any concept of fragmentation, instead it
utilizes \a{IP} fragmentation to deliver payloads larger than \a{MTU}.

Controllers might be capable of performing \a{IP} packet fragmentation on the
chip. In contrast to \a{TCP} segmentation offload, \a{IP} fragmentation
can be done without touching the transport layer \a{PDU}, thus only the \a{IPv4} header
checksum must be updated.

The offload techniques described so far are often referred to as \emph{Stateless
Offloads} despite the fact that e.g.\ \a{LRO} needs to maintain state. Because
of its convenience, we will use this terminology as well.

\section{TCP Offload Engine}

Some \a{NIC}s are equipped with a full \a{TCP} stack Offload Engine (TOE). Using this
engine, the host leaves full \a{TCP}/\a{IP} stack processing on the \a{NIC}.
There are two main approaches. In the less invasive one, the software stack
initiates the connection using regular mechanisms, then hands the stream over
to the \a{NIC}. From that moment, the \a{NIC} offers a complete stream
interface, and handles all the \a{TCP}-related work -- segmentation, congestion
control, retransmissions and so on.

The second and even more intrusive approach leaves the \a{TCP} connection
handling on the hardware from the very beginning. Essentially, the \a{TOE} driver
replaces the \a{TCP} stack completely.

While \a{TOE} might seem superior to partial offloads, it brings a lot of
controversy. Processing \a{TCP} is not a simple task at all, and complex code
cannot avoid bugs and security flaws. Updating a kernel is a simple task
compared to updating a firmware in a controller. Furthermore, the operating
system cannot control the extent of features provided by the \a{TOE}, and users
would probably be confused by missing features like firewall or \a{QoS}. This holds even more so
when we consider that these features are still configurable, but have no effect
on the offloaded \a{TCP} stack, because the respective packets bypass them.

\section{Multiple Queues}

A lot of opportunities to improve performance arise from creating multiple
queues for the communication between the host and the \a{NIC}. The maximum number
of queues in question is specific to the controller and depends on the
associated purpose. As we will see, multiple
queues can help in more cases than only on multi-processor systems.

\subsection{Multiple Send Queues}

Let us start with increasing the number of egress queues per port. To handle
multiple queues, the controller must multiplex them on the wire. The algorithm for
multiplexing frames from multiple queues in fact performs frame scheduling.

With the knowledge of the scheduling algorithm, the host can offload scheduling
to the \a{NIC} by selecting the proper queue for every frame. This requires
both the algorithm and the number of queues to be compatible with the desired
behavior.

We will show two examples of how multiple queues can be used. In the first one, assume that
the scheduling algorithm is round-robin, thus every queue is treated equally
(with respect to the number of frames, not bytes sent). In a sense, the host can
treat every queue as a separate interface to a shared medium. On
a multiprocessor system, it is possible to dedicate one queue to every core,
offloading the synchronization between the cores to the \a{NIC}. The downside of
this approach is that the bandwidth is not equally and deterministically divided
between processes, instead it is affected by the number of threads and thread
scheduling.

In the other example, consider a strict-priority scheduling algorithm. In this
model, the host can differentiate several traffic classes (e.g.\ prioritize
interactive traffic before bulk traffic) and distribute them among egress
queues appropriately. This technique can cut a bit of latency introduced by
buffering in the hardware queues of the device. Some scheduling still needs to
be done in the software, as there can be multiple sources of frames which
need to be multiplexed into one queue.

Scheduling offload is very hard to adopt, because subtle differences between
scheduling algorithms of the host and \a{NIC} can violate the intent of the
system administrator. The \a{NIC} driver must make sure that the current
scheduling algorithm is compatible with what the \a{NIC} employs.

Recent controllers offer some flexibility in terms of configuration of the
scheduling algorithm. For example, the Intel XL710 controller \cite{XL710}
defines queue sets, which can be arranged to a tree. Leaf nodes are queue sets
tat can be arranged in a tree. The leaf nodes are queue sets,
the inner nodes select its children in a configurable combination of weighted
strict priority and weighted round robin order. The bandwidth is distributed
equally among the queues in a set.

\subsection{Multiple Receive Queues}

Multiple ingress queues can be utilized quite
easily. The controller has to distribute the frames received from the wire to
the queues and the selection of algorithm the to do so creates space for offloads.

\subsubsection{Receive-Side Scaling}

To take advantage of multiple processors available for processing
network traffic, a simple mechanism called \emph{Receive-Side Scaling} was
specified by Microsoft in the Network Driver Interface Specification \cite{NDIS}.

The idea is that similarly processed frames should be handled by the same
\a{CPU} to maximize the benefits of caching. Also, frames from a single flow are
likely to be processed similarly. Therefore, an \a{RSS}-enabled \a{NIC}
extracts the source and destination addresses, possibly also the \a{TCP}/\a{UDP}
source and destination ports, and calculates a hash function of those. The lower
bits of the result are used to select the target queue. Queues are then uniformly
distributed among \a{CPU}s and the interrupt affinity is set accordingly.

One might dispute \a{RSS} being considered an offload, as it does not have a
software predecessor. We can take \a{RSS} as a reason why multiple receive queues
exists and as an inspiration for all other multi-queue offloads.

\subsubsection{Differentiated Services}

Instead of merely distributing frames in a stochastic manner, the queues can be
assigned a specific purpose. Every \a{IP} packet carries a \acrfull{TOS} field,
which can be used to assign a traffic class to every frame. Traffic classes can
then be mapped to receive queues by the \a{NIC}. Knowledge of the priority
mapping then allows the software stack to handle certain traffic classes with higher
priority than the others.

Differentiating services as soon as possible is important under heavy load. For
example, \a{TCP} avoids congestion by lowering the transmission rate when a segment is lost.
This means that the rate is only lowered when some network element decides to drop
a packet. Unless some more sophisticated algorithm to drop packets earlier is
employed, packets are dropped only when a queue is overfilled. That results in
queues being kept rather full, introducing delays for high-priority traffic as
well.

\subsubsection{Advanced Classification}
\label{offload:classification}

The \a{NIC} can select the target receive queue based on a more complex
algorithm, considering not only the \a{TOS} field in the \a{IP} header, but
also other fields. Usually, the set of fields the controller is capable of
matching on is limited. The variants might be mutually exclusive or fixed in
linear order of execution. Popular variants include:

\begin{itemize}
	\item Ethertype
	\item Source or destination \a{MAC} address
	\item Transport layer 5-tuple (protocol, source and destination address,
	      source and destination port)
	\item \a{VLAN} ID or other tunneling header fields
\end{itemize}

In addition to the predefined header fields, the \a{NIC} can offer a way to define
new header fields. The number of configurable fields tends to be very low. On the
other hand, several switch \a{ASIC}s are equipped with packet parsers fully
programmable by firmware updates, and therefore we can expect similar
flexibility in \a{NIC} controllers in a foreseeable future.

Usually, the matching rules are arranged into tables. There can be either a fixed pipeline of
tables, or a table hierarchy might be defined at runtime, or a combination of
both. Every rule carries a queue number, which is used whenever a packet
matches the rule.

Multiple modes of matching on a field can be distinguished. First, the field
might be examined for an exact value. Or, every rule can contain a \emph{value}
and a \emph{mask}. Rarely, the fields can be compared with ranges or
a longest-prefix match can be selected.

In the exact-match mode, a controller can use a special type of memory called
\acrfull{CAM} -- a memory where lookup of a row with a given value is done on all
rows in parallel. Or, it can use hash tables and store the tables in \a{RAM},
but that requires handling collisions.

The mask-value rules are used frequently in the networking world, e.g.\ for routing. For every rule
individually, the field value is first masked by the mask and then compared
with the expected value. In hardware, this operation can be realized on all
rules in parallel by using a \acrfull{TCAM} -- a memory addressable by keys in which
the rules can ignore individual bits. However, this requires the memory to store at
least three states per bit. Therefore, the flexibility is paid for with
much higher number of transistors for the same size of the table. Due to their
construction, \a{TCAM}s also consume a lot of power and take up a lot of space
on the chip. Therefore, tables placed in \a{TCAM} are usually a lot more
constrained than tables in \a{CAM} or \a{RAM} -- in both the total size of the
used fields and the number of rows.

Range matching can be implemented with a \a{TCAM} extended for this purpose
or mapped to regular \a{TCAM} matches. Such mapping can consume as much as $2
\cdotop \lceil \log_2 (B - A) \rceil$ \a{TCAM} rules to represent $x \in (A,
B)$ rule. Longest prefix matching is usually implemented by \a{TCAM}s with
priority such that the lookup result is always the first matching rule. Then,
ordering prefixes from the longest to the shortest makes \a{TCAM} perform
a longest-prefix match.

\section{Flow Offload}
\label{offload:flow}

In the last few years, controllers of the network interfaces started to resemble
switch controllers. It might seem strange at first, as \a{NIC}s have usually
very few external ports compared to switches. However, modern high-performance
\a{NIC}s usually feature \acrfull{SR-IOV}, which presents the controller as
multiple \a{PCIe} devices to the host. Usually, the first device is called
\emph{physical} function, the others \emph{virtual} functions. The virtual
functions can be handed out to virtual machines, removing the overhead of
communicating through the hypervisor. Usually, the virtual functions can be
restricted or partially configured via the physical function, supporting
virtualization deployments.

The \a{SR-IOV} can be seen as multiple virtual
ports going out from the \a{NIC}, naturally creating the need for switching.
The usual scheme is that the adapter external ports as well as the physical and
virtual functions are connected as external ports to an inner switch. Such
requirement becomes even more apparent with multi-host network adapters, such
as the Mellanox ConnectX-5 EN \cite{mlx5-pb} based adapters.

The existence of an inner switch is not hidden by the controller. Inner switches
are usually not that performant and flexible as fully-fledged switch controllers,
but we can already observe inner switches adopting useful features and
complexity of standalone switches.

When it comes to virtualization, a \acrfull{SDN} is usually
deployed. The ``software defined'' aspect means that the physical topology is
hidden, creating a virtual overlay network over the physical one.
To give an example, consider a data center running virtual machines, that are
migrated to balance load. Traditional network elements would react too
slowly to enable this scenario, so an externally configurable, more flexible
switch is used instead. The distributed nature of forwarding is suppressed by
a controller that controls multiple switches in the network and plans paths
for packet flows. \a{SDN}-configured switches do not try to find paths by
themselves, they need to be programmed by the controller explicitly. The
communication between the switch and the controller is realized by a protocol
designed for this purpose -- for example the OpenFlow protocol \cite{openflow}.

To perform forwarding in an \a{SDN} environments, switches feature a flow table. When
an unknown packet is received, it is forwarded to the controller. It then
identifies the flow and installs a rule into the switch flow table. The next time
a packet from this flow is received, it is forwarded without controller
intervention. As the memory of the switch is limited, unused flow table entries
are evicted.

This model is commonly described as a layered one, the controller being
a control plane and the forwarder being a data plane. While the control plane
focuses on flexibility and features, the data plane focuses on performance. This
separation can be observed in many instances in the networking world.

So far, the inner switches in \a{NIC}s do not have the required configurability
(e.g.\ they cannot communicate with the controller using the OpenFlow protocol)
to fully support \a{SDN}. Instead, software switches are being deployed. This
opens an opportunity for offloading, which is slowly being implemented by the
\a{NIC}s. Controllers with \a{SR-IOV} have the ability to offload the virtual
function selection.

To a person with theoretical education in networking, switching is a matter of
decision based on the destination \a{MAC} address, and all of this might seem
as an over-engineered solution. Nevertheless the destination \a{MAC} address is no
longer the only field used for selecting the destination port. Usually, the same
classification engine used for steering packets among receive queues is used
for the target port selection. As the receive queue is bound to one port, both
switching and receive queue selection can happen at the same time, such
a solution is not common though.

\subsection{\a{VLAN} and Tunnel Offload}

Another new responsibility of switches is tunnel handling. For several years, \a{VLAN}-handling
features can be found even in \a{SOHO} switches. The switch can
transparently emulate multiple networks by adding and removing \a{VLAN} tags on
the client ports. To improve virtual network isolation, the port traffic
can be usually filtered based on the \a{VLAN} tags. Similar features with different
tunneling protocols can be found in high-end switches.

When a host runs multiple virtual machines, it probably needs to be connected
to multiple virtual networks. In other words, the tunnel processing must be
moved from the standalone switch to the switch inside the host -- either the
software one, or the inner switch inside a \a{NIC}.

Presence of tunnelling prevents the flow offloading described in Section \ref{offload:flow},
because the virtual machine would see packets wrapped in the tunnel header.
Therefore, filtering and encapsulation/decapsulation of
tunnels must be done as well to enable flow offloading for these scenarios.

The problem of tunnel offloading is its future compatibility and flexibility.
It takes a lot of time since the controller is designed until it is used in
a network adapter available on the market. Furthermore, the card is expected to
last a few years. On the other hand, new tunnelling protocols are still being
developed (e.g.\ the Geneve protocol, first drafted in 2014, is still not
standardized but is in active development \cite{ietf-nvo3-geneve-06}). The
functionality of tunnel engines can be superseded by programmable actions
described in Section \ref{offload:match-action}.

\subsection{Access Control Lists}

Together with switching and/or advanced classification, access control can
be implemented in hardware. With all the classification apparatus, dropping
matched flows is just a simple extension to switching or queue selection.

Let us show one example where offloading the \a{ACL}s can play a significant
role -- a \a{DoS} protection. Suppose a server providing a service, which is
under \a{DoS} attack. Incoming requests undergo accounting that can consider
arbitrary properties of the request (source address, source subnets, or even
some domain-specific properties like the \a{API} key being used). Flows
exceeding the configured rate are considered malicious and a rule dropping the
frames of this particular flow is installed. From now on, the traffic from the
attackers can no longer disrupt the service, because the packets are dropped
before reaching the first software component.

This scenario could be handled even better by dropping the malicious flows earlier, for
example on the boundary router. But the flexibility and cost-effectivity of the
described solution is a thing to consider.

\subsection{Match-Action Offload}
\label{offload:match-action}

All the advanced classification offloads described so far can be further
generalized by moving closer to the hardware implementation.

A packet processor usually features a pipeline that can be described as
follows. When a packet arrives, it is parsed by a parser. Metadata and
individual header fields are extracted to a structure following the packet
through the pipeline. Then, a series of match-action steps is performed. In
every step, the packet is matched to a table previously described in Section
\ref{offload:classification}. From there, a Match
Index is obtained and used to lookup an entry in an action table, which
contains a chain of actions for every Match Index possible. Actions are usually
operations on the metadata (set field, copy field value to another field, etc.)
or the packet as a whole (drop, pop header). The action results might be
reflected in the packet data directly, or just in the metadata carried around.
At the end of the match-action pipeline, there is a deparser, which moves
the changed metadata back to the packet to reflect all actions. The last step is
necessary to avoid touching the packet data directly from the pipeline, which
is complicated.

This claim of the previous paragraph is supported by the existence and spread of the OpenFlow \cite{openflow}
protocol. The protocol serves for communication between switches and \a{SDN}
controllers, as explained in Section \ref{offload:flow}. The protocol operates
on tables, header fields and actions, and assumes that the switch packet
processing pipeline has the architecture described.

In the most common case the processing pipeline is fixed. It might be
linear or have branches, but usually cannot contain cycles. Every table in
a fixed pipeline has a predefined set of headers matched (and the mode of
matching), and a set of actions allowed.

For example, the switching decision of an ordinary switch can be described as a table
performing a lookup on destination \a{MAC} address. If the address is known,
the specified outgoing port is used. If not, the packet is mirrored to all
ports on behalf of the default action.

As another example, we can take the \a{VLAN} engine of a simple switch. This step
is performed in the egress pipeline of a port (specifically, after the forwarding database
is consulted). The table matches on the \a{VLAN} ID. When the engine is turned
off, the table is empty and the default action is to pass the packet without
modifications. If the engine is turned on for the port, the default action is
to drop the packet. When the port is connected to a trunk link, the table matches
\a{VLAN} IDs of the networks the trunk link belongs to, and the action for
all of them is to pass without modifications. If, on the other hand, the link is
connected to a host unaware of the virtual networks, the only entry in the table
matches the \a{VLAN} the host is connected to and the action is to pop the \a{VLAN}
header. Remember that the default action remains to drop the packet, so the
engine performs filtering in the same step as well.

Some switch silicons have the pipeline at least partially programmable.
The software can define a new graph of tables, which is inserted into the
pipeline. Those tables can match an arbitrary subset of headers matched and can be
used to offload almost any real-world packet processing scenario of known
protocols.

The programmability of the pipeline alone is not sufficient to implement
protocol-agnostic packet processing. For that to be possible, the parser needs
to extract fields from the unknown protocol to match on them. The OpenFlow
protocol (in version 1.5) defines a set of matchable fields, which
cannot be extended. That limits its usage for switches with programmable
pipelines \cite{openflow}.

As stated earlier, there are several switch controllers on the market that are
equipped with a parser programmable by firmware updates. Such a controller can be
used to implement the processing of protocols that are invented later than the
switch, or application-specific protocols in general.

And how does that relate to \a{NIC}s? We have seen many cases of \a{NIC}s adopting
features from switch drivers, and we expect that this is going to be another
one. Supporting this claim is the Mellanox ConnectX-5 controller, which already
supports the creation of a graph of tables which do match-action processing of
the flow (see Section \ref{mlx:pipeline}).



% Removed paragraphs for easier access:
%
% The pipeline can be formally described in the \a{P4} language\cite{P4}.
% A program in \a{P4} specifies two aspects of the pipeline: the header parser
% and the match-action pipeline. The language is expressive enough to describe
% many existing fixed pipelines, yet constrained enough to allow implementation
% of a compiler to partially-programmable pipeline configuration. There is
% a handful of \a{P4} compiler backends at the time of writing. The language was
% designed with both contemporary and future hardware in mind.
% 
% One of the software switches being deployed is Open vSwitch. The software
% system consists of a userspace daemon and a kernel module. The kernel module
% serves as a dumb forwarder, which forwards according to its \acrfull{FDB}. The
% kernel module however does not learn -- frames which are unknown to the \a{FDB}
% are trapped up to the userspace daemon. The daemon can be arbitrarily complex,
% because it runs in userspace. After the daemon decides what to do with the
% frame, it can install a flow into the kernel -- insert an entry into the
% \a{FDB}. From now on, the frames belonging to the flow are forwarded by the
% simple algorithm in kernel, which is faster.
%
% This model is not specific to software switches, actually, it was used in
% advanced hardware switches first. The layers are usually called a \emph{Data
% plane} and \emph{Control plane}.

% The
% industry developed such that (not only) data center providers need to load
% balance with respect to source IP address, differentiate servers according to
% transport-layer ports, forward traffic according to tunnelling protocols and
% so. Sure, all of these can be done ``properly'' following the layered network
% model, but a shift towards distributed and more powerful switching brings
% higher throughput and shorter, more deterministic latencies is apparent.
% 
% Still, we have not mentioned how this relates to offloading -- everything was
% purely about software yet. The thing is that switch controllers (such as in
% Mellanox Spectrum) might offer an interface, through which the software can
% install flows directly into the hardware, and the frames can be then forwarded
% without software intervention. As the resources in hardware are limited, the
% kernel module must be preserved to handle excess flows. One can say that the
% hardware is a data plane for which the kernel serves as a control plane. The
% Spectrum switch in question essentially behaves like a bunch of \a{NIC}s, which
% are capable of communicating together without any host assistance.
% 
% Similar trend can be observed in the \a{NIC} world as well. Controllers often
% possess a mechanism for fine-grained control of switching. Its implementation is
% usually joined with the classification engine explained above -- the receive
% queue selection is just extended with port selection. It is worth noting that
% this mechanism can be specific for ingress traffic from the adapter point of
% view, and does not affect egress traffic.

\chapter{Selected Controller Capabilities}
\label{chap:nics}

This chapter documents the features of contemporary \a{NIC}s with respect to
offloading. We have chosen the following controllers for thorough analysis:

\begin{itemize}
	\item Intel 82599 10 Gigabit Ethernet Controller
	\item Intel Ethernet Controller XL710
	\item Mellanox ConnectX-4
	\item Chelsio Terminator 6
	\item Netronome NFP-6000
\end{itemize}

The selection was influenced by the support for match-action offloading, which is the
primary focus of the thesis. We have chosen the most recent controllers
supported by the Linux kernel. The order is arbitrary, though we preferred
controllers with publicly available specifications, as we know more about them.

% Intel-specific abbreviations
\abbr{DCB}{Data Center Bridging}
\abbr{VMDq}{Virtual Machine Device Queues}

% Used by sections to refer to manuals.
\newcommand{\sect}[1]{}

\section{Intel 82599 10 Gigabit Ethernet Controller}
\label{nic:82599}
\renewcommand{\sect}[1]{\cite[#1]{82599}}

This controller was released in mid-2016 and since then, it was embedded into
many adapters from different vendors -- Intel, HP, Dell, and others. The
controller supports Ethernet speeds of 10 Gbps. It is connected to the host
through 8 lanes of \a{PCIe} 2.0. To the network, it can be connected through
2 independent interfaces. Compared to its older sibling, 82598, it supports
\a{LRO} and \a{SR-IOV} among other improvements,

Its full specification \cite{82599} is publicly available for download from the
vendor web portal. The specification covers both the supported features and their
configuration interface for the controller driver. Unless otherwise specified,
the information in this chapter is sourced from there.

In the Linux kernel, the driver responsible for this controller is called
\sw{ixgbe}. It is in the upstream since 2007, introduced by commit
\linuxcommit{9a799d71}.

\subsection{Checksum Offload}

The controller supports calculation of the \a{IP} and \a{TCP}/\a{UDP}
checksums. However, the pseudo-header checksum must be computed by the software. The
support is not generic, as the controller always inserts the checksum at offset
6 (\a{UDP}) or 16 (\a{TCP}) bytes from the beginning of the transport layer packet.
The offset to the transport layer packet is configurable though, so the driver
could exploit this to emulate the generic mode \sect{7.2.5}. The Linux driver
does not do so. \a{SCTP} CRC32 computation and validation is supported as well.

On the receive side, the controller supports \a{IP} and \a{TCP}/\a{UDP}
checksum validation only for bare \a{TCP} or \a{UDP} over \a{IP} packets, no
tunnels are supported \sect{7.1.11}.

\subsection{Segmentation Offload}

The controller supports \a{TCP} and \a{UDP} segmentation for transmission. However, the
\a{UDP} segmentation is not implemented as \a{IP} fragmentation of
a single datagram, but as splitting of the \a{UDP} datagram into more datagrams.
The maximum size of a packet to be segmented is 256 KB. No advanced tunnelling can
be involved, the engine can handle only up to two VLAN headers.

\acrfull{RSC}, which is Intel's name for \a{LRO}, dynamically keeps at most
32 flow contexts per port, and coalesces \a{TCP} segments of those flows.
\a{RSC} can be turned off and on individually for every receive queue.
Interestingly, the controller is unable to coalesce \a{TCP} segments transmitted
over \a{IPv6} (while it supports their transmit segmentation).

\subsection{Multiple Receive Queues}

There can be as many as 128 receive queues configured. The majority of the receive
pipeline is dedicated to queue selection \sect{7.1.2}. First, hardware
switching is performed, and a \emph{queue pool} is selected. From there, the
packed is examined by a variety of filters, which can select a concrete queue
in the pool. In terms of configurability of the pipeline, the controller behaves
differently when \emph{Virtualization} is enabled. In this context, Virtualization
refers to a mode where multiple software entities receive packets through the
controller, and the controller offers additional features to support the use
case. Otherwise, the available queues can be used to improve
single-host performance.

Let us first explore the pipeline when Virtualization is disabled. No switching
is performed, as all received packets are received by one operating system.
The whole mechanism is devoted to selecting one of the 128
queues available. It is up to the software how it will assign the queues, yet
few limitations exist, as we will see at the end of the pipeline.

The pipeline is constructed to consult a fixed sequence of tables.
Every table can either hit the packet and select a queue index, or miss the
packet. The first table that hits the packet determines the final queue  number. At the
end of the pipeline, there is a \a{DCB}\footnote{\acrlong{DCB}} and \a{RSS} ``filter'', which hits
every packet and determines the queue if it was not determined by the previous
filters already.

\a{RSS} and \a{DCB} might not cover all the queues available. In that case,
the driver can offload classification of packets and use the free queues to
return the classification result.

Let us walk through the individual tables in the pipeline.
\begin{description}
	\item[L2 Ethertype filter] \hfill \\
		Intended to steer packets of a specific ethertype to
		a particular queue. An example use case is an early classification of
		\a{LLDP} or \a{IEEE} 802.1X packets. This filter table is
		also used to mark the packet for other offloads (\a{FCoE}, \a{IEEE}~1588).
	\item[FCoE Redirection Table] \hfill \\
		Used to manually assign a queue based on 3 least significant bits of
		the Fibre Channel Originator/Responder Exchange ID. As those IDs can be
		assigned uniformly, the table can serve as a \a{FCoE}-specific \a{RSS}.
	\item[L3/L4 5-tuple Filters] \hfill \\
		These rules match on any subset of the transport layer protocol, the source
		and destination \a{IPv4} addresses and the transport-layer port used.
		Unfortunately, the fields can match only concrete values and not
		e.g.\ network prefix of an address. The filter is useful to steer specific
		flows to a dedicated queue. There can be at most 128 such filters.
	\item[SYN Packet Filter] \hfill \\
		\a{TCP} packets with the \texttt{SYN} flag can be steered into
		a dedicated queue to mitigate \texttt{SYN} flood attacks.
	\item[Flow Director Filters] \hfill \\
		\label{sec:82599-fdir}%
		An advanced flow classifier. Apart from selecting the receive queue,
		the packet is marked with a tag configurable by software. Flows can be
		matched either exactly (max. 8 K filters) or by hashing the input
		values (max. 32 K filters). Matching is available for the VLAN tag, the
		\a{IP} version, the source and destination \a{IP} addresses, the
		transport-layer protocol, and the source and destination ports.
		Furthermore, the filter can
		match on any 2 bytes in the first 45 bytes of the packet (offset
		defined globally).

		The matched flows can be dropped. The dropped flows can be either actually
		dropped or just redirected to a dedicated queue. The matched flows can also
		be tagged with a 15-bit software-selected unique identifier.

		The amount of memory dedicated for Flow Director filters is
		configurable. The memory is shared with the receive buffer for packets.
\end{description}

When none of the previous filters matched and selected the receive queue, \a{DCB}
and \a{RSS} takes place depending on the configuration. When the \a{DCB} mode
is enabled, it extracts the 2 or 3-bits of the \a{PCP} field from the \a{VLAN}
header (\a{DCB} assumes all packets are \a{VLAN} tagged) to select
a \emph{Traffic Class Index}. \a{RSS} computes the flow hash by a fixed
algorithm (with configurable random key) and selects a configurable number of
least-significant bits to compute an \emph{\a{RSS} Index}. Those two indices are
then used to compute a queue index. The software should refrain from using queues
assignable by this algorithm for filter targets.

When Virtualization is enabled, the queues are distributed evenly among queue
pools, with high-order bits of the queue index defining the target pool.
Virtualization in this context does not necessarily mean only \a{SR-IOV} -- the
controller allows a different mode called \acrfull{VMDq}, which can be seen as
switching performed only to select the high-order bits of the receive
queue. This mode is intended to be used along with a software switch, which can
use the classification information. In the \a{SR-IOV} mode, queue pools
correspond to virtual functions.

The inner switch does not learn. Instead, it consults a multitude of tables to
fill the target pool list. Among others, it consults the destination \a{MAC}
address with respect to unicast, multicast and broadcast tables separately.

The switch operates in one of two modes -- with replication enabled or
disabled. Replication allows to copy the received packet to multiple queue
pools at once. If replication is disabled, the software is responsible for
distributing packets among multiple targets, because the packet will be
received by one queue only. In this mode, the inner switch is used purely for
classification. When replication is enabled, the software can configure
which pools will receive broadcasts, which multicasts and so on. This mode is
better suited for use with \a{SR-IOV}.

\subsection{Multiple Transmit Queues}

For transmission, the controller also opens 128 queues. However, not all of
them are scheduled in every configuration. Depending on the use of Virtualization
and \a{DCB}, only 64 queues might be dequeued.

Either way, two scheduling phases are performed. First, the packets are
dequeued from the 128 queues to at most 8 packet buffers. Then, the packets are
taken out from the buffers and sent to the \a{MAC} for transmission.

Queues are distributed between queue pools (virtual functions) and traffic
classes. When there are multiple queues dedicated to one traffic class inside
a pool, they are always dequeued in a frame-by-frame round-robin order. Queues
are distributed between classes to reflect class priority 
(high-priority classes have fewer queues than low-priority classes, because less
traffic is expected to be buffered.)

The scheduling algorithms differ for every mode of operation, and their
description here would be a mere copy of \cite{82599}, section 7.2.1, and
therefore is omitted.

From the offloading point of view, in majority of modes the controller employs
a weighted-strict-order scheduling of different traffic classes. This fact
could be used to offload priority scheduler with an appropriate number of bands.

\subsection{Other offloads}

Apart from the classification offloads mentioned, the controller features two
security offloads: LinkSec (MACsec) and IPSec. For both of them, the software
must establish the Security Associations itself, and then install them to the
hardware tables. The hardware is capable of offloading AES-128. Both
authentication and encryption is supported, provided they are using the same
\a{SA} (IPSec only). LinkSec can also be configured from the management
controller. \sect{7.8 and 7.12}

Another inline functionality that is emphasized in the specification is the
\a{FCoE} support. The controller offers additional functions like Fibre
Channel \a{CRC} computation or \a{FCoE} segmentation. \sect{7.13}

\section{Intel Ethernet Controller XL710}
\label{nic:xl710}
\renewcommand{\sect}[1]{\cite[#1]{XL710}}

\abbr{VSI}{Virtual Station Interface}
\abbr{NVM}{Non-Volatile Memory}

Although the XL710 is not a direct successor of 82599, but more like a member
of another branch of Intel networking \a{ASIC}s, they are similar in design
principles and features.

The silicon was announced in 2014. It is embedded in adapters from Intel and
Lenovo, plus adapters from less known vendors from China. As all other controllers from
Intel, its specification \cite{XL710} is publicly available.

The Linux kernel driver for this controller is called \sw{i40e}, corresponding
to the 40 Gbps speed. The patches introducing the driver were posted even
before the specification was released, in 2013.

The controller offers two variants of connection to the network -- either
through 2 independent 40 Gbps ports or through 4 independent up to 10 Gbps
ports. The ports can be connected either directly to the medium or to an
external \a{PHY} using \a{MAUI}. The controller connects to the host through
PCIe 3 with 8 lanes. A simple calculation shows that the maximum network
bandwidth (80 Gbps) is slightly higher than that of the PCIe interface.

Part of the configuration presented here is stored in a non-volatile memory
off-chip, and is therefore persistent most of the time. Such configuration
changes the way how the device presents itself to the host and manages static
allocation of resources. We can say that this configuration is part of the
firmware, which is quite easily modifiable.

The controller presents itself as up to 8 physical functions. The important
difference from virtual functions created with \a{SR-IOV} is that the drivers
of the physical functions are in charge of configuring the virtual functions
assigned to them. There can be as many as 128 virtual functions in total,
arbitrarily divided among physical functions (configuration stored in \a{NVM}).

One could say that the multi-presence of physical functions introduces another
layer of virtualization, when 8 almost-isolated environments can be created to
run inner virtualized networks over physical wires shared by all of them.
However, any of the drivers for the physical functions can request
configuration of the global resources, such as the firmware and the
non-volatile memory. So, the physical functions cannot be just handed out to
customers to deploy their own virtual networks.

The controller supports a similar range of offloads as the 82599 controller. The
big difference between them is the support for various tunnelling protocols
across all the offloads: \a{IP}-in-\a{IP}, Teredo, \a{IP}-in-\a{GRE},
\a{MAC}-in-\a{GRE} (\a{NVGRE}), \a{VXLAN} and Geneve.

\subsection{Stateless Offloads}

Both checksum offloads and \a{LSO} are supported with the extended support for
tunneling. Surprisingly, no form of \a{LRO} is supported.

Regarding the checksum offload for TCP and UDP, the pseudo-header must be
computed in software. For \a{MAC}-in-\a{UDP} tunnels, the support does not
cover the outer \a{UDP} header. Instead, those protocols are (un)supported in
a generic way, because the software specifies only the total length between
the outer and inner \a{IP} headers, allowing the inner transport-layer checksum to
be computed without specifying the concrete tunnelling protocol in use. This
limitation should not pose a problem, because these protocols have only
optional checksum \sect{8.3.4.3, 8.4.4.3}. Similarly, the \a{LSO} engine
supports \a{TCP} segmentation of packets up to 256 KB even when they are
encapsulated in a tunnel.

\subsection{Multiple Queues}

The traffic to host is delivered through one of 1536 available queue pairs
(a transmit and a receive queue form a pair). The distribution of queues is
persistently configured for physical functions, which can then dynamically
assign them to virtual functions at runtime. While one physical function can be
assigned all 1536 queues, virtual functions are limited to obtaining up to 16 queues.

The internal switch architecture does not work directly with functions but with
entities called \glspl{VSI}, which represent generic packet destinations or
virtual switch ports. There can be as many as 384 \glspl{VSI},
representing physical functions, virtual functions, switch control ports,
traffic snoopers or just another target assigned to a physical function.

The internal switch configuration options are very rich. Just explaining the
internal switch architecture occupies 150 pages in the specification
\sect{7.4}. We will try to avoid going into detail.

The basic idea is as follows. Whenever a packet is received on one of the
physical ports, an outer \a{VLAN} tag (called service \a{VLAN} tag, or just
S-tag) is stripped and determines an ID of the internal virtual switch that is
then used for switching. This layer can be turned off, in which case the default
internal virtual switch is used.

There can be as many as 16 internal virtual switches, each spanning
a disjoint set of \a{VSI}s. Every switch can run either as a fully-featured
manageable switch (allowing \a{VM}-to-\a{VM} traffic), or just port
aggregator, which relies on external switch looping \a{VM}-to-\a{VM}
traffic back. At most one physical port can be connected to every switch.

All the internal switches are not learning, and they must be managed in
order to deliver any traffic. The forwarding database can be configured with
rules matching on the  Ethertype (for control traffic), the \a{MAC} address (optionally
hashed), the \a{VLAN} ID, or the \a{MAC} address together with the \a{VLAN} ID. Individual
ports can be marked as promiscuous for unicast and multicast traffic separately.

As the switching is realized on \a{VSI}s and not functions, the host can use
the switching capabilities to offload a software switch without incorporating
\a{SR-IOV}. Intel calls this mode of operation \a{VMDq} 2, which extends the
\a{VMDq} mode we have seen in the 82599 controller. The key idea is to assign
all concerned \a{VSI}s to the physical function and use the information created
when the packet is switched to accelerate the switching in software. In this
mode, the controller supports switching based on fields from up to the transport
layer header -- destination \a{IP} address, tunnel ID, inner \a{MAC} address in
a tunnelled packet or a combination of these.

After the list of target \a{VSI}s is created, the packet goes through a series
of filters very similar to those of 82599. We will not go through them again,
let us just have a look at changes.

The engine was extended with support for all the tunnelling protocols. That
especially means that the filters can be used on tunnelled packets (which miss
all filters on 82599) and that the tunnelling header fields are available to
match on.

The flexible field on which the Flow Director matches was extended to extract 16
bytes from up to 3 offsets within the payload of a packet \sect{7.1.4}. The
payload can be defined roughly as the first packet header that is not
identified by the internal parser.

Parser identifies fields in the first 480 B of the packet. From these, a 128B
field vector is constructed. The Flow Director filters are able to match on up
to 48 bytes. At most 8192 rules can be inserted in total. The memory is shared
for all functions -- every function has a configurable portion of guaranteed
space and the rest is available freely.

The RSS engine was extended to support multiple hash functions. Apart from the
Microsoft-defined Toeplitz hash function (used in other RSS implementations),
it supports a simpler variant when software needs to compute the value as well.
Also, the maximum number of queues used for RSS was raised to 64 queues, but
for physical functions only, because virtual functions can have at most 16
queues in total.

On the transmit path, the packet goes through a sequence of filters mainly
provided for security purposes (anti-spoofing, validating \a{VLAN} tag, etc.).
Then switching is performed to determine whether the traffic is local or
needs to be scheduled to a wire. As the switching topology is guaranteed to be
a tree rooted at the physical port, the controller can divide the available
bandwidth hierarchically. Every switching element can be configured to
divide its bandwidth between the entities connected to the element. At the lowest
level, \a{VSI}s can configure how their bandwidth is distributed among their
queues. Depending on the configured mode, the bandwidth is split among traffic
classes -- either at the root of the tree, under the switching elements or
under \a{VSI}s.


\section{Mellanox ConnectX-4}
\abbr{PRM}{Programmer's Reference Manual}
\abbr{MPFS}{Multi Physical Function Switch}
\abbr{E-Switch}{Ethernet Switch}
\abbr{TIR}{Transport Interface Receive}
\abbr{TIS}{Transport Interface Send}

This controller from Mellanox was announced in late 2014. It is a combined
network controller for both Ethernet and InfiniBand. The fourth version is not
the most recent one, as Mellanox already produces ConnectX-5 and develops
ConnectX-6. We decided to include ConnectX-4 mainly due to the
\acrfull{PRM}~\cite{mlx-prm} being public. However, the open driver for
ConnectX-5 is already merged into the Linux kernel, and we will try to provide
updated information where applicable.

Counterintuitively, the controller is driven by the \sw{mlx5\_core} driver in the
Linux kernel, while the \sw{mlx4} driver supports controllers up to ConnectX-3.
The rationale seems to be that ConnectX-3 is a 40 Gbps controller, hence the
suffix~4.

For complete configuration, an external collection of tools is needed. The
kernel drivers are dedicated to controlling individual network functions, not
the adapter as a whole.

The controller is capable of presenting up to 16 physical functions and up to
256 virtual functions using \a{SR-IOV}. Both of these need to be configured by
the vendor-provided firmware utilities. Only then can the virtual functions be
``activated'' by standard means.

Connection to the host is realized using PCIe Gen 3 x16. To the network, the
controller opens 4 or 8 \a{SerDes}\footnote{Serializer-Deserializer. Such
interfaces can be configured to support various \a{PHY} adapters.} lines with
25 Gbps throughput each, which
can be used to create an adapter with 2 physical 100 Gbps ports.

The global configuration of the controller is stored in non-volatile memory, and
needs to be configured using an out-of-kernel utility. Such configuration
includes mainly switching between the Ethernet and InfiniBand modes (if
applicable), enabling \a{SR-IOV} and the number of functions created.
Interestingly, the controller is capable of running one Ethernet and one
InfiniBand port.

\subsection{Switch Layout}

While the PRM~\cite{mlx-prm} provides enough information about the processing
pipeline, it does not describe the operation of the packet processing before
the packet reaches the \a{PCI} function. A brief explanation was provided in
a cover letter for patches introducing the \a{SR-IOV} support for the kernel driver
by Or Gerlitz~\cite{lwn-mlx-sriov}.

The controller has two layers of switches. The first one, \acrfull{MPFS}, is
responsible for switching unicast traffic among physical functions. Broadcast
and multicast traffic is always flooded. The \a{MPFS} is managed through
so-called L2 Table, whose configuration interface is covered in the PRM
\cite{mlx-prm}. Basically, the L2 Table matches packets based on the destination
\a{MAC} address and optionally the \a{VLAN} tag, and selects the target physical
function. Therefore, the \a{MPFS} is actually a very simple, non-learning
switch.

The second layer, \acrfull{E-Switch}, exists for every physical function.
It operates on entities called \emph{Vports}. Initially, there are two Vports
-- one for the uplink, one for the physical function. Unmatched traffic always
goes to one of these two, depending on the traffic direction. The
\a{E-Switch} must be configured with rules to direct traffic to virtual
functions as well. This is possible only from the driver of the physical
function.

For configuration purposes, every driver is responsible for managing its
own Vport context. Whenever the context is updated by a virtual function, the
driver of physical function receives an event. It can check the validity and
conformation to any local policy and then update the configuration accordingly.

\subsection{Pipeline}
\label{mlx:pipeline}

The pipeline is made of simple elements which are chained together. For any of
these elements, there can be multiple instances, resulting in significant
flexibility. Instances are created at runtime through a command interface.
Every instance can be configured separately.

A received packet arrives at a root \emph{Flow table}. Flow tables are allocated
in a \a{TCAM} and can match on all available fields. Those include fields from the
Ethernet, \a{VLAN}, \a{IPv4}, \a{IPv6}\ \a{UDP}, \a{TCP}, \a{VXLAN} and
\a{GRE} headers. For \a{MAC}-in-\a{UDP} tunnels, inner fields are supported as
well.

Flow tables contain \emph{flow groups}. A flow group defines a mask for all the
available fields, which selects bits that are required to match. A flow group
then contains one or more \emph{flows}, which define the matched value.
A matched flow can either be passed to the next stage, forwarded to another flow
table or dropped.

Newer versions of the controller extend the available actions to create
a flexible match-action pipeline. The possibilities now include manipulation
with tunnel headers (encapsulation, decapsulation), individual header
modification or even IPSec encryption and decryption. We can see the new actions being used in the
\linuxfnc{drivers/net/ethernet/mellanox/mlx5/core/en\_tc.c}{1859}{parse\_tc\_nic\_actions} function
in the \sw{mlx5} driver.

To prevent cycles when forwarding is used, every table must have a level
defined. Forwarding is possible only to a table with higher level. The root
table is the only table with level 0.

Once in the packet lifetime, multiple target flow tables can be specified. From
there, the packet is cloned and processed by multiple paths. The split is
possible only when forwarding from a table with level $< 64$ to a table with
level $\ge 64$, enforcing the unicity condition.

To illustrate the simplicity of the engine in the background, the controller
does not check whether the matched fields are valid in the context of the packet
-- for example, matching the destination \a{IPv4} address does not automatically
check whether the packet is \a{IPv4}. Unless the rules are programmed to check
the protocol first, they might be matching on garbage.

Another action that can be performed only once in the pipeline is tagging the
flow with an arbitrary \emph{Flow tag}. The tag is then reported out-of-band in
the completion event.

Once the last table is processed, the packet is forwarded to a specified
\acrfull{TIR}. This entity is responsible for performing stateless offloads,
which will be discussed further. Also, \a{TIR} can perform \a{RSS} by selecting
the target Receive Queue indirectly based on the output from a configurable hash
function and a redirection table. With regard to available hash functions, the
driver has similar flexibility as the Intel XL710 controller.
Again, the driver is responsible for configuring the Flow tables so that only
packets that are subject to \a{RSS} are delivered to \a{TIR} performing
\a{RSS}. If the tables are misconfigured, the result of the hash function is
undefined.

The receive queues are created dynamically as well. The responsibility of a~Receive
Queue is to store the packet and report to a Completion Queue, which may result
in interrupting the host. Unexpectedly, the Receive Queue might strip
a \a{VLAN} tag from the received packet first. We could not find this
feature used in the Linux kernel driver.

For transmitting a packet, the pipeline is reversed. First, the packet reaches
a Send Queue, where the packet is stored until it is to be scheduled. From
there, packets are withdrawn by \acrfull{TIS} instances, counterparts of
\a{TIR}. \a{TIS} is responsible for performing \a{LSO}, if applicable. Also,
a fixed priority is assigned to \a{TIS}, which presumably plays a major role in
scheduling.

One would expect a match-action pipeline to be implemented on egress as well.
It does exist, but at the egress of the \a{E-Switch}. Therefore, it is not
covered in the PRM~\cite{mlx-prm}. In the source code of the Linux driver, we
can see the rules being constructed in
\linuxfnc{drivers/net/ethernet/mellanox/mlx5/core/en\_tc.c}{2386}{parse\_tc\_fdb\_actions}.

\subsection{Stateless Offloads}

The controller supports checksum verification and calculation. As opposed to
Intel controllers, it computes the pseudo-header checksum by itself and ignores
the initial value of the checksum field.

Unexpectedly for such an advanced controller, all the stateless offloads are
supported only for pure \a{TCP}/\a{UDP} over \a{IPv4}/\a{IPv6}, no tunnelling
headers must be involved. This limitation is remedied in more recent versions
thanks to the possibility to peel off the tunnel headers before they reach the
\a{TIR}, or add them after they are processed with \a{TIS}.

\section{Chelsio Terminator 6}

% Refused to give specs.
% 
% driven by cxgb4
% 
% Good source: DPDK commit http://dpdk.org/dev/patchwork/patch/10340/
% 
% U32: https://www.spinics.net/lists/netdev/msg396215.html
% 
% conjecture: filters share memory with TOE, because TIDs ~= filters
% 
% adapter 1:1 sge 1:N queues
%     |__ 1:N TDI 1:1 filter / ...
% 
% classification: fixed parser
% 	- struct ch_filter_tuple
% 
% actions:  static int fill_action_fields()
% 	- gact shot --> drop
% 	- mirred --> switch (only if the port belongs to the same card obviously)
% 		- optionally, rewrite headers == NAT
% 
% Check cxgbtool
% https://lwn.net/Articles/542643/ Add support for Chelsio T5 adapter
% 
% --- 
% 

The controller ships in many variants, all of which enable a subset of
functions. While that probably allows Chelsio to set more suitable pricing,
it is quite confusing. For example, we cannot say for sure that any two features
described here can be used at the same time.

The architecture of the \a{ASIC} is best described in a paper published by
Chelsio \cite{chelsio-t6}, it however does not go into much detail. As the
vendor refused to share any information with us, we have to resort to guessing.
Many details are exposed in the manual for the proprietary driver extension
and configuration utility \cite{chelsio-uw}. Still, both manuals only
explain the capabilities of the controller from the user's point of view, not the
hardware capabilities in detail. For this kind of information, we had to
carefully examine the source code of the Linux kernel driver.

The driver in Linux kernel, \sw{cxgb4}, is shared for all controllers from the
Terminator series, starting from Terminator 4. Therefore, we know the union of
the capabilities of all these controllers, because we cannot differentiate
them. As we do not expect features to be removed in newer versions, we suppose
that the current state reflects the capabilities of Terminator 6.

As usual for modern high-end \a{NIC}, \a{SR-IOV} is supported. The controller handles up
to 256 virtual instances, mapped to virtual or physical functions. Even though
the controller can identify itself as 8 physical functions, only the first
4 are capable of \a{SR-IOV}. With the factory firmware, each of them can
control 16 virtual functions. However, some controller variants (probably
differing only in firmware) can be configured with up to 62 virtual functions per
physical function, giving us the maximum of 256 functions in total. Unfortunately, neither
the available documents nor the source code gives hints about the inner switch
functionality.

\subsection{Stateless Offloads}

The controller supports all of the stateless offloads. An interesting
requirement is placed on the software in case of \a{IPv4} header checksum for
\a{LSO} of \a{MAC}-in-\a{UDP} tunnels -- the driver has to compute the checksum
of the outer \a{IPv4} header without the total length field, which is different
for the last segment created.

For the receive checksum, the controller is capable of computing the checksum
of the received packet in a generic way, as described in Section
\ref{sec:linux-csum}. It does so as the first controller from our list.

As for the tunnelling support, it seems that \a{VXLAN} and Geneve is supported
for both checksum and segmentation offloads. Even though \a{GRE} is defined as one of the constants for
supported tunnels, it is not used elsewhere in the driver code.

\subsection{Match-Action Offload}

There is a step in the pipeline where custom classification is performed and
programmable actions are taken. The controller always contains a TCAM to
install up to 496 rules, and optionally also memory dedicated for half
a million hashed rules. The action part might seem a bit constrained at first,
but in the end it is quite powerful.

The match-action pipeline starts with a fixed parser, extracting a similar set
of fields we have already seen -- the \a{MAC} addresses, Ethertype, 2 layers of
\a{VLAN}s, \a{IP} addresses, \a{TCP} and \a{UDP} ports. In the kernel, the
available fields are represented in the
\linuxstruct{drivers/net/ethernet/chelsio/cxgb4/cxgb4.h}{1009}{ch\_filter\_tuple} structure.

From these fields a compressed vector is constructed to be matched on. The
compressed vector contains only the \a{IP} addresses and the L4 ports, extended with up
to 36 bits of the fields from above. It is very important to note that the
selection of bits creating the match vector is global and changeable only by
reinitializing the controller.

The compressed vector is used to look up an entry in a table, be it a \a{TCAM} or
a hash table. In both cases, the software representation of the entry is the
\linuxstruct{drivers/net/ethernet/chelsio/cxgb4/cxgb4.h}{1042}{ch\_filter\_specification} structure.
From its layout, we can clearly see the possibilities.

An interesting difference from the other vendors is that there is a fixed number of
slots for filters, and the slots are allocated by the software. The order of
filters is not arbitrary, as filters with lower indices have higher priority.

For matched flows, exactly one of three actions can be taken: pass, drop or
switch. When the packet is passed, its receive queue can be selected (otherwise
it is selected by \a{RSS} automatically). More possibilities open with the switch
action.  First, the switch action instructs the controller to loopback
the packet to a port. However, it can also modify some header fields. Namely,
it can alter the \a{MAC} addresses, push, change, or pop the VLAN tag and/or modify the \a{IP}
addresses and the \a{L4} ports. That means the controller is capable of offloading
e.g.\ routing with stateless \a{NAT}.

\subsection{Other Offloads}

Chelsio emphasizes a lot of other offloads the controller supports. The range
of additional functions the controller features is:

\begin{itemize}
\item \a{TCP}\acrfull{TOE}, % HACK
\item \acrfull{RDMA} over \a{iWARP} offload,
\item Both target and initiator offload of \a{iSCSI} and \a{NVMe-oF},
\item \a{FCoE} initiator offload,
\item Crypto offloads (IPSec, \a{TLS}, \a{SMB}, \dots),
\item Open vSwitch offload,
\item Bonding and active failover offload.
\end{itemize}

If the \a{NIC} driver had to support all of these, the driver would have to be
an ugly hybrid driver spread across the whole kernel. It is not the case,
instead, the upper-layer drivers (such as the \a{iSCSI} kernel implementation)
communicate with the controller through the \a{NIC} driver using a network-like
protocol, therefore the \a{NIC} driver itself does not care much about the
extended features.

\section{Netronome NFP-6000}

\abbr{FPC}{Flow Processing Core}

As the last examined controller, we have chosen Netronome NFP-6000. The design
of the NFP controller series is presumably completely different from other
vendors. Instead of being a highly specialized packet processing silicon, the
controller is a generic programmable compute unit with features for packet
processing and network connectivity. The Netronome controllers NFP-4000 and
NFP-6000 share the same architecture, but differ in the number of functional
units used.

Information in this section is based mostly on papers published by Netronome
\cite{nfp-4k-too,nfp-prm,nfp-micro-c}, with few bits deduced from the Linux
kernel source code.

The controller is used in adapters by Netronome exclusively. It is attached to
the host via up to four independent PCIe Gen 3 x8 interfaces, connecting to up
to four CPU sockets in one host. They can handle up to four 100 Gbps Ethernet
interfaces, with both integrated \a{MAC} or a transmitter connected via SFP+,
QSPF or CXP.

All controllers are composed from so-called islands, which have isolated
responsibilities. Most of the islands are connected with a bus, which can
exchange data between them. The architecture is modular, allowing to create
different versions of the same controller design.

The islands include:

\begin{description}
	\item[Ingress MAC and Packet Processing] \hfill \\
		Receives packets from the network interface. Parses headers, verifies
		checksums and constructs packet metadata. The packet payload is sent to
		memory units, packet metadata to Flow Processing islands.
	\item[Flow Processing] \hfill \\
		Performs arbitrary processing of packets, using the packet metadata.
	\item[Egress Packet Processing and \a{MAC}] \hfill \\
		Reorders packets from the same flows, and schedules them to the network.
	\item[ARM Subsystem] \hfill \\
		Contains a fully-featured ARM processor, which is able to run Linux.
		This processor can be used to configure or monitor the controller as
		well as run any other application.
	\item[Crypto] \hfill \\
		Specialized circuits to support Flow Processing units in encryption and
		decryption of arbitrary data.
	\item[Memory units] \hfill \\
		Globally-accessible memory units to store tables or data into. Apart
		from up to 30 MB of on-chip memory, there is an external memory unit which
		supports up to 24 GB off-chip DDR 3.
	\item[PCIe] \hfill \\
		Used for communication with the host. Packets sent by the host are
		processed similarly to ingress packets, the payload is stored in the
		memory and metadata are passed to Flow Processing islands.
\end{description}

An important building block is a \acrfull{FPC}, which is a programmable 32-bit
processor core designed for packet processing. The core runs up to 8 threads,
which are cooperatively scheduled when waiting for data, similarly to threads
on \a{GPU}s.

Flow processing islands are made of 12 \a{FPC}s each. They do the most of the
packet processing work. Apart from forming the Flow processing islands,
\a{FPC}s are spread in lower numbers in other islands as well, making even the
fixed parts of the pipeline programmable.

The \a{FPC}s can be programmed using an open source \a{SDK} to perform any
processing. The \a{SDK} provides a framework to program the packet processing
using the \a{P4} or C language, or allows to write programs running on bare
metal.

The processor is not fully featured, it has a simple architecture (e.g.\ cannot
calculate with floating point numbers, there is no stack and so on), but
certainly is more flexible than a match-action pipeline. The controller is
therefore able to do any packet processing for any application, provided the
program fits into the instruction memory.

If not configured and programmed by the user, the controller ships with
firmware, which emulates the behavior of a conventional \a{NIC}. The firmware
offers several ``apps'', which define the capabilities of the controller from
the operating system point of view.

As the capabilities of this controller are implemented by software, it does not
make sense to describe the controller according to the current version. We may
assume that the controller fits well into any reasonable offload model.

Taken to the extreme, the controller is capable of running Open vSwitch by
itself, on the \a{ARM} processor, offloading the heavy lifting to the \a{FPC}s.
Both without the host intervention. Still, the ports of the virtual switch
correspond to individual network interfaces presented to the host, creating
a very unusual scenario of a separate machine running Open vSwitch, connected
via PCIe to the host.


\chapter{Hardware Offloading in Linux}

Before we dive into how Linux supports various offloading techniques, how they
are configured and implemented, let us recall the development
model of the networking stack in Linux kernel. Linux
community does not publish any guidelines for network device vendors like
e.g.\ Microsoft does with its \acrfull{NDIS} \cite{NDIS}. In the Windows world,
vendors create controllers with the driver interface in mind, and are
restricted to features specified by \a{NDIS}. In the Linux world, both the
drivers and the driver interfaces follow the design of the hardware. The
development is spontaneous and lacks centralized control. Therefore, all available
solutions are somewhat improvised by definition.

\section{Ethtool}

Until recently, the only gateway to the hardware offloading features was the
\cmd{ethtool} utility. Its main purpose is to communicate with and
control the network device drivers. The name might suggest it is restricted to
drivers of wired Ethernet adapters, but it is not. It can be used to control e.g.\ WiFi
drivers as well. Before we review the implementation of various offloads, let
us have a look at the tool usage and output.

The general syntax for \cmd{ethtool} is as follows:

\begin{shell}
$ ethtool [action] <ifname> [arguments]
\end{shell}

Here \texttt{<ifname>} represents the (required) name of the network
interface, and \texttt{[action]} selects a particular functionality of the
tool, which can be further parameterized by \texttt{[arguments]} (both
optional).

The utility is developed along with the kernel and shares the kernel
versioning. It should be available in all Linux distributions.

\section{Features}

For every network interface, there is a set of flags called \emph{features}.
The flag set serves as a shared data structure at multiple places in the
kernel. Most notably, the driver uses them to report the capabilities of the
\a{NIC}, and the network stack to control them individually.

\begin{listing}
	\begin{shell}[fontsize=\scriptsize]
$ ethtool --show-features eth0
Features for eth0:
rx-checksumming: on
tx-checksumming: on
	tx-checksum-ipv4: off [fixed]
	tx-checksum-ip-generic: on
	tx-checksum-ipv6: off [fixed]
	tx-checksum-fcoe-crc: off [fixed]
	tx-checksum-sctp: off [fixed]
scatter-gather: on
	tx-scatter-gather: on
	tx-scatter-gather-fraglist: off [fixed]
tcp-segmentation-offload: on
	tx-tcp-segmentation: on
	tx-tcp-ecn-segmentation: off [fixed]
	tx-tcp-mangleid-segmentation: off
	tx-tcp6-segmentation: on
udp-fragmentation-offload: off
generic-segmentation-offload: on
generic-receive-offload: on
large-receive-offload: off [fixed]
rx-vlan-offload: on
tx-vlan-offload: on
ntuple-filters: off [fixed]
receive-hashing: on
highdma: on [fixed]
rx-vlan-filter: off [fixed]
vlan-challenged: off [fixed]
tx-lockless: off [fixed]
netns-local: off [fixed]
tx-gso-robust: off [fixed]
tx-fcoe-segmentation: off [fixed]
tx-gre-segmentation: off [fixed]
tx-gre-csum-segmentation: off [fixed]
tx-ipxip4-segmentation: off [fixed]
tx-ipxip6-segmentation: off [fixed]
tx-udp_tnl-segmentation: off [fixed]
tx-udp_tnl-csum-segmentation: off [fixed]
tx-gso-partial: off [fixed]
tx-sctp-segmentation: off [fixed]
tx-esp-segmentation: off [fixed]
fcoe-mtu: off [fixed]
tx-nocache-copy: off
loopback: off [fixed]
rx-fcs: off
rx-all: off
tx-vlan-stag-hw-insert: off [fixed]
rx-vlan-stag-hw-parse: off [fixed]
rx-vlan-stag-filter: off [fixed]
l2-fwd-offload: off [fixed]
hw-tc-offload: off [fixed]
esp-hw-offload: off [fixed]
esp-tx-csum-hw-offload: off [fixed]
rx-udp_tunnel-port-offload: off [fixed]
rx-gro-hw: off [fixed]
\end{shell}
	\caption{An \cmd{ethtool} command showing features of an Intel I219-LM \a{NIC} on kernel 4.16.}
\label{lst:ethtool-show-features}
\end{listing}

The \cmd{ethtool} command can be used to display the current state of all features with
the \Verb|--show-features| action, as shown on Listing \ref{lst:ethtool-show-features}.
In the first column, we can see all features that are supported
by the current kernel. Next, the output shows which features are currently enabled
on the interface by listing them as being \texttt{on}. In case the feature is
supported but disabled at the moment, it is shown as being \texttt{off}. In
case the feature is not supported, it is listed as \texttt{off [fixed]}. The
last case, \texttt{on [fixed]}, is for features that cannot be disabled,
because they do not correspond to offloads but rather to general-purpose properties,
such as the ability to use high memory for \a{DMA}.

For features that can be turned on and off from the userspace, one can use the
\cmd{ethtool} utility as well, namely with the \texttt{features} action. To give
an example, the following command can be used to enable \texttt{ntuple-filters}
and disable \texttt{generic-receive-offload}.

\begin{shell}
# ethtool --features eth0 ntuple on gro off
\end{shell}

Unfortunately, the feature names differ from those listed by
\Verb|--show-features|. To further confuse the user, not even the manual
reveals the mapping between the names explicitly, only a textual description is used.

In the kernel source code, the feature flags are defined as macros in the
\linuxfile{include/linux/netdev\_features.h} file. Throughout the following text, we
will reference the features using these macro names.

\subsection{Checksum Offload}
\label{sec:linux-csum}

Linux makes use of checksum offloading. As there are no knobs to tune the
behavior, the checksum computation is either offloaded or not. The only means
of configurating the checksum offload are the feature flags. However, it would
be hard to cover all the possible cases which the hardware can offload with any
static description -- imagine all the combinations of tunnel headers, \a{VLAN}
tags, \a{IPv6} extension headers, etc. Instead, the stack handles offloading
the checksum computation on a per-packet basis.

An extensive documentation of checksum offloading in Linux can be found in
\linuxfile{include/linux/skbuff.h}. A short overview follows.

The \skb{} structure carries multiple fields related to checksumming. Most
notably, the \field|ip_summed| field indicates the current state of the \skb{}
with respect to checksum. The meaning of its values differs for packets being
sent and received.

\subsubsection{Ingress Direction}

First, let us explore the receive path, which is simpler. The dedicated feature
flag \macro|NETIF_F_RXCSUM| is used to control receive checksum offload in the
driver. However, the stack does not rely on driver behavior, it always examines
the \field|ip_summed| field. When the driver receives a packet from
the \a{NIC}, it is too late to change the state of an offload. Therefore, the
driver is expected to use the meta-information given by the device to detect what
checksums were verified, and modify the \field|ip_summed| field accordingly.
The options are (not in original order):

\begin{description}
	\item[\macrofmt{CHECKSUM\_NONE}] the device did not perform any kind of
		validation.
	\item[\macrofmt{CHECKSUM\_PARTIAL}] the packet comes from a virtual source
		with offloaded transmit checksum, therefore the checksum is not
		expected to be verified.
	\item[\macrofmt{CHECKSUM\_UNNECESSARY}] some of the checksums were verified to
		be correct by the device. The zero-based index of the last verified
		checksum is denoted in the field \field|csum_level|.
	\item[\macrofmt{CHECKSUM\_COMPLETE}] the device computed the whole packet
		checksum, which the driver fills into the \field|csum| field.
\end{description}

From the point of view of the stack, the most flexible and future-proof option
is \macro|CHECKSUM_COMPLETE|. As the stack has to parse the headers anyway, it
is easy to compute their checksums and subtract them from the computed
checksum. Therefore, the \a{NIC} accelerates the verification in all layers without
having to understand them.
\label{tag:linux-rxcsum}

\subsubsection{Egress Direction}

The transmit path is a bit complicated with the presence of \acrfull{GSO}, which we
describe in the next section. For now, let us focus on plain checksum
offloading without \a{GSO}. In contrast to the receive path, when the packet
leaves any software entity (networking stack, driver), the entity must ensure
that the packet already has a valid checksum or the following entity is
prepared to compute it. The driver indicates its ability to compute checksums
by the feature flags. There are five of them:

\begin{description}
	\item[\macrofmt{NETIF\_F\_IP\_CSUM} and \macrofmt{NETIF\_F\_IPV6\_CSUM}]
		indicates the ability to compute\break one's complement checksum for
		\a{TCP}/\a{UDP} over \a{IPv4} and \a{IPv6}, respectively. Deprecated in
		favor of \macro|NETIF_F_HW_CSUM|.
	\item[\macrofmt{NETIF\_F\_FCOE\_CRC} and \macrofmt{NETIF\_F\_SCTP\_CRC}]
		indicates the ability to calculate \a{CRC} for \a{FCoE} and \a{SCTP} packets,
		respectively.
	\item[\macrofmt{NETIF\_F\_HW\_CSUM}]
		indicates that the driver can compute any one's complement checksum as
		defined by the fields in the \skb{} structure.
\end{description}

As for the configuration, the \cmd{ethtool} command controls all the flags at
once, it is not possible to selectively enable or disable only a subset of the
available checksum offloads using the current userspace \a{API}.

We can see that none of the flags consider the checksum of the \a{IPv4} header.
The reason for that is simple -- it is not expensive to calculate the checksum
for the 20 bytes of the header, especially when the header is constructed in the
software already. An obvious exception is an \a{IP} packet emitted by \a{TSO},
where the \a{IP} packets are constructed by the controller itself.

We can also see that there were attempts to cover the simplest cases, which
were then superseded by the generic one's complement checksum capability. The idea
is that the driver can check whether the controller will be able to compute the
checksum. If the combination of headers is recognized by the controller, the
driver instructs it to do so. In the other case, it just computes the checksum
in software. Still, the driver must be prepared to accept packets which are
already checksummed (or does not require any checksum to be computed in
general).

The slice of a packet to be checksummed in the generic cases (including
\a{FCoE} and \a{SCTP}) is defined as a suffix of the packet starting at
the position specified in the \field|csum_start| field of the \skb{} structure.
The driver shall ensure that the checksum will be written at offset
\field|csum_offset|. In case the controller does not support computing the
checksum in a generic way, the driver should check the values in these fields
to make sure they will be recognized by the controller.

In order to simplify the driver code, a helper \fnc|skb_csum_hwoffload_help| is
provided. Whenever the driver cannot be sure that the controller will compute
the checksum, it can just call the helper and it will compute the checksum in
software.

Seemingly, the situation complicates when tunnelling is incorporated, as there
are two headers which include checksums -- e.g.\ an inner \a{TCP} packet
wrapped in an outer \a{UDP} packet (with \a{IP} and \a{MAC} headers in between).
While it might happen that the \a{TCP} checksum is already computed because
the tunnel is only encapsulating traffic from elsewhere, it is definitely
possible that the traffic is local and therefore both checksums need to be
filled in.

There is a surprising clever trick. When the checksum of the outer packet is
computed, it is defined as a sum of partial checksums of all its parts. The
part that is most expensive to compute is the inner \a{TCP} packet with the payload.
However, since the inner packet carries its own checksum, the checksum of the
whole inner packet including the checksum field is not affected by its
contents. This means that the checksum field for the outer packet does not
depend on the inner \a{TCP} packet at all. Thus, the outer checksum can be
computed inexpensively in the software from the headers only, and the hardware
checksum offload can be used to compute the expensive inner checksum. This
technique is called Local Checksum Offload and was implemented by Edward
Cree \cite{linux-lco}.

\subsection{Segmentation Offload}

The Linux networking stack utilizes various segmentation offloads.
Moreover, it pushes the idea even further, and implements software techniques
to reduce the number of stack traversals. In the end, these software techniques
naturally extend to the hardware-offloaded techniques.

Quite isolated is a utilization of \a{LRO}. Again, once the driver
receives a packet, \a{LRO} is already done. Therefore, the only support which is
needed from Linux is a way to configure whether \a{LRO} should be enabled or
not. As expected, the \macro|NETIF_F_LRO| feature flag serves exactly this
purpose.

When \a{LRO} is not available or has to be disabled (e.g.\ because of routing),
there is still a way to coalesce packets to move data around in bigger chunks.
Linux implements so called \acrfull{GRO}. Before we explain it, it is necessary
to introduce the \a{NAPI}.

\a{NAPI} is a mechanism of the Linux kernel that reduces overhead induced by interrupts. When
a network device receives a packet, it copies it to a prepared \a{DMA}
buffer in the host memory, marks that buffer as valid, and interrupts the host.
A \a{NAPI}-compatible driver then does not receive the packet in the interrupt
handler itself, but schedules a polling softirq handler instead. Most
importantly, it disables the interrupt temporarily. The initial packet as well
as those received in the meantime are processed in a polling
loop. When no more packets are available, the polling loop ends and enables the
interrupt again. This way, packet processing is not being interrupted by
reception of new packets, resulting in considerably higher packet processing
rate.

When the packets are already received in batches, the \a{GRO} mechanism works
to merge them if possible. As the \skb{} structures are received by \a{NAPI},
a \struct|gro_list| of \skb{}s is built. Any newly received packet is first
compared with the members of the list, checking whether the two are similar enough to be
merged into one.

An important feature of \a{GRO} is that it is not limited to any particular
protocol layering. The mechanism is fully generic (hence the name), and
individual protocol handlers might decide what information can be lost by
merging packets. As a rule of thumb, packets that are candidates for merging
must contain the same sequence of headers and only a few selected fields might
differ.

The received packets are kept in the \struct|gro_list|, until they are delivered
(passed to the upper layer), which happens once any of the following condition holds:

\begin{itemize}
	\item The \a{GRO} protocol handler decides to deliver a packet. This happens for example when a \a{TCP}
		packet with any flag arrives, because flagged packets cannot be
		coalesced. Then both packets are delivered immediately, in the correct
		order.
	\item The \struct|gro_list| would become too large (more than 8 entries in
		the current kernel). Then the oldest packet is delivered to make room
		for the new one.
	\item The \a{NAPI} polling loop is over. Then all packets are delivered at
		once.
\end{itemize}

In contrast to \a{LRO}, \a{GRO} does not coalesce packets
in a lossy way. Merged packets not only have to belong to the same flow, but
also have similar characteristics like their timestamp. The goal is to allow
\a{GRO}-merged packets to be later split into the same segments on output.
Therefore, \a{GRO} can be used on routers and bridges as well.

Some \a{NIC} vendors, being aware of \a{LRO} limited usability, implemented a more
strict version of \a{LRO}, which passes enough metadata about the original
segmentation of packets to re-segment them later. These \a{NIC}s can then
offload \a{GRO} to the hardware. Examples include recent Broadcom and Qlogic
controllers. This feature is indicated by the \macro|NETIF_F_GRO_HW| feature flag.

The memory layout of the \skb{} structures allows for creation of fragmented
buffers. This feature is used heavily in \a{GRO}, as the packets are not copied
into one big continuous buffer -- instead, their fragment lists are
concatenated, which is a constant operation.

The counterpart of \a{GRO} is \acrfull{GSO}. The \a{GSO} mechanism segments
a super-packet just before it is passed to the driver -- so that the driver
code is not complicated by segmentation. The super-packets might come from
\a{GRO}, or be directly created from the data sent through a socket.

As we have seen, there are \a{NIC}s that can segment the packet in the
hardware, provided its protocol layering is compatible. When this is the
case, the network device indicates the situation using a feature flag, and the
segmentation is not performed. We can see \a{TSO} as one of the special cases
here.

There is yet another technique that sometimes allows to offload segmentation
of tunnelled packets to hardware without specialized hardware support. The basic
requirement is that the hardware must support segmentation, where the outer
headers are just bytewise copied.

Suppose that all segments of the packet (including the last one) are the same size. Then,
none of the outer header fields need to be changed when the inner packet is
segmented, including the outer checksum. The idea is similar to Local Checksum
Offload, checksum cancels out the changes in the inner packet, making the outer
checksum constant.

The initial requirement of having equally-sized packets is easy to achieve.
Instead of giving up on late segmentation, the payload is split in two. The
first is sized to an integral multiple of \a{MSS}, the second holds the last
segment of the different size.

Unfortunately, the \a{IPv4} ID will be the same for all packet segments.
However, \a{TCP} streams require \a{IPv4} packets to carry the ``Don't
Fragment'' flag, therefore the ID should not be examined by any network device,
as proposed in RFC 6864 \cite{RFC6864}.

Because the segmentation is only partially offloaded, this feature is called
Partial \a{GSO}. It is controlled by the \macro|NETIF_F_GSO_PARTIAL| feature flag.

\subsection{\a{TCP} Offload Engine}

In Linux, full \a{TCP} stack offload is not supported at all. It is not just
unsupported, it is actively rejected. There are many reasons for that,
summarized in the article about \a{TOE} at Linux Foundation Wiki \cite{lf-toe}.
The most relevant reasons for doing so include low flexibility in supporting
the solution, complicated updates and possible security flaws.

\section{Multiple Queues}

Regarding the reception of packets into multiple queues itself, there is not much
the network stack can do. The packet itself is received by the \a{NIC} driver,
which constructs an \skb{} structure and hands it out to the stack for further
processing. Similarly, using multiple hardware queues for transmission is
nothing complicated. However, the idea to scale the number of channels that
process the packets lead to several techniques.

As a quick way to see how many queues are available and  being used, one can
use the \cmd{ethtool} command:

\begin{shell}
$ ethtool --show-channels eth0
Channel parameters for eth0:
Pre-set maximums:
RX:		0
TX:		0
Other:		1
Combined:	63
Current hardware settings:
RX:		0
TX:		0
Other:		1
Combined:	63
\end{shell}

The numbers represent the maximum and currently configured number of queues. If the
device allows to use a different number of receive and transmit queues, they
are given in the \texttt{RX} and \texttt{TX} rows. When the device requires
to instantiate queues in pairs, one for each direction, the numbers are listed as
\texttt{Combined}.

\subsection{\acrlong{RSS}}

First, let us consider pure \a{RSS}. To benefit, the card must
be able to allocate and use as many hardware interrupt vectors as there are queues. Also,
the interrupt vectors must be pinned to individual \a{CPU} cores. This is done
automatically by the \a{NIC} driver, as the interrupt vectors have to be
registered even if \a{RSS} is not enabled.

In a sense, \a{RSS} goes directly against the effort of \a{NAPI} to reduce the number of
interrupts. Instead of packets being handled in batches, they are spread out to
multiple queues, which trigger individual interrupts to be processed.
To maximize the overall throughput, is most beneficial to configure one receive queue
per \a{CPU} core.

\a{RSS} is configured through the \cmd{ethtool} utility, namely its
\Verb|--rxfh| action. The current setup can be queried with the
\Verb|--show-rxfh| action.

\subsection{Receive Packet Steering}
\label{sec:rps}

Similar to software segmentation offloads, the Linux network stack utilizes
a software variant of \a{RSS} called \acrfull{RPS}. The key idea of both mechanisms is to
distribute packet processing to multiple cores as soon as possible. When
\a{RPS} is enabled for a given receive queue, then every packet received by
that queue is hashed and redirected to a \a{CPU} determined by the hash. This
happens as one of the first things after the packet exits the \a{NIC} driver,
in \linuxfnc{net/core/dev.c}{4018}{netif\_rx\_internal}.

\a{RPS} cannot be as beneficial as \a{RSS}, as the \a{CPU} handling the
packet first has to parse the packet headers just to compute the hash.
On the other hand, it is by far more flexible than the hardware implementation,
as it can handle multiple protocols, tunnelled packets, various hashing
functions and so on. It is also hardware independent, and can work even with
devices with a single hardware receive queue.

Unless handling mixed traffic that is only partially supported by \a{RSS} or
when the number of queues supported by \a{RSS} is significantly lower than the number of
\a{CPU} cores, it makes little sense to enable both \a{RSS} and \a{RPS}.

\a{RPS} in Linux is configured for every receive queue separately through
\texttt{sysfs}. In the directory of the queue,
e.g.\ \texttt{/sys/class/net/eth0/queues/rx-0/}, there is a file named
\texttt{rps\_cpus}, which contains the hexadecimal representation of a bitmap of
\a{CPU} threads where to redirect the packets using \a{RPS}. It is disabled by
default, one can enable it by setting bits in the bitmap.

\subsection{Receive Flow Steering}

As a third offload technique with similar acronym, Linux implements \acrfull{RFS}.
The idea behind it is very simple. Instead of moving the packet processing (and
the application) to the \a{CPU} where the packet was received, redirect the
packet to where the application is running and the packets are processed. It is
important to understand that while it might be worth moving the application
once, the scheduler might need to migrate it elsewhere to balance load.

When configured, \a{RFS} essentially only extends the lookup mechanism after
the \a{RPS} hash is computed. The lower bits of the hash are used to find an
entry in the global \struct|rps_sock_flow_table|. If the entry belongs to the flow
being examined, \a{RFS} steers the packet to the \a{CPU} given in the entry
instead of falling back to \a{RPS}.

Entries are added to the flow table by protocol layers, with every packet
processed or awaited by calling \fnc|sock_rps_record_flow|.
Entries are never explicitly removed, instead, they are being replaced by new
flows with the same low-order bits of their hash. The size of the table is
configurable through the \Verb|net.core.rps_sock_flow_entries| sysctl variable.

As there might be multiple \a{CPU}s waiting for the packet, the target \a{CPU}
field might change quickly, potentially delivering newer packets earlier than
older ones. To overcome this issue, two layers of tables are involved. The layer
that is actually used for steering is local to the receive queue, and the
target \a{CPU} for a flow in that layer only changes when no packets of that flow are
waiting in the current \a{CPU} queue. The size of the local table is
configurable through a the \Verb|rps_flow_cnt| \texttt{sysfs} file in the queue
directory. A more detailed description can be found in the message of the commit
implementing \a{RFS} \cite{linux-rfs}.

As we have seen, the controllers usually feature some classification mechanism
which allows to select the receive queue based on packet headers, and therefore could be
used to offload \a{RFS} to hardware. When the \macro|NETIF_F_NTUPLE| network device feature
is enabled, the \fnc|ndo_rx_flow_steer| callback is invoked with the information
about the flow and the target receive queue whenever a new flow is to be steered.
As receiving the packet to a wrong queue is not a serious error, there is no mechanism to
remove a flow from the hardware. The hardware is expected to recycle older flow
entries just as the software does.

Among the controllers we examined, only Mellanox ConnectX offloads \a{RFS} in
Linux. Instead of replacing the flows based on their hash value, it uses entries in
the Flow Table (as described in section \ref{mlx:pipeline}) in a circular order.

\subsection{Ethtool Network Flow Classification}

While \a{RFS} improves cache locality, it does so only for flows that have
been already seen. For example, if a dual-core system was running two virtual
machines on dedicated \a{CPU} cores, the
system would have to redirect half of the flows on average, because they would
be initially sent to the wrong \a{CPU} core by \a{RSS} or \a{RPS}.

Some \a{NIC}s (from our selection both the Intel controllers and Mellanox ConnectX)
implement an optional \cmd{ethtool} operation to support explicit flow
classification. Through the \cmd{ethtool} utility, these controllers can be
configured to steer the matched packets to a specified receive queue. This mechanism
has no software counterpart, therefore it is not considered to be an offload.
If the mechanism is not implemented or enabled in the \a{NIC} driver, it just
cannot be used.

The setup starts with the user invoking \cmd{ethtool} with the \Verb|--config-nfc|
action. Several common header fields can be specified for the
classification, among others the \a{L4} port, the \a{IP} addresses, and the \a{VLAN} tags.
Every rule then carries an index of the target receive queue where the matched
packets will be sent. If the queue is specified as $-1$, the packet is to be dropped by the
controller.

Later, this mechanism was extended to store the target virtual function in the
high-order bits of the queue number. This extension allows to select a receive
queue of a different network device than the rules are installed to -- quite
drastically changing the purpose of the mechanism. However, this was not
adopted by any other controllers than the Intel ones, because as we have seen, other
controllers perform switching separately from selecting the receive queue.

\subsection{Transmit Packet Steering}

In contrast to various scaling techniques on the receive side, there is no need
to scale on the transmit side. Packets are generated by applications, which are
naturally run by available \a{CPU} threads (as defined by local policy). The
only optimization the stack offers is \acrfull{XPS}.

The technique is essentially a careful hardware transmit queue selection based on the
originating \a{CPU}. For every transmit queue, a set of \a{CPU}s which may use
this queue for transmission can be specified. A reverse mapping is constructed
and whenever a \a{CPU} needs to send a packet, a queue is selected
using the flow hash, similarly to \a{RPS}. The queue number is
then recorded for the socket so that next packets are sent using the same
queue, preventing reorders.

\a{XPS} is mainly an optimization to moderate the congestions on queue locks.

\section{Express Data Path}
\label{sec:xdp}
\iltodo{Review this section.}

When talking about high-performance packet processing in Linux, we cannot avoid
mentioning \acrfull{XDP}. The idea of \a{XDP} is to allow the user to inject an
arbitrary program to process packets early in the stack, even before \skb{}
structure is allocated. The expected use cases include early packet dropping
for \a{DoS} protection or wire-speed load-balancing or forwarding.

Obviously, running a user-specified machine code in the kernel context is
usually considered a bug. In the case of \a{XDP}, a special restricted
instruction set is used -- the \acrlong{BPF}. \a{BPF} is designed to be safe when
executed in the kernel context. Most notably, \a{BPF} programs cannot contain
backward jumps and therefore they always ends after a finite number of
instructions. Also, the instruction set is simplified enough so that programs
can be verified whether they access only valid memory.

The initial usecase of \a{BPF} was to allow packet sniffers to filter packets
before they are passed to userspace \cite{bpf-usenix}. The user attaches
a \a{BPF} program to a socket, the kernel verifies the program and runs it
for every received packet. Depending on the result code, the packet is copied
to the sniffing socket.

In Linux, the original \a{BPF} instruction set was enhanced to better suit modern
processor architectures. Also, support for data structures like arrays, hash
tables or tries was added. Actually, the instruction set no longer reminds the original \a{BPF},
but the name has not changed. Sometimes, the enhanced set is called eBPF while
the original cBPF (classic \a{BPF}). In the Linux context, \a{BPF} almost
exclusively refers to eBPF.

There is an in-kernel \a{JIT} compiler from the enhanced \a{BPF} to the native
instruction set. Therefore, running \a{BPF} programs does not come with any
performance penalty. The compiler handles cBPF programs as well, by
transcribing them to eBPF first.

\a{XDP} programs are invoked as soon as a packet is received by the driver,
before \skb{} structure is allocated. That means that \a{XDP} needs to be
explicitly supported by the driver, even though it does not depend on the hardware at all.
The program executed in \a{XDP} can decide what to do with the packet using return codes.
It can just pass the packet, in which case it is processed as usual. The packet
can be also dropped, in which case the slot in the receive queue can be instantly reused
with the same backing pages, reducing the overhead of dropped packets to a bare
minimum. Finally, the packet can be modified and sent out using the same
network device.

Unexpectedly, \a{XDP} can be offloaded to the hardware. Currently only the
Netronome NFP controllers are able to do so, due to their software nature, but
we can expect more controllers to support running \a{BPF} programs in the future.

\section{Traffic Control}
\label{sec:tc}

\newcommand{\qdisc}{\a{qdisc}}

The Traffic Control (\a{TC}) subsystem exists for handling packets of different
classes in Linux. Before we get to the most advanced packet modification offload
techniques, we need to look briefly at how \a{TC} works.

One could say that the main purpose of \a{TC} is to assign a label (called
\emph{traffic class}) to every packet and then act on the packet depending on
its traffic class. For example, treat packets with interactive traffic with
priority.

Some manuals use the TC abbreviation for Traffic Class. To avoid confusion, we
will always refer to Traffic Control with \a{TC} in the thesis.

The \a{TC} subsystem was designed to support the Differentiated Services architecture in
its full flexibility -- it allows to differentiate packets into classes, and apply
policing, shaping, and scheduling. The subsystem is very flexible and generic,
thus we do not aim to describe it completely.

The subsystem is configured via netlink. However, there are only a few
userspace applications that control the \a{TC} subsystem. Usually, the
configuration is manual, in which case the \sw{tc} utility from the \sw{iproute}
package is used.

\subsection{Queue Disciplines}

The \a{TC} runtime configuration consists of a tree of Queue Disciplines, for
every network device and direction of travel. In both the \sw{tc} utility and
all the available literature, Queue Disciplines are called \qdisc{}s for short. We
will stick to this convention.

Let us consider the egress direction first. There, the \a{TC} subsystem serves as the main buffer
between the applications and the network interface. Whenever a packet is to be sent
by an application, it is \emph{enqueued}. When the driver is ready to emit
a packet (there is an empty slot in the hardware queue), a packet is
\emph{dequeued} from \a{TC}. The implementation of these two operations is what
defines a \qdisc.

Take for example the \texttt{pfifo} \qdisc{}, which is probably the simplest
one possible. It is a simple \a{FIFO} queue, for which the enqueue and dequeue
operations have the standard meaning. The number of packets in a queue is
bounded and when the queue is full, new packets are dropped.

There are two flavors of \qdisc{}s -- \emph{classless} and \emph{classful}.
The classful \qdisc{}s constitute the inner nodes in the \qdisc{} tree, as they
have a child \qdisc{} for every class of packets. The classless \qdisc{}s
are the leaves of the tree and actually store each packet
until it is dequeued.

One would expect the classless \qdisc{}s to treat all packets the same,
but that is not true. Classless and classful in the \a{TC} context refer
to the presence of child \qdisc{}s, not differentiating classes of packets.

A typical example is the \texttt{pfifo\_fast} \qdisc, which is classless. It
uses the \a{TOS} of the packet to select one of three bands\footnote{Band is
just another name for a traffic class, but let us avoid the term for a classless
\qdisc.}. Every band is a simple \texttt{pfifo}-like queue. The bands are
dequeued in a strict priority order. Therefore, this classless \qdisc{}
prioritizes packets depending on their \a{TOS} field.

Another good example is the \texttt{tbf} \qdisc{}, which is classful even though
it has only a single child class. The purpose of the \texttt{tbf} \qdisc{} is to shape the
traffic dequeued from the inner \qdisc{} using the Token Bucket Filter algorithm.

An alternative \qdisc{} taxonomy is suggested by the subsystem author, Alexey
Kuznetsov, in the short review in \linuxfile{net/shed/sch\_api.c}, where he
calls classless \qdisc{}s ``queues'' and classful \qdisc{}s
``schedulers''. The naming is not perfect either, because some ``queues'' (like
\texttt{pfifo\_fast}) can perform scheduling as well, while some ``schedulers''
(like \texttt{tbf}) might not. As most of the world seems to stick to the
class-based taxonomy, we will use it as well.

Classless \qdisc{}s can do more than just queue packets. For example there is the
\texttt{red} \qdisc{}, which implements the Random Early Detection algorithm to
prevent congestion. In short, this \qdisc{} randomly drops packets when it is
filling up, with the expectation that the flows that might cause the congestion
have higher probability to be selected, because they have a higher number of
packets going through. Because a dropped packet is usually a signal for the sender to
slow down (e.g.~in~\a{TCP}), this can prevent the flow from congesting the link
early.

Classful \qdisc{}s are what adds flexibility to the system. Their purpose is not
to directly enqueue and dequeue packets, but to select a child \qdisc{} to
delegate the operation to. Classful \qdisc{}s mostly differ in the dequeue
operation implementation, as they define in what order the child \qdisc{}s are
dequeued, thus performing scheduling. For example, the \texttt{prio} \qdisc{}
dequeues inner classes in a strict priority order.

As another example, the \texttt{htb} \qdisc{} can be used to divide the available
bandwidth among multiple classes using the Token Bucket Filter algorithm,
hierarchically (hence its name, Hierarchical Token Bucket). The exact scheduling
algorithm is quite complex, but allows the classes to share bandwidth with
guaranteed rates while allowing to steal the unused bandwidth from others.

The implementation of the enqueue operation might be interesting as well, but
is usually dominated by the need to select the child \qdisc{} based on
arbitrary packet characteristics. For this purpose, the \a{TC} subsystem
contains filters.

\subsection{Filters}

Filters are runtime instances of classifiers. Filters are attached to \qdisc{}s
and do the actual classification of packets. The separation of responsibilities between
\qdisc{}s and filters follows the Unix philosophy, where simple tools can
be stitched together to create complex policies. Many classifiers are
available to the user. If there is no classifier that would serve
the purpose, the user is encouraged to implement a new one.

\begin{figure}
	\begin{tikzpicture}[
			y = -1cm,
			qdisc/.style={draw,rounded corners, minimum width = 4cm, text depth=.25ex, text height=1.5ex, minimum height = 0.8cm,thin},
			place/.style={draw, inner sep=1mm, circle, thick},
		]
		\begin{scope}[shift={(5.7,-0.5)}]
			\node[qdisc,fill=hl-red] (red) at (0, 0) {\texttt{red} (classless)};
			\node[qdisc,fill=hl-blue] (tbf) at (0, 1) {\texttt{tbf} (classless)};

			\filldraw[rounded corners,fill=hl] (-3.2, -0.4) rectangle (-2.4, 1.4)
				(-4, -0.4) -- (-3.6, -0.4) -- (-3.6, 1.6) -- (2.4, 1.6) -- (2.4, -0.4) -- (3.2, -0.4) -- (3.2, 2.4) -- (-4, 2.4) -- cycle;

			\node at (0, 2) {\texttt{prio} (classful)};
			\node[rotate=-90] at (-2.8, 0.5) {filters};

			\draw[thick, ->] (-4.4, 0.5) to (-4, 0.5);
			\draw[thick, ->] (-3.6, 0.5) to (-3.2, 0.5);
			\draw[thick, ->] (-2.4,0) to (red.west);
			\draw[thick, ->] (-2.4,1) to (tbf.west);
			\draw[thick, ->] (red.east) to (2.4,0);
			\draw[thick, ->] (tbf.east) to (2.4,1);
			\draw[thick, ->] (3.2,0.5) to (3.6, 0.5);
		\end{scope}

		\draw[very thick,->,decorate,decoration={snake,post length=1mm}] (-.5, .5) -- (.5, .5);

		\begin{scope}[shift={(-2.7,0)},font=\ttfamily]
			\node[minimum size=1.3cm,draw,fill=hl,circle] (prio) {prio};
			\node[minimum size=1.3cm,draw,fill=hl-red,circle,below left=4mm of prio] (red) {red};
			\node[minimum size=1.3cm,draw,fill=hl-blue,circle,below right=4mm of prio] (tbf) {tbf};
			\draw[thick, ->] (prio) -- (red);
			\draw[thick, ->] (prio) -- (tbf);
		\end{scope}

	\end{tikzpicture}
	\centering
	\caption[Example \qdisc{} tree]{The composition of \qdisc{}s. Filter block attached to the
	\texttt{prio} \qdisc{} is used to select the inner \qdisc{}. Inspired by
	figures in \cite{diffserv-linux}.}
\end{figure}

Filters are invoked to look at the packet and decide whether the packet
belongs to some traffic class. Usually, filters directly select the child
\qdisc{} for the packet to be enqueued to. Multiple filters can be attached
to a \qdisc{}, in which case the priority assigned to them matters. Filters are
always created for a single network protocol only -- in particular, filters for
\a{IPv4} and \a{IPv6} are not shared and must be instantiated twice if
needed.

As one of the oldest classifiers, there is \texttt{u32}, the Ugly (or
Universal) 32-bit classifier. It is able to match on any 32-bit word on the
network layer. The match can also contain a mask, which selects the matched bits
individually. Moreover, the classifier can use the masked word to index
a hashtable. Implementation-wise, matching a concrete key and performing
a lookup in a hashtable always alternates, but both can be trivialized to match everything in the step.
The \texttt{u32} classifier tree can be created such that it is very efficient
compared to combining other classifiers.

We would like to mention the \texttt{flower} classifier as well. This
classifier uses the kernel flow dissector to extract the header fields and allows
to match on them. Initially it was called \texttt{openflow}, because it was
supposed to match on fields defined by the OpenFlow protocol \cite{openflow}. As
there is no reason to limit the classifier to these fields, it is extensible with
other common header fields.

An important distinction between the \texttt{u32} and \texttt{flower}
classifiers is that while \texttt{flower} uses
header fields identified by the kernel flow dissector, \texttt{u32} requires the
user to identify the headers and fields themselves. On the other hand, it is
possible to use \texttt{u32} for proprietary protocols without modifying the
kernel code.

\subsection{Actions}

Because the \a{TC} subsystem has the ability to classify packets, it seemed convenient to
use the classification for more than just selecting the target traffic class.
Initially, filters could return an action code, which could for example drop the
packet immediately (\macro|TC_ACT_SHOT|) or restart the classification
(\macro|TC_ACT_RECLASSIFY|).
\footnote{Actually, these were at first implemented by the policing framework of
\a{TC}, and the macros were prefixed with \macrofmt{TC\_POLICY\_}. Policing was
later generalized to actions.}

With the knowledge of filters and simple actions, we can finally introduce the
\texttt{ingress} \qdisc. As it makes little sense to shape or schedule incoming
traffic, the \a{TC} subsystem has limited purview on the ingress side.
Because there was no leaf \qdisc{} which would actually not queue packets,
a special no-op \qdisc{} called \texttt{ingress} was created.
The \texttt{ingress} \qdisc{} cannot have children, but performs
classification. Its only initial purpose was to classify traffic and perform policing
-- e.g.~drop traffic exceeding the configured bandwidth.

Finally, a need for executing multiple actions arose. To satisfy it,
another type of runtime entity was introduced -- an action. In the source code,
they are called filter extensions, but the user manipulates them as
actions. After an action is executed, it returns the action code, as the
filters initially did.

Some classifiers were extended to accept attaching multiple actions to them. When such a
filter matches, the first action is executed. Another action result code was
introduced (\macro|TC_ACT_PIPE|), which cause the next action to be executed. This
way, the user can program the subsystem to do a lot of packet processing in the
kernel.

For example, there is the \texttt{mirred} action, which can perform packet
mirroring or redirection. It can be used both to make the packet appear on
ingress of a different network device, or to be sent out with one.

As a special action there is \texttt{gact}, the generic action. It does
nothing except returning a specified action code, emulating the original
functionality of filters.

Several other actions are available to modify the packet data or metadata,
such as \texttt{pedit}, \texttt{skbmod}, \texttt{skbedit}, or \texttt{csum}. As
a special case of packet modification, \a{VLAN} tags and tunnel headers can
be added or stripped by the \texttt{vlan}, \texttt{ife}, or \texttt{tunnel\_key}
actions. The list is not exhaustive, new actions can be implemented.

The whole pipeline is not necessarily linear. There are action codes that
make the execution engine jump between actions or restart the classification.

\subsection{Actions on Egress}

The classification-action pipeline allows the user to perform a lot of packet
processing in the kernel. However, it required a classful \qdisc{} in order to
be able to execute filters and actions. Also, it was hard to configure \a{TC}
for both \a{QoS} and packet processing at the same time.

Therefore, Daniel Borkmann introduced the \texttt{clsact} \qdisc
\cite{linux-tc-clsact}. Essentially, it is an \texttt{ingress} \qdisc{} to which
two filter vlocks can be attached. One of them is used as the ingress chain as
with the \texttt{ingress} \qdisc, the other one gets executed for egress traffic
with a new hook. This way, the \a{TC} subsystem is invoked twice for an egress
packet -- first for the egress chain in the \texttt{clsact} \qdisc, second for the
root egress \qdisc.

\subsection{Modularity}

The \a{TC} subsystem is modular. \Acrshort{qdisc}s, classifiers and actions
can be distributed as kernel modules and loaded into the subsystem at runtime.
They can even be developed separately from the mainline. Therefore, it is not
possible to describe every possible behavior of the subsystem.

\subsection{Offloading}

As we have seen, there is a considerable overlap in the packet processing
capabilities between controllers and \a{TC}. This resulted in what kernel
engineers call \a{TC} offloading.

All the techniques described below are controlled by a single feature flag,
\macro|NETIF_F_HW_TC|. Unlike other features, this one is rather used to
disable the behavior -- enabling it might not change anything.

At first, \a{TC} offload was limited to offloading transmit priority scheduling
to support \a{QoS} \cite{linux-hw-qos}. The network device gained a new
callback, \fnc|ndo_setup_tc|, which was called to setup a number of traffic
classes. The driver then configured the hardware scheduler and the stack with
a mapping of classes to transmit queues. The stack then used the priority
queues for priority traffic, improving latency by skipping the hardware
buffers. The behavior was triggered by attaching an \texttt{mqprio} \qdisc,
which was created specifically for this purpose.

Quite recently, the mechanism was extended to also support other \qdisc{}s, and
most notably, classifiers and actions. So far, \texttt{mqprio}, \texttt{cbs},
\texttt{red} and \texttt{prio} \qdisc{}s and \texttt{u32}, \texttt{flower},
\texttt{matchall}, and \texttt{bpf} classifiers are at least partially
supported by some drivers.

The mechanism works as follows: whenever the \a{TC} configuration is modified in
a way interesting for some driver, an event is generated. This event is
announced to the driver through the \fnc|ndo_setup_tc| callback. The callback
is now effectively just a joined callback for different events, which are
distinguished by its argument of the \struct|tc_setup_type| type.

In reaction, the driver parses the event data and decides whether it is feasible to
offload what triggered the event. For example, the \sw{ixgbe} driver offloads the
\texttt{u32} classifier using the Flow Director filters described in section
\ref{sec:82599-fdir}. The scheme of the offload is as follows:

\begin{enumerate}
	\item The driver handles the block creation event (more about blocks
		later), in which it registers another callback.

	\item The callback is invoked to handle \texttt{u32}-specific events, such
		as creation or deletion of inner nodes of the \texttt{u32} tree. The
		most interesting case is the creation of a key node (\struct|tc_u_knode|),
		which performs matching of a concrete value, that is handled in
		\linuxfnc{drivers/net/ethernet/intel/ixgbe/ixgbe\_main.c}{8869}{ixgbe\_configure\_clsu32}.

	\item Even though the Flow Director rules support a flexible field, the
		offload does not consider it. Instead, it tries to match the hardware parser to
		the \texttt{u32} tree, identifying the matched fields. When the fields
		cannot be identified, the rule cannot be offloaded.

	\item Actions for the rule are parsed in \fnc|parse_tc_actions| in
		the same file. We can see that the driver recognizes actions to drop
		the packet (as redirection to the drop queue) and to redirect the
		packet to another function of the same device. If the action is not
		supported, the rule cannot be offloaded.

	\item A Flow Director rule equivalent to the \a{TC} rule is inserted into
		the hardware table.
\end{enumerate}

This generic procedure is followed by the drivers of all the examined controllers
as well. Offloading the \texttt{flower} filters is a bit simpler -- the driver does not
need to parse the fields, because the classifier matches on well-known fields.
The drivers vary mostly in the actions that are supported.

The mechanism works quite well as an offload. The user can use any \qdisc{}s,
classifiers and actions and provided the code is correct, compatible rules
are offloaded. But what happens when only a subset of rules can be offloaded?
As the offloaded rules are performed before they reach \a{TC}, how can the driver
be sure that executing the rules in different order preserves the policy
defined by the user?

The answer is unfortunate -- it cannot. Currently, offloading a subset of rules
can result in behavior which is different from that of \a{TC} in software. This
is one of the biggest design flaws of the solution, which we try to address
with the subsystem proposed in Chapter \ref{chap:rfc}.

A partial workaround is provided by the ability to control whether individual rules
can and will be offloaded. For relevant classifiers, two flags can be specified:
\texttt{skip\_hw} and \texttt{skip\_sw}. When \texttt{skip\_hw} is set, the
rule will not be offloaded. When \texttt{skip\_sw} is set, failure to offload
the rule will result in the rule being rejected by the software as well.
It is currently recommended to use these flags explicitly when \a{TC}
offloading is enabled.

\subsection{Shared Blocks}
\label{sec:tc-shared-blocks}

One of the root design features of \a{TC} is that all rules are specific to
a network device. Whenever a single controller presents multiple devices to the
system, \a{TC} must be be configured independently for each. Combined with
limited and expensive resources available for offloading the \a{TC} rules, this
independence started to pose a problem.

The issue was solved recently by Jiří Pírko with the introduction of shared filter
blocks \cite{linux-shared-blocks}. The patch introduces a new, global, runtime
entity of \a{TC} configuration -- blocks. These fit in between \qdisc{}s and
filters. When no shared block is specified, a new private one is created for the
\qdisc{}, preserving backwards compatibility. In the other case, two or more
\qdisc{}s may share the defined policy.

The boilerplate needed to support \a{TC} offloading, however, got more
complicated. Instead of handling events directly, the driver must register
a callback on a newly-created block. The callback will then receive events from
inside the block. Also, a future idea is that binding a block will replay all
events, which is not yet supported. Instead, binding a block with any offloaded
rule is forbidden.\footnote{Which is a bug confirmed by the author. Any
configured rule should prevent the block from being bound.}

When it comes to hardware resource utilization, there is still an important
unresolved problem. When the rules are to be compiled into table entries, it is
usually necessary to fix the table dimensions in advance. For example the Flow
Tables present in Mellanox ConnectX controller series must be allocated with
a maximum number of rows in mind, and the Flow Groups must have fixed masks.
Currently, the driver relies on grouping similar rules together and heuristics
in table size allocation.

\chapter{Proposed subsystem}
\label{chap:rfc}

The current state of \a{TC} offloading is not ideal. It somehow works in
practice, but certainly has some drawbacks. These drawbacks support the
motivation to explore other options. We would like to shortly review the most
important problems of the \a{TC} offloading:

\begin{description}
	\item[Broken partial offloading] \hfill \\
		When the user does not specify \Verb|skip_sw| flags, the hardware can
		offload only a subset of rules. It is not guaranteed that the policy is
		preserved.
	\item[Flexibility] \hfill \\
		While the flexibility of the \a{TC} subsystem is good for the user, it
		complicates offloading. The user works with graph of rules which may
		run programs, but the hardware needs tables and simple actions.
	\item[Historic API] \hfill \\
		The \a{TC} subsystem is complex and hard to understand. The code is
		burdened with almost 20 years worth of ad-hoc extensions. Its
		documentation is not up-to-date. Both developers of drivers and users
		have hard time understanding the runtime structures.
	\item[Bad error reporting] \hfill \\
		The only feedback from the subsystem with regard to offloading is
		a flag, whether a filter was offloaded or not. When it was not, the
		user has no way of knowing why.
\end{description}

\noindent With these drawbacks in mind, we would like to propose a subsystem, which would
overcome them. The subsystem is designed to be an offloadable representation of
a match-action pipeline, which is present (in restricted forms) in contemporary
\a{NIC}s. Along with avoiding the drawbacks of \a{TC} offloading, we focused on
following goals:

\begin{itemize}
	\item Allow to use as much packet-processing capabilities of modern \a{NIC}
		as possible.
	\item The subsystem must follow the Linux nature of being a hardware
		commoditizer, and allow a single configuration to run with any \a{NIC}.
		Of course, it can be offloaded only when the \a{NIC} is compatible, but
		the functionality should stay the same.
	\item When the user wants to program the controller directly, there are
		other solutions available. The subsystem should integrate well with the
		kernel datapath.
	\item Offloading the work of the subsystem should be as easy as possible.
		The \a{API} against drivers should follow this purpose the most.
	\item The hardware usually needs to allocate resources long in advance.
		Allow to restrict the resources in software to ease (or allow) the
		offload.
	\item The user should be able to understand the subsystem easily. The
		userspace utilities should interact with the subsystem through an
		understandable \a{API}.
	\item The offload must be restricted enough to preserve the defined policy.
		Even when it is not possible to offload the setup completely.
	\item Recent controllers have good classification engines, but not so rich
		possibilities in modifying the packets. It should be possible to
		offload the classification independently.
	\item It is not possible to drop \a{TC} from the kernel. The subsystem must
		not interfere with \a{TC}, if not used.
	\item There are multiple ways how to represent the policy for the scenario.
		Help the userspace with creating a setup which will be offloadable.
	\item When the subsystem is not used, it should not slow down the
		networking stack. We would like to avoid creating new hook as well.
	\item When the work cannot be offloaded, doing the work in software should
		not be extremely expensive, when it comes to performance.
\end{itemize}

\noindent The key idea is to replace the \a{TC} role in the \a{ACL}, flow, and
match-action pipeline offloading,
without actually replacing \a{TC}. This proposal cannot be considered final and
will be a subject to discussion at the networking mailing lists before a patch
will be created. Also, it is not meant as ``all or nothing'' -- we present many
ideas from which only a subset might be implemented in the end.

\section{The Big Picture}

The proposed subsystem is well separated -- it is not a module of
\a{TC} or netfilter, but rather a completely standalone entity. It has its
own \a{API} against the userspace and drivers. This way, we can focus on creating
an interface that is clean and straightforward to use.

The subsystem information base is stored per network namespace. In other
words, entities created from inside a namespace are visible for all network devices
inside that namespace. This way, we allow drivers to allocate resources once and
share them between multiple \netdev s of the same controller. This decision
is supported by implementation of shared blocks in \a{TC}, as mentioned in
Section \ref{sec:tc-shared-blocks}.

The subsystem has a userspace configuration utility and a kernel module.
The utility is a part of the \texttt{iproute2} package, and follows
the same usage principles. The kernel module does the hard work in packet
processing. It will be possible (and desirable) to control the subsystem
through the netlink \a{API}, the utility is mainly for observability and manual testing
purposes. The expected primary customer of the userspace \a{API} is a high-level
software tool (e.g.\ intrusion detection system, \a{SDN} controller, ...) More
about expected usage later.

The subsystem works with following entities: \emph{tables}, \emph{header
fields}, \emph{flows}, and \emph{actions}. These terms are somewhat overloaded in
the networking world, but in this chapter, we use them to reference the runtime
entities of the proposed subsystem, unless specified otherwise.

From the top level view, \emph{tables} form a directed graph. It
is prohibited to create cycles, but it is not checked by the kernel
module. There are several predefined types of tables, but their
behavior is not different from the ``generic'' type from the point of
view of the module. The purpose of type is to restrict the model in order to
ease offloading for simpler devices.

Similarly, there is a preconfigured parser of several known \emph{header fields}
(\a{MAC} addresses, Ethertype, \a{VLAN} tag, \a{IP} addresses, \a{TOS}, ports,
\dots). The parser tree provides a standalone
description of the packet header types and fields that can be extracted from them. This
parser is extensible by generic \emph{header fields} at runtime to support
custom protocols. A detailed description follows in Section~\ref{rfc:parser}.

When a table is created, the set of header fields it uses is
defined. This set is fixed and cannot be changed later. If the table is of one
of the known types, the set of usable fields is restricted to a predefined
subset. For example, a table of known table type ``\a{IP} 5-tuple filter''
could match on a subset of transport layer protocol, source and destination \a{IP}
addresses, source and destination ports. If the table is of the generic type, it
can use any subset of all the defined header fields.

Together with the header fields, a parser for the table has to be declared at
creation time. We expect that for most of the time, the default Ethernet parser
will be used.

We call the table entries \emph{flows}. The system is expected
to be continuously modified by inserting and deleting flows from tables. Every
flow has an individual chain of \emph{actions} assigned.

\begin{figure}[h]
	\begin{tikzpicture}[y=0.7cm]
		\fill (0,1) rectangle (7,0) [fill=gray!20];

		\draw[very thin]
			(3, -1) -- (3, 3)
			(6, -1) -- (6, 3)
			(7, -1) -- (7, 3);
		\draw
			(0, -1) -- (0, 3)
			(0, -1) -- (8, -1)
			(0, 3) -- (8, 3)
			(0, 2) -- (8, 2)
			(8, -1) -- (8, 3);

		\node at (1.5, -0.5) {\vdots};
		\node at (4.5, -0.5) {\vdots};
		\node at (6.5, -0.5) {\vdots};
		\node at (1.5, 2.5) {\dots};
		\node at (4.5, 2.5) {\dots};
		\node at (6.5, 2.5) {\dots};
		\node at (1.5, 1.5) {\dots};
		\node at (4.5, 1.5) {\dots};
		\node at (6.5, 1.5) {\dots};
		\node at (1.5, .5) {\dots};
		\node at (4.5, .5) {\dots};
		\node at (6.5, .5) {\dots};

		\draw[decoration={brace,amplitude=3mm},decorate,very thin] (0,3) -- (7,3);
		\draw[decoration={brace,amplitude=3mm},decorate,very thin] (0,-1) -- (0,2);

		\node at (3.5, 4) {Header fields};
		\node[rotate=90] at (-.6, .5) {Flows};

		\node[circle,inner sep=0.7mm,draw] at (7.5, 1.5) {};
		\node[circle,inner sep=0.7mm,draw] (ac1) at (7.5, 0.5) {};
		\node[draw,rectangle,minimum size=5mm] (ac2) at (9, 0.5) {};
		\node[draw,rectangle,minimum size=5mm] at (9.5, 0.5) {};
		\node[draw,rectangle,minimum size=5mm] at (10, 0.5) {};
		\node[draw,rectangle,minimum size=5mm] at (10.5, 0.5) {};
		\node[draw,rectangle,minimum size=5mm] at (11, 0.5) {};
		\draw[-Stealth] (ac1) to[bend left=45] (ac2);
		%\node at (2, 3.5) [text width=4cm] {Table};
		\node (fk) at (8.2,-0.5) [coordinate,label=right:Flow keys] {};
		\node at (10, 1.5) {Action chains};

		\draw (7,0) to (fk) [help lines];
	\end{tikzpicture}
	\centering
	\caption[Subsystem entities]{The entities in a table.}
\end{figure}

Now we can finally explain what the main processing looks like: Say that
the subsystem is called to process a packet using table $T$. The header fields
that $T$ uses are extracted from the packet using the parser of $T$, and are
joined to create a \emph{flow key}. The flow key is used to search the table
and identify a \emph{flow}. When the flow is identified, a sequence of
\emph{actions} is executed. If \emph{next table} was set while $T$ was processed,
the processing repeats with the next table.

To prevent infinite loops, the process stops after a constant number of
iterations. This limit is rather high and reaching it emits a warning in
a rate-limited manner. To prevent malformed packets from reaching the system,
the packet trapped in such loop is dropped.

\subsection{Matching modes}

Every table uses one of the predefined matching modes on the header fields. Those we
propose are inspired by options available in the hardware pipelines:

\begin{description}
\item[Exact] All used fields must match exactly.
\item[Hashed] The flow key is hashed and then only hashes are compared.
\item[Mask-value] Emulate a \a{TCAM} search. The flow key is compared only on
bits specified by the flow.
\item[Range] Every field is checked to be inside an interval.
\item[Longest-prefix match] A flow with the longest matching prefix is selected.
\end{description}

In the hashed mode, we do not specify the hash function to be used. The mode
serves as a hint for the hardware that it can afford collisions. In fact, the exact
mode can be implemented with hash tables as well.

\subsection{Actions}

Apart from action chains assigned to individual flows, the table contains three
more chains. The first is the \emph{default} chain, which is executed when no
particular flow is matched. The other two are a \emph{pre-} and
\emph{post-}chain, which are executed every time. The execution starts with the
pre-chain, follows with the flow chain (or default chain) and finishes with
the post-chain.

Any action chain can be changed at runtime. Modifying a chain can result in
change of table offload status -- either it can enable offloading of the previously
not-offloaded table or vice versa.

An action chain is composed of primitive actions. The set of available
primitive actions is specified by the subsystem and generally is not
expected to be extendable by kernel modules. By design, the action system is
much simpler than that of \a{TC}.

Primitive actions can be parameterized. In terms of implementation, we propose
a simple union containing possible argument types to keep things simple.

To create a common understanding of action purpose and flexibility, an
incomplete list of primitive actions follows:

\newcommand{\act}[1]{\texttt{#1}}
\begin{description}[noitemsep]
\item[\act{drop}] \hfill \\
	Drop the packet. Immediately stops processing.
\item[\act{set next table <T>}] \hfill \\
	After the processing of this table finishes, table \texttt{<T>} is
consulted next. Multiple usages of this action overwrite each other.
\item[\act{stop}] \hfill \\
	Stop processing this table immediately.
\item[\act{set field <field> <value>}] \hfill \\
	Modify the header field \texttt{<field>} to contain \texttt{<value>}.
\item[\act{copy value <field1> <field2>}] \hfill \\
	Copy the value from \texttt{<field1>} to \texttt{<field2>}.
\item[\act{set queue <Q>}] \hfill \\
	On ingress, set the receive queue to which the packet will be enqueued
	(makes sense only in hardware). On egress, set the transmit queue (makes
	sense only in software).
\item[\act{mirror to port <P>}] \hfill \\
	Send a copy of the packet to egress on port \texttt{<P>}. As the port is
	device-dependent, using such action disables offloading of the flow for all
	other devices.
\end{description}

Even though the action system is simplified, we could extend it with
conditionals and create a really flexible system. On the other hand, the
purpose of the subsystem is to create better offloading opportunities, and more
complicated packet-processing code should be implemented with different
technologies (\a{XDP}, custom firmware, ...).

To support more advanced features of modern \a{NIC}s, we generally prefer adding
a slightly complex primitive action than emulating it with several primitive ones.
The older or less powerful controllers are not able to offload more complex chains
anyway, and more specialized actions will result in less complex drivers.
A reasonable overlap in action functionalities is acceptable.

Let us extend the example of the ``IP 5-tuple filter''. Such filter can be defined
as a table with mask-value match mode to allow wildcard rules. The default
action is empty, passing the packet through. Flows inserted to the table
are created with the action chain containing only the \act{drop} action.
The same table could be created as a whitelist filter by using the \act{drop}
action in the default chain and empty chains for selected flows.

\subsection{Offloading}

To allow \a{NIC} drivers to offload the subsystem work, a separate \a{API} is
provided. Through this \a{API}, the driver registers for updates on a table.
Those will be delivered by invoking callbacks specified in a structure of
operations. Events will notify the driver of inserted flows and other
runtime modifications. At the time of registration to a table, the current
state is completely replayed (not in original order). This asynchronous
approach is needed to avoid locking the table for driver introspection.

There are two more details which we need to mention before we explain the
offload in detail. First, tables can be created with a maximum size specified. The
driver can use this property to allocate resources in advance. However, the
size can be changed at runtime, but the resize action can result in the offload
being stopped.

Second, action chains are stored separately from flow matching. When
a flow is matched, an action chain ID is obtained. The chain ID is then used to
do another lookup in a hash table of action chains. This allows for two major
optimizations: identical chains can be merged into one and classification can
be offloaded separately from action execution.

When a table is created, the driver can check for unknown fields or impossible
combination of fields, and map the table into the controller pipeline. If the
driver is rather simple one, it can just map known table types. If on the other
hand the controller is fully programmable, it knows all the information to
configure it.

Then, the driver can check the table-wide chains for unknown action primitives or
impossible action chains. Because of the purpose of the system, the driver can
be rather strict -- for example, filter tables should contain chains comprised
of either a \act{drop} action or nothing.

Let us start with the ingress path. There already is a flag
\field|tc_skip_classify| in the \skb{} structure which indicates that
the packet was already classified (and acted upon) by the hardware. We plan to
use this bit (see section \ref{rfc:tc} for better insight into this decision)
for the same purpose.

The driver configures the card to offload any prefix of the table graph. If the
graph can be offloaded completely, it can just mark the packet to be skipped
and deliver the packet to the networking stack. In the other case,
the graph was processed just partially and needs to be finished in software. By
vendor-specific means, the driver should know in which phase the processing was
interrupted, and therefore know where to continue. For the purpose of partial
offloading, the \a{API} of the subsystem exposes the executor state
structure, which should be created accordingly by the driver. Then, the driver
should call the executor of the subsystem to finish the processing.

The egress part is a bit trickier. The driver can instruct the subsystem to
completely avoid processing the packet in the software pipeline. This is
configurable per \netdev. An unprocessed packet is then received by the
driver. Again, if the table graph can be offloaded completely, no action needs
to be done in software and the packet can be given directly to the hardware.
Conversely, when the graph is not offloadable as a whole, the driver must
run the executor in software until it can be sure that all reachable tables are
offloaded. Then it can hand out the packet to the controller.

The subsystem allows partial offloading of a single table. As already said, the
controller can perform only classification, passing the action chain ID to the
software out of band. However, the subsystem in principle allows to offload
a subset of flows from a table, provided that the driver ensures the policy is
preserved. For example, the driver can evaluate that the ``IP 5-tuple blacklist
filter'' is idempotent for missed packets, thus can be performed both in the
hardware and the software. Then, offloading a subset of rules to the hardware
serves as a optimization.

However, such usage is discouraged, because the driver would have to
implement the algorithm to select offloaded flows, moving the complexity to
individual drivers. The encouraged solution is to offload tables of limited
size only, forcing the user to implement the rule aging in the upper layers.

\section{A note on OpenFlow}

A reader familiar with OpenFlow can notice that the subsystem can serve as
a backend to implement OpenFlow-compatible software switch. Indeed, this was
one of the design goals of the subsystem. If the proposed subsystem was
implemented, one would just need to write a daemon for communication with the
\a{SDN} controller and translate it into calls of the proposed subsystem
\a{API}.

\section{Compatibility}
\label{rfc:tc}

So far, we did not explain when and how is the subsystem going to be inserted
into the current software pipeline. An obvious answer would be to add another
hook in there, as close to the hardware as possible (probably under \a{TC}). We
decided that it is not necessary, and we would rather avoid doing it.

Instead, we propose implementing a classifier for \a{TC}. In the architecture
of \a{TC}, classifiers are in charge of executing actions, therefore it shall
be acceptable that our classifier would act on the packet on its own. As for
the \a{TC} actions, we can prohibit attaching them to our filter.

The model we propose is similar to the \texttt{bpf} \a{TC} classifier, which is already present
in the upstream. Also, \a{eBPF} programs can modify the packets as well, so the
separated action model should not pose a problem for the community.

To further support hardware offloading, we would extend the recently-added
block mechanism in \a{TC} with a special type of block. The proposed subsystem
would expose blocks of this type to the \a{TC}, and userspace could attach them
to \sw{clsact} ingress/egress hooks. As only one block can be hooked there, the
driver could be sure that the proposed subsystem pipeline is the only
processing that happens.

Summed up, what we propose is essentially replacing the \sw{clsact} hooks in
a non-intrusive way. This has the advantage of exposing our own userspace
\a{API} separate from \a{TC}, while reusing as much in-kernel infrastructure as
possible without sacrificing any of the goals. For example, we can reuse the
\fnc|ndo_setup_tc| callback up to the point where blocks are examined.

As the subsystem is very generic, its functionality naturally overlaps with
some more specific features of the Linux kernel. For example, it can
theoretically be used instead of \a{RFS}. It is probably not a good idea to do
so, as the software path would be almost certainly slower and the offloaded
table more generic than \a{RFS}, making its offload in fact more complicated.
However, it is up to the driver if it utilizes a part of its hardware pipeline
to offload \a{RFS} or the proposed subsystem.

For a concrete example, take the Intel Flow Director filters. They can be used
to offload \a{RFS}, Ethtool Flow Classification, or the subsystem we propose.
The driver can either make a fixed decision, switch between these offloads
during compilation, let the user select using driver-specific configuration
means, or use some heuristics to select the most beneficial offload
automatically.

One could argue that for generic, low-level packet processing the kernel
already contains \acrfull{XDP}. However, offloading classification is
considerably simpler than running arbitrary \a{BPF} program.

\section{Configurable parser}
\label{rfc:parser}

This whole section is proposed as an optional extension. It can be implemented
separately and afterwards.

Apart from using the kernel flow-dissector, we would implement our own packet
parser. The parser graph would be represented as a runtime entity, and thus
could be modified from userspace. The parser graph could be extended to support
new header fields or even new protocol headers.

The data structure representing a header field would carry a \fieldfmt{type}.
Every well-known header field would have its own type, and those types would be
still extracted by the flow dissector or taken from the \skb{} structure directly.
Only the fields whose type is ``generic'' would require parsing the packet
to extract their value.

The parser graph is composed of nodes representing various protocols. We
call the nodes \emph{headers}. Every header field is contained in exactly one header.
Headers describe how the parser walks the packets and identifies protocols.
Header fields describe how to extract a field value from a header instance.

Packet parsing would be similar to table processing. The parser
starts with a header defined by the link layer protocol -- for us, Ethernet. If there is
a header field which needs to be extracted from the Ethernet header, it is
extracted. Then, a mask-value match is performed on predefined fields to
determine the next header. If no next header matches, the parsing is over.

The predefined parser would be fixed, and it would not be possible to overwrite
the default rules and header fields. It would be however possible to extend the
parser with new fields and headers hooked up to places where the processing
would end previously. This is necessary to avoid changing the definition of
well-known fields.

If there are no unknown fields allowed in a table, the driver does not need to worry
about the configurable parser. If there are some, the driver can have a look
on the parser tree definition. In the simple case, the field would be just
a previously-unknown field of a known header. Several controllers feature
matching on a whole header or arbitrary part of it, enabling offload of those
fields. In the most complicated case, new parser states would be defined. If
the controller features a programmable parser, the driver can can program it
accordingly using the information from the parser tree.

\section{Introspection}

So far, the system was completely independent of the controllers present in the
system. In the real world, the user (or userspace software) would probably want
to know how to arrange the tables to make them most effective. We are proposing
introspection capabilities, which the userspace could utilize to optimize for
offloading.

The first challenge to be solved is that devices usually have only
partially programmable pipeline. For this purpose, the driver would export
a graph of fixed tables along with \emph{extension points} -- places where
generic tables could be created.

Next, we have to describe these fixed tables. Throughout the chapter, we complicated
matters by defining things both generic and static (well-known) -- tables,
parser states (headers), header fields. Now it comes in handy -- the driver can
use these values to describe known fields in a compact way. For globally
unknown but fixed entities, the driver can make use of the generic types.

Next, we have to describe these fixed tables. In common cases, the driver might
use the well-known table types. Where the table is fixed but not well-known, it
can be described by the generic table type, for which the usable header fields,
maximum size, etc. are defined (if applicable).

The hardest difficulty is to describe how much flexibility the user has in
defining generic entities (configurable parser, generic tables in extension
points). For headers and header fields, we think that describing the
configurability is of no use, because if the use case requires matching on
generic field or whole new protocol header, it probably cannot be avoided.
On the other hand, a description of the flexibility permitted in table definition could be very
useful. However, we think that we just cannot cover arbitrary hardware (contemporary
or future one) accurately enough, unless the description itself is
Turing-complete. But then it would just be too complex for the purpose.

Instead, we suggest to go similar way the \a{TC} offload used -- enforce
table offload by a flag. Whenever a table cannot be offloaded by the device,
the binding of its block should fail. Similarly, when a flow inserted to such
a table cannot be offloaded, or the table is resized beyond hardware limits, the
action should fail immediately.

Also, the userspace has the option of querying the device model name and
version, and looking up the available features in its own database. Such database
and the model it describes might be much more flexible than that in kernel,
which needs to be backwards compatible forever.

Note that while mixing offloadable and non-offloadable rules in \a{TC} can
result in unexpected behavior, the same situation does not happen here. The driver is
forced to offload a complete prefix (or suffix) of the pipeline -- marking table as forcefully
offloaded results in transitively enforced offloading of other tables. Say that
table $T$ is marked with \texttt{skip\_sw} flag. When bound to ingress, the
driver must offload all tables from which $T$ is reachable. When bound to
egress, all tables reachable from $T$ must be offloaded.

To support drivers in these operations, the table graph would be maintained
explicitly with lists of neighbors in both directions. This also allows the
userspace to look for cycles faster.

\section{Acting as a P4 Backend}

As the subsystem behavior closely follows the current hardware, we can use the existing
tools for programming the hardware pipelines. The \a{P4} language \cite{P4} serves exactly
this purpose. It allows a system administrator to define the behavior of
a pipeline in a restricted imperative language. The program can then be compiled
for a specific hardware pipeline. The compilers can already handle situations
where pipelines are partially programmable.

We allow for creation of a backend for such compiler. The compiler would
translate a \a{P4} program to a series of commands (or full state description) that
would emulate the desired behavior using our subsystem. The compiler could even
introspect the current hardware and optimize the program for it, resulting in
a pipeline which is well-offloaded.

\section{Simplifying Overlay}

We admit that the mechanism might be overly generic and flexible. It is however
necessary to cover all the different hardware designs and prepare for the
future. The complexity of the system does not prevent us from wrapping it in
a simple interface -- it would not be possible the other way around.

For example, a ``library'' could be created for drivers. The library could
serve drivers which are completely inflexible in terms of generic tables and
parsers. We could select several well-known table types and allow the driver to
register callbacks for flow insertion/deletion on these. The boilerplate for
drivers would be reduced to a bare minimum of implementing operations in the spirit
of ``filter this particular \a{VLAN} ID on this port'' or ``insert an entry
into FDB''.

Similarly, we could create a library for userspace. The user of the library
would be abstracted from the complexity of the system, and would work with
well-known tables only. Those tables could be provided with a sensible
interface, understanding how to parse and display addresses, ports, constants
and so on.

\section{Performance}

If we look at the system as a whole, we can say it is an emulator of
a programmable hardware pipeline. As such, we do not expect it to have
miraculous performance when run in software. Yet, if we compare it with
current state of the art, we expect it to be similar to \a{TC} with \sw{flower}
filters installed. It probably would not be faster than a hand-tuned \sw{u32}
classifier tree, which can be optimized much further. However, such optimized
solution is unlikely to be offloaded by any current driver.

Furthermore, we have the advantage of a more restricted behavior, and therefore
we could utilize some clever algorithms to speed up matching of mask-value
rules. The tables have the interface of a dictionary, unlike the \a{TC} rules, which are
a programming language. A dictionary could be optimized into a decision tree
automatically, while \a{TC} rules must be evaluated strictly in order to
preserve correctness.

Besides, we expect to solve an existing performance problem -- rule insertion rate.
Currently, inserting a rule into \a{TC} requires the \struct|rtnl_mutex| to be
taken. There are ongoing efforts to remove it, but the
parallelization of previously serial code is a notoriously hard problem. By
contrast, inserting a flow into a hash-table or a tree could be implemented
with fine-grained locking or even lockless data structures.

For use cases with maximum required software performance, neither our subsystem nor
\a{TC} is ideal. \a{XDP} serves those purposes much better. When the features
of the Linux networking stack are not required, it might be even better to
employ a kernel-bypassing solution like \a{DPDK}.


\chapwithtoc{Conclusion}

\iltodo{Review language.}

We succeeded in our goal to design a mechanism that would allow the user to
define a policy to process packets in the kernel, while allowing to offload the
policy into compatible \a{NIC}s. We did not realize the initial plan to pick
a few common features and create a specific mechanism for them, but instead we
proposed a generic mechanism to offload match-action processing.

We did so after carefully studying the pipeline of five high-end \a{NIC}s.
The subsystem we proposed is descriptive enough to represent the majority of
classification and packet-modification features these \a{NIC}s have while
being restricted enough to be offloaded easily.

Unfortunately, we have not succeeded in getting access to confidential
documents to support the research. Multiple vendors rejected to give us
information, that is not already public. Therefore, the thesis builds upon
public information only.

The proposed subsystem could replace the \a{TC} offload. It is better suited to be
offloaded to hardware, and thanks to a different approach it resolves some
serious problems of \a{TC} offload.

It is not the first work of its kind. Most notably, John Fastabend proposed
a Flow API \cite{flow-api}, that might seem very similar for a spectator. The
Flow API however served to precisely describe a fixed pipeline of the
controller and allow the userspace to program it directly, bypassing the
kernel. In contrast, our subsystem starts with doing the work in the kernel and
then offering the drivers to offload it.

Another project that is somewhat similar is the support for Flowtables in
Netfilter, implemented by Pablo Neira Ayuso \cite{flowtables}. The Flowtables
are limited to offloading forwarding, and take a ``reversed'' approach -- it is
the user who decides which particular rules should be offloaded. The drivers
create the tables and user fills them up.

To support our statements, we created a proof-of-concept implementation, which
is attached to the electronic version of the thesis. Otherwise, it is available
on Github\footnote{\href{https://github.com/Aearsis/mat}{\texttt{https://github.com/Aearsis/mat}}}.

The documentation of the implementation can be found in Appendix
\ref{app:doc}. The demonstration shows that the subsystem can be created, can
be used to process packets and that drivers can offload its work.

While studying the source codes, we discovered two bugs in the Linux kernel.
Both of them has been reported to the respective maintainers. Unfortunately,
only one of them replied and confirmed the bug. As for the other one, we are
currently testing the patch and will submit it soon.

In the future, we plan to send the proposal of the subsystem to the Linux
NetDev community.


\include{bibliography}

\listoffigures
% \listoftables

\printglossaries

\appendix
\chapter{The Demonstrator}
\label{app:doc}

The attached source code contains an implementation of the subsystem under
a~working title Match-Action Tables, MAT for short. In this appendix, we would
like to document the source code from a high-level point of view. We do not
discuss implementation details, as they are documented in the source code.
Also, we do not discuss the functionality itself, as it closely follows the
model presented in Section \ref{chap:rfc}.

You will not find an implementation of a Linux kernel module in the archive.
Instead, we created an environment where the implementation looks like if it
was written as a kernel module -- we could say that the MAT subsystem is
written against a \emph{mocked} kernel. Most importantly, all of the code
actually runs in the userspace, even though it simulates a code being run in
both the kernel mode and the userspace.

\section{Usage}

The source code is written in the C language and follows the GNU89 standard.
No libraries are required to compile it. The code still uses some Linux
headers for convenience, we do not expect it to be compilable on other
platforms. Due to the target audience, we do not consider it a problem.

The archive contains a \texttt{Makefile} that should compile all the sources by
running \texttt{make}. If the compilation was successful, you can run any of
the tests from the \texttt{tests} directory. Also, you can run the
\texttt{test.sh} script that runs all the tests and compares their outputs to
the expected ones.

The tests usually do some configuration and simulate receiving packets in
between individual configuration steps. As the subsystem produces debug messages to the
output, you can see what the individual components do. The expected behavior is
usually printed before running the steps and documented more extensively in
the source code.

The sources of the tests are more interesting than their outputs. They show
the expected userspace interface and the configuration of the subsystem. To
simulate packet reception, the test code acts both as the userspace and the
\a{NIC} driver, as shown on Figure \ref{fig:mat-comps}.

\begin{figure}
	\begin{tikzpicture}[
			fill=hl, very thin,
			mod/.style={minimum height=.8cm,minimum width=2.6cm,draw,fill=white,inner xsep=0.5em}
		]
		\filldraw (0,0) -- (0, 7.2) -- (7.5, 7.2)
			-- (7.5, 6) -- (0.5, 6)
			-- (0.5, 1.2) -- (7.5, 1.2)
			-- (7.5, 0) -- cycle;
		\node at (1.5, 6.6) {Test};
		\node[mod,minimum width=4cm] (tst-netdev) at (4.9, 0.6) {\netdev{} impl.};
		\node[mod] (tst-nl) at (4.05, 6.6) {setup};

		\filldraw (0.7, 1.4) rectangle (7.5, 3.1);
		\node at (4.05, 2.7) {Mock kernel};
		\node[mod] (krn-tc) at (2.2, 2) {TC};
		\node[mod] (krn-netdev) at (5.6, 2) {netdev iface};

		\filldraw (0.7, 3.6) rectangle (7.5, 5.4);
		\node at (4.05, 4.55) {Match-Action Tables};
		\node[mod] (mat-nl) at (4.05, 5.4) {netlink iface};
		\node[mod] (mat-block) at (2.2, 3.7) {block iface};
		\node[mod] (mat-netdev) at (5.6, 3.7) {driver iface};

		\draw[very thick] (5.6,1) -- (krn-netdev)
			(krn-netdev) -- (krn-tc)
			(krn-tc) -- (mat-block)
			(krn-netdev) -- (mat-netdev)
			(mat-nl) -- (tst-nl);
	\end{tikzpicture}
	\centering
	\caption[Components of the demonstrator]{The components of the demonstration implementation.}
	\label{fig:mat-comps}
\end{figure}

\section{Implementation Overview}

At the top level, we can split the implementation into three components. First,
there is the kernel mock. It defines structures like \skb{} or \netdev{}, but
also contains bits extracted from the internal API of \a{TC}. Also, we
extracted the implementation of linked lists and several general-purpose
macros. To avoid copying the \struct|idr| structure\footnote{Radix tree used to
allocate and map integer identifiers to objects.} implementation, we simply
used linearly increasing values and a very simple hashtable/array. Also, we
implement the kernel dynamic memory allocation routines \fnc|kzalloc| and
\fnc|kfree| using the \sw{libc} \fnc|calloc| and \fnc|free|, just to make the
code look like a kernel source code. In contrast, we decided not to implement
\fnc|printk| to make it obvious that the prints are for demonstration purposes
only.

Second, there is the implementation of the MAT subsystem. We will describe this
part in detail in the rest of the appendix. Together with the kernel mock, the
implementation is linked to a single archive \texttt{libmat.a} to represent
``the kernel''.

Finally, the archive contains a few test scenarios to demonstrate how the
subsystem is expected to be used. As already mentioned, the test scenarios
interact with the kernel from both userspace and hardware ends. Let us have
a~look at them first.

\subsection{Tests}

There are five tests included:

\begin{description}
	\item[\texttt{exact}, \texttt{hash}, and \texttt{tcam}] \hfill \\
		These configure a table of the given type, set the default action chain
		and insert a rule to drop matching packets. The tests demonstrate how
		the tables are configured and test the software implementation of the
		subsystem.

	\item[\texttt{simple-nic}] \hfill \\
		This test demonstrates the driver interface and the hardware offload
		mechanism. The emulated \a{NIC} contains a blacklist 5-tuple filter in
		its pipeline. The test configures a table which matches on IP addresses
		and simulates receiving packets. It demonstrates how the ingress
		processing is moved to the \a{NIC}.

	\item[\texttt{parser}] \hfill \\
		This test extends the configurable parser with the \a{VXLAN} header.
		The parser tree is printed before and after to demonstrate the
		extension. Then, a simple table is created to show how the driver can
		read the parser state to offload it.

	\item[\texttt{multi-table}] \hfill \\
		A chain of tables demonstrating a more complex pipeline is tested.
		The tables represent a non-optimal firewall and unicast \a{MAC} filter.
\end{description}

\noindent Please note that having multiple demonstrations of the offload
mechanism would require to implement all the functionality to simulate the
hardware processing, duplicating the functionality of the subsystem in the
tests. As the principles stay the same for other types of tables, we believe
one demonstration is enough to show the principles of offloading.

\subsection{Match-Action Tables}

\begin{figure}
	\centering
	\begin{tikzpicture}[
			text depth=0.25ex,
			text height=1.5ex,
			inner xsep=.5em,
			module/.style={rectangle,draw,minimum width=2cm,minimum height=1cm,rounded corners=1mm,fill=white},
			subsys/.style={rectangle,draw,minimum width=1cm,minimum height=1cm,rounded corners=5mm},
		]
		\node[module] (table) {table};
		\node[module] (netlink) [right=of table] {netlink};
		\node[module] (executor) [below=of table] {executor};
		\node[module] (tc) [left=of executor] {tc};
		\node[module] (parser) [above=of table] {parser};
		\node[module] (tcam) [left=of parser] {tcam};
		\node[module] (netdev) [left=of table] {netdev};
		\node[module] (uapi) [above=of netlink] {uAPI headers};
		\node[subsys] (userspace) [right=of netlink] {userspace};
		\node[subsys] (drivers) [left=of netdev] {drivers};
		\node[subsys] (TC) [left=of tc] {TC};

		\draw (tc) -- (executor)
			(netdev) -- (executor)
			(netlink) -- (userspace)
			(netlink) -- (parser)
			(table) -- (tc)
			(table) -- (parser)
			(table) -- (executor)
			(table) -- (netdev)
			(table) -- (netlink)
			(netlink) -- (executor)
			(TC) -- (tc)
			(netdev) -- (drivers)
			(netdev) -- (parser)
			(parser) -- (tcam);

		\node[module,draw=none,fill=none,right=of executor] {\bf MAT};

		\scoped[on background layer]
			\node [fill=hl, fit=(tc) (parser) (uapi) (executor)] {};
	\end{tikzpicture}
	\caption[Modules of the demonstrator]{The modules of our implementation. Highlighted is the scope of the
	MAT subsystem. Relations show communicating modules.}
	\label{fig:modules}
\end{figure}

The implementation of the subsystem can be split into several modules, as seen
on Figure \ref{fig:modules}. Not all dependencies are displayed.

The implementation is rather simplified. It is not meant to be
production-ready, but rather serve as a demonstration of thoughts and
principles. It should not even be used as a base for the real implementation,
because some of the things we ignore are hard to add as an afterthought. Most
notably, we purposefully ignore:

\begin{description}
	\item[Multithreading and synchronization] \hfill \\
		As synchronization in the kernel is vastly different from that of
		user\-space, we decided to omit it completely. This alone should be
		a reason to write the real implementation from scratch.

	\item[Deconfiguration] \hfill \\
		To deliver the ideas, it is necessary to support creating the
		configuration. However, there is a lot of bookkeeping code to support
		removing or changing the configured entities. We decided to keep
		things simple for the demonstration.

		Also, when things are not removed, they are usually never deallocated.
		We believe that the destruction routines are easy to imagine.

	\item[Error paths] \hfill \\
		As the implementation focuses on the ideas, we consider handling error
		paths an unnecessary noise. Also, as there are no routines to handle
		removing things, there is no sensible way how to handle errors.

	\item[Netlink method calling] \hfill \\
		Where the netlink interface is just a plain call to a function with
		fixed arguments, we decided to just call the function directly. The
		netlink protocol is designed to be very flexible and binary
		compatible, but not particularly readable. Hiding the actual interface
		into netlink messages would introduce unnecessary clutter.
\end{description}

\noindent However, we did not simplify the important things. We strictly separate the
kernel and userspace memory areas. Data structures can be shared only from the
userspace to the kernel, not the other way around. Such structures are declared
in a shared header file, which is separate from the implementation. As already
said, the header file contains also the definitions of functions, which can be
easily transformed into simple calls over the netlink interface.

The \texttt{netdev} and \texttt{tc} are simple modules that serve as brokers to
communicate with other parts of the kernel. In the real-world implementation, these
would probably get thicker to better separate the implementation from the drivers,
supporting the stability of the kernel \a{ABI}.

\subsubsection{Parser}

The \texttt{parser} module keeps the representation of the packet parser. The
parser is defined completely as a runtime data structure to allow extensions.
The module defines two important structures: \struct|mat_parser| and
\struct|mat_header_field|. Instances of both are identified by globally unique indices.

In the following paragraphs, we use the term ``parser'' for both the instances of
the \struct|mat_parser| structure and the parser tree. In fact, the parser tree is
nothing more than the root parser instance for the Ethernet header.

An instance of \struct|mat_parser| is created for every protocol defined. The
parser is used to identify fields from a packet of the corresponding protocol
and to determine the parser for the next-layer header.

The \struct|mat_header_field| structure instances correspond to individual
header fields. The structure contains everything that is needed to extract and
interpret the field value from the byte stream. In case the field is expected
to be used to match on, it is assigned to a parser.

Every parser (except for the innermost protocols) has two fields by default
-- \Verb|nexthdr| and \Verb|hdrsize|. The former identifies the field used
to determine the next-layer parser, the latter determines the distance to the
next header in bytes.

To define which parser will be used to parse the next header, a mask-value
match is performed. The rules are stored in an instance of \struct|mat_tcam| structure,
provided by the \texttt{tcam} module. The structure simulates a \a{TCAM}.

The preconfigured parser identifies some basic fields and shows how to solve
several challenges. As an example of a challenge, the \emph{Ethertype} field is
identified by the \a{NIC}s as the Ethertype of the last \a{VLAN} tag, or the
\a{MAC} header directly if no \a{VLAN} tags are present. As such header field can
be attached to at most one parser, we define the Ethertype as a standalone
parser and make the Ethernet and \a{VLAN} parsers ``look ahead'', out of the
area that is skipped by parsing the corresponding header. For illustration, have a~look at
Figure \ref{fig:ethertype-parser}.

\begin{figure}
	\begin{tikzpicture}[
		y = -0.8cm, x=7mm,
		helper/.style = {
			font=\scriptsize,
			draw,
			inner ysep=1mm,
			outer sep=0,
			text depth=0.25ex,
			text height=1.5ex,
			minimum width=1.4cm,
		},
		field/.style = {
			helper,
			fill = yellow!30,
		}]

		\node[helper] at (0,-1) {\a{MAC}};
		\node[helper] at (2,-1) {\a{MAC}};
		\node[helper] at (4,-1) {\texttt{0x8100}};
		\node[helper] at (6,-1) {VLAN};
		\node[helper] at (8,-1) {\texttt{0x0800}};
		\node[helper] at (10,-1) {IP};
		\node[helper] at (12,-1) {Payload};

		\node[field] at (0,0) {\a{MAC}};
		\node[field] at (2,0) {\a{MAC}};
		\node[helper,color=red] (type-1-u) at (4,0) {\texttt{0x8100}};

		\node[helper] (type-1-d) at (4,1) {\texttt{0x8100}};
		\node[field] at (6,1) {VLAN};
		\node[helper,color=red] (type-2-u) at (8,1) {\texttt{0x0800}};

		\node[field] (ethertype) at (8,2) {\texttt{0x0800}};

		\node[field] (ip) at (10,3) {IP};

		%\node at (-1, -1) [label=left:Packet data:] {};
		\node at (-1, 0)  [label=left:Ethernet:] {};
		\node at (-1, 1)  [label=left:VLAN:] {};
		\node at (-1, 2)  [label=left:Ethertype:] {};
		\node at (-1, 3)  [label=left:IP:] {};

		\draw[->] (type-1-u) to (type-1-d);
		\draw[->] (type-2-u) to (ethertype);
		\draw[->] (ethertype) to (ip);
	\end{tikzpicture}
	\centering
	\caption[An example of packet parsing]{Parsing a \a{VLAN}-tagged packet with the preconfigured parser.
	Parser states and their scope are in horizontal layers. Look-ahead fields
	shown in red, identified fields in yellow. Size not to scale.}
	\label{fig:ethertype-parser}
\end{figure}

It is important to note that the preconfigured parser tree should never be actually
used to extract fields. The values of well-known fields are already extracted
by the kernel flow dissector and thus present in the \skb. However, the
definition needs to be precise so as not to break the offload to the programmable parsers.

\subsubsection{Table}

The \texttt{table} module stores the configured tables. There are two main
structures defined: \struct|mat_table| and \struct|mat_flow_key|.

The purpose of the \struct|mat_table| structure is to keep all the information
about a table. Even though it is one of the largest structures in the codebase,
there is nothing unexpected.

The \struct|mat_flow_key| structure is a pure data carrier for flow keys. Both
its size and interpretation of content is determined by the table for which the
flow key is created. Therefore, the flow key is always associated with
a concrete table instance, even though it does not carry a pointer to the table.

When a table is created, a mapping of fields to the flow key bits is computed.
This part solves Bin packing problem to distribute variable-sized fields to
the least number of 64-bit \emph{parts} of the flow key. The number of parts is
then recorded in the table.

As the flow key interpretation depends on the table, it is the \texttt{table}
module that manipulates it, including filling the flow key with the data parsed
from the packet.

Every table holds its flows and action chains separately. Action chains are
stored in a hash table, in which the chains are identified by a unique 64-bit
number, \emph{Action ID}. In the future, the ID could be computed from the
content of the chain, allowing to merge identical chains into one.

Determining the Action ID varies with the table type. Tables of the \emph{Hash}
type use the flow hash directly, and therefore do not need to store any additional
data. Other types use different data structures following their purpose.

\subsubsection{Executor}

The software functionality of the subsystem is implemented in the
\texttt{executor} module. That includes both executing the action chains and
taking the packet through the pipeline of tables.

To allow fine-grained control by the device drivers, which might need to
resume the processing in software after offloading it partially to hardware,
the module exports its control
structure, \struct|mat_executor|. The structure contains the execution state
as well as a few variables that control the execution.

To start the executor from the beginning, the exported \a{TC} filter
initializes the structure and calls \fnc|mat_executor_run|. The executor
then performs all the necessary steps, until the final verdict is returned.

When the driver needs to start the execution from a different point (e.g.\ when
the hardware classified the packet, but is not able to execute the actions),
the driver initializes the \struct|mat_executor| structure, fills the known
information and uses the structure to resume the execution in software.

The \struct|mat_executor| structure holds a \a{TTL} number. With every step
executed, the \a{TTL} is decreased. When the \a{TTL} drops to zero, the
execution is terminated with a special result code.

When the \a{NIC} partially offloads the egress path, the driver has to run the
executor for any prefix of the table graph, such that the remaining suffix is
offloaded. The driver can either precisely choose the \a{TTL} to make the executor
stop at the right time, or just run the execution step-by-step.

The \a{TTL} field also protects from being caught in an infinite loop when the
subsystem is misconfigured.


% \chapwithtoc{Attachments}

\openright
\end{document}
