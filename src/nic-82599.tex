\section{Intel 82599 10 Gigabit Ethernet Controller}
\label{nic:82599}
\renewcommand{\sect}[1]{\cite[#1]{82599}}

This controller was released in mid-2016 and since then, it was embedded into
many adapters from different vendors -- Intel, HP, Dell, and others. The
controller supports Ethernet speeds of 10 Gbps. It is connected to the host
through 8 lanes of \a{PCIe} 2.0. To the network, it can be connected through
2 independent interfaces. Compared to its older sibling, 82598, it supports
\a{LRO} and \a{SR-IOV} (among other improvements).

Its full specification \cite{82599} is publicly available for download from the
vendor web portal. The specification covers both the supported features and their
configuration interface for the controller driver. Unless otherwise specified,
the information in this chapter is sourced from there.

In the Linux kernel, the driver responsible for this controller is called
\sw{ixgbe}.
%It is in the upstream since 2007, introduced by commit \linuxcommit{9a799d71}.

\subsection{Checksum Offload}

The controller supports calculation of the \a{IP} and \a{TCP}/\a{UDP}
checksums. However, the pseudo-header checksum must be computed by the software. The
support is not generic, as the controller always inserts the checksum at offset
6 (\a{UDP}) or 16 (\a{TCP}) bytes from the beginning of the transport layer packet.
The offset to the transport layer packet is configurable though, so the driver
could exploit this to emulate the generic mode \sect{7.2.5}. The Linux driver
does not do so. \a{SCTP} CRC32 computation and validation is supported as well.

On the receive side, the controller supports \a{IP} and \a{TCP}/\a{UDP}
checksum validation only for bare \a{TCP} or \a{UDP} over \a{IP} packets, no
tunnels are supported \sect{7.1.11}.

\subsection{Segmentation Offload}

The controller supports \a{TCP} and \a{UDP} segmentation for transmission. However, the
\a{UDP} segmentation is not implemented as \a{IP} fragmentation of
a single datagram, but as splitting of the \a{UDP} datagram into more datagrams.
The maximum size of a packet to be segmented is 256 KB. No advanced tunnelling can
be involved, the engine can handle only up to two VLAN headers.

\acrfull{RSC}, which is Intel's name for \a{LRO}, dynamically keeps at most
32 flow contexts per port, and coalesces \a{TCP} segments of those flows.
\a{RSC} can be turned off and on individually for every receive queue.
Interestingly, the controller is unable to coalesce \a{TCP} segments transmitted
over \a{IPv6} (while it supports their transmit segmentation).

\subsection{Multiple Receive Queues}

There can be as many as 128 receive queues configured. The majority of the receive
pipeline is dedicated to queue selection \sect{7.1.2}. First, hardware
switching is performed, and a \emph{queue pool} is selected. From there, the
packed is examined by a variety of filters, which can select a concrete queue
in the pool. In terms of configurability of the pipeline, the controller behaves
differently when \emph{Virtualization} is enabled. In this context, Virtualization
refers to a mode where multiple software entities receive packets through the
controller, and the controller offers additional features to support the use
case. Otherwise, the available queues can be used to improve
single-host performance.

Let us first explore the pipeline when Virtualization is disabled. No switching
is performed, as all received packets are received by one operating system.
The whole mechanism is devoted to selecting one of the 128
queues available. It is up to the software how it will assign the queues, yet
few limitations exist, as we will see at the end of the pipeline.

The pipeline is constructed to consult a fixed sequence of tables.
Every table can either hit the packet and select a queue index, or miss the
packet. The first table that hits the packet determines the final queue  number. At the
end of the pipeline, there is a \a{DCB}\footnote{\acrlong{DCB}} and \a{RSS} ``filter'', which hits
every packet and determines the queue if it was not determined by the previous
filters already.

\a{RSS} and \a{DCB} might not cover all the queues available. In that case,
the driver can offload classification of packets and use the free queues to
return the classification result.

Let us walk through the individual tables in the pipeline.
\begin{description}
	\item[L2 Ethertype filter] \hfill \\
		Intended to steer packets of a specific ethertype to
		a particular queue. An example use case is an early classification of
		\a{LLDP} or \a{IEEE} 802.1X packets. This filter table is
		also used to mark the packet for other offloads (\a{FCoE}, \a{IEEE}~1588).
	\item[FCoE Redirection Table] \hfill \\
		Used to manually assign a queue based on 3 least significant bits of
		the Fibre Channel Originator/Responder Exchange ID. As those IDs can be
		assigned uniformly, the table can serve as a \a{FCoE}-specific \a{RSS}.
	\item[L3/L4 5-tuple Filters] \hfill \\
		These rules match on any subset of the transport layer protocol, the source
		and destination \a{IPv4} addresses and the transport-layer port used.
		Unfortunately, the fields can match only concrete values and not
		e.g.\ network prefix of an address. The filter is useful to steer specific
		flows to a dedicated queue. There can be at most 128 such filters.
	\item[SYN Packet Filter] \hfill \\
		\a{TCP} packets with the \texttt{SYN} flag can be steered into
		a dedicated queue to mitigate \texttt{SYN} flood attacks.
	\item[Flow Director Filters] \hfill \\
		\label{sec:82599-fdir}%
		An advanced flow classifier. Apart from selecting the receive queue,
		the packet is marked with a tag configurable by software. Flows can be
		matched either exactly (max. 8 K filters) or by hashing the input
		values (max. 32 K filters). Matching is available for the VLAN tag, the
		\a{IP} version, the source and destination \a{IP} addresses, the
		transport-layer protocol, and the source and destination ports.
		Furthermore, the filter can
		match on any 2 bytes in the first 45 bytes of the packet (offset
		defined globally).

		The matched flows can be dropped. The dropped flows can be either actually
		dropped or just redirected to a dedicated queue. The matched flows can also
		be tagged with a 15-bit software-selected unique identifier.

		The amount of memory dedicated for Flow Director filters is
		configurable. The memory is shared with the receive buffer for packets.
\end{description}

\noindent When none of the previous filters matched and selected the receive queue, \a{DCB}
and \a{RSS} takes place depending on the configuration. When the \a{DCB} mode
is enabled, it extracts the 2 or 3-bits of the \a{PCP} field from the \a{VLAN}
header (\a{DCB} assumes all packets are \a{VLAN} tagged) to select
a \emph{Traffic Class Index}. \a{RSS} computes the flow hash by a fixed
algorithm (with configurable random key) and selects a configurable number of
least-significant bits to compute an \emph{\a{RSS} Index}. Those two indices are
then used to compute a queue index. The software should refrain from using queues
assignable by this algorithm for filter targets.

When Virtualization is enabled, the queues are distributed evenly among queue
pools, with high-order bits of the queue index defining the target pool.
Virtualization in this context does not necessarily mean only \a{SR-IOV} -- the
controller allows a different mode called \acrfull{VMDq}, which can be seen as
switching performed only to select the high-order bits of the receive
queue. This mode is intended to be used along with a software switch, which can
use the classification information. In the \a{SR-IOV} mode, queue pools
correspond to virtual functions.

The inner switch does not learn. Instead, it consults a multitude of tables to
fill the target pool list. Among others, it consults the destination \a{MAC}
address with respect to unicast, multicast and broadcast tables separately.

The switch operates in one of two modes -- with replication enabled or
disabled. Replication allows to copy the received packet to multiple queue
pools at once. If replication is disabled, the software is responsible for
distributing packets among multiple targets, because the packet will be
received by one queue only. In this mode, the inner switch is used purely for
classification. When replication is enabled, the software can configure
which pools will receive broadcasts, which multicasts and so on. This mode is
better suited for use with \a{SR-IOV}.

\subsection{Multiple Transmit Queues}

For transmission, the controller also opens 128 queues. However, not all of
them are scheduled in every configuration. Depending on the use of Virtualization
and \a{DCB}, only 64 queues might be dequeued.

Either way, two scheduling phases are performed. First, the packets are
dequeued from the 128 queues to at most 8 packet buffers. Then, the packets are
taken out from the buffers and sent to the \a{MAC} for transmission.

Queues are distributed between queue pools (virtual functions) and traffic
classes. When there are multiple queues dedicated to one traffic class inside
a pool, they are always dequeued in a frame-by-frame round-robin order. Queues
are distributed between classes to reflect class priority 
(high-priority classes have fewer queues than low-priority classes, because less
traffic is expected to be buffered.)

The scheduling algorithms differ for every mode of operation, and their
description here would be a mere copy of \cite{82599}, section 7.2.1, and
therefore is omitted.

From the offloading point of view, in majority of modes the controller employs
a weighted-strict-order scheduling of different traffic classes. This fact
could be used to offload priority scheduler with an appropriate number of bands.

\subsection{Other offloads}

Apart from the classification offloads mentioned, the controller features two
security offloads: LinkSec (MACsec) and IPSec. For both of them, the software
must establish the Security Associations itself, and then install them to the
hardware tables. The controller is capable of offloading AES-128. Both
authentication and encryption is supported, provided they are using the same
\a{SA} (IPSec only). \sect{7.8 and 7.12}

Another inline functionality that is emphasized in the specification is the
\a{FCoE} support. The controller offers additional functions like Fibre
Channel \a{CRC} computation or \a{FCoE} segmentation. \sect{7.13}
