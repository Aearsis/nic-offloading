\section{Chelsio Terminator 6}

% Refused to give specs.
% 
% driven by cxgb4
% 
% Good source: DPDK commit http://dpdk.org/dev/patchwork/patch/10340/
% 
% U32: https://www.spinics.net/lists/netdev/msg396215.html
% 
% conjecture: filters share memory with TOE, because TIDs ~= filters
% 
% adapter 1:1 sge 1:N queues
%     |__ 1:N TDI 1:1 filter / ...
% 
% classification: fixed parser
% 	- struct ch_filter_tuple
% 
% actions:  static int fill_action_fields()
% 	- gact shot --> drop
% 	- mirred --> switch (only if the port belongs to the same card obviously)
% 		- optionally, rewrite headers == NAT
% 
% Check cxgbtool
% https://lwn.net/Articles/542643/ Add support for Chelsio T5 adapter
% 
% --- 
% 

The controller ships in many variants, all of which enable a subset of
functions. While that probably allows Chelsio to set a more suitable pricing,
it is quite confusing. For example, we cannot say for sure hat any two features
described here can be used at the same time.

The architecture of the \a{ASIC} is best described in a paper published by
Chelsio\cite{chelsio-t6}, it does not however go into much detail. As the
vendor refused to share any information with us, we have to resort to guessing.
A lot of details are exposed in the manual for the proprietary driver extension
and configuration utility\cite{chelsio-uw}. Still, both these manuals only
explain the capabilities of the controller from a user's point of view, not the
hardware capabilities in detail. For this kind of information, we had to
carefully examine the source code of the Linux kernel driver.

The driver in linux kernel, \sw{cxgb4} is shared for all controllers from the
Terminator series, starting from Terminator 4. Therefore, we know the union of
the capabilities of all these controllers, because we cannot differentiate
them. As we do not expect features to be removed in newer versions, we suppose
that the current state reflects the capabilities of Terminator 6.

As usual, \a{SR-IOV} is supported by the controller. The controller handles up
to 256 virtual instances, mapped to virtual or physical functions. Even though
the controller can identify itself as 8 physical functions, only the first
4 are capable of \a{SR-IOV}. With the factory firmware, each of them can
control 16 virtual functions. However, some controller variants (probably
differing only in firmware) can be configured up to 62 virtual functions per
each, giving us the maximum of 256 functions in total. Unfortunately, neither
the available documents nor the source code gives hints about the inner switch
functionality.

\subsection{Stateless Offloads}

The controller supports all of the mentioned stateless offloads. An interesting
requirement is placed on the software in case of \a{IPv4} header checksum for
\a{LSO} of \a{MAC}-in-\a{UDP} tunnels -- the driver has to compute the checksum
of the whole \a{IPv4} header without the total length field. We can only guess
why, as the controller should only fragment \a{UDP} datagrams, not create new
ones.

For the receive checksum, the controller is capable of computing the checksum
of a received packet in a generic way, as described in section
\ref{sec:linux-csum}. It does so as the first controller from the examined
ones.

As for the tunnelling support, it seems that \a{VXLAN} and Geneve is supported
for both offloads. Even though \a{GRE} is defined as one of the constants for
supported tunnels, it is not used elsewhere in the driver code.

\subsection{Match-Action Offload}

There is a step in the pipeline where custom classification is performed and
programmable actions are taken. The controller always contains a TCAM to
install up to 496 rules, and optionally also a memory dedicated for half
a million hashed rules. The action part might seem a bit constrained, but is
pretty powerful.

The match-action pipeline starts with a fixed parser, extracting a similar set
of fields we have already seen -- \a{MAC} addresses, ethertype, 2 layers of
\a{VLAN}s, \a{IP} addresses, \a{TCP} and \a{UDP} ports. In the kernel, all the
available fields are represented in the structure
\linuxstruct{drivers/net/ethernet/chelsio/cxgb4/cxgb4.h}{1009}{ch\_filter\_tuple}.

From these fields a compressed vector is constructed to be matched on. The
compressed vector contains only the \a{IP} addresses and L4 ports, extended with up
to 36 bits of the fields from above. It is very important to note that the
selection of bits creating the match vector is global, and changeable only with
the driver being reloaded.

The compressed vector is used to look up an entry in a table, be it a \a{TCAM} or
hash table. In both cases, the software representation of the entry is a structure
\linuxstruct{drivers/net/ethernet/chelsio/cxgb4/cxgb4.h}{1042}{ch\_filter\_specification}.
From its layout, we can clearly see all the possibilities.

An interesting difference from other vendors is that there is a fixed number of
slots for filters, and the slots are allocated by the software. The order of
filters is not arbitrary, as filters with lower indices have higher priority.

For matched flows, exactly one of three actions can be taken: pass, drop or
switch. When the packet is passed, its receive queue can be selected (otherwise
it is selected by \a{RSS} automatically). The possibilities open with the switch
action. In the beginning, switch action instructs the controller to loopback
the packet to a port. However, it can modify some header fields along. Namely,
it can alter \a{MAC} addresses, push, change, or pop VLAN tag and/or modify \a{IP}
addresses and \a{L4} ports. That means the controller is capable of offloading
e.g.\ routing with stateless \a{NAT}.

\subsection{Other Offloads}

Chelsio emphasizes a lot of other offloads the controller supports. The range
of additional functions the Unified Wire Adapter covers:

\begin{itemize}
\item \a{TCP}\acrfull{TOE} % HACK
\item \acrfull{RDMA} over \a{iWARP} offload
\item Both target and initiator offload of \a{iSCSI} and \a{NVMe-oF}
\item \a{FCoE} initiator offload
\item Crypto offloads (IPSec, \a{TLS}, \a{SMB}, \dots)
\item Open vSwitch offload
\item Bonding and active failover offload
\end{itemize}

If the \a{NIC} driver had to support all of these, the driver would have to be
an ugly hybrid driver spread across the whole kernel. It is not the case,
instead, the upper-layer drivers (such as \a{iSCSI} kernel implementation)
communicate with the controller through the \a{NIC} driver using a network-like
protocol, therefore the \a{NIC} driver itself does not care much about the
extended features.
